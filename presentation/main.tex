\documentclass[10pt,xcolor={table,dvipsnames},c]{beamer}
\usepackage{todonotes}
\usepackage[normalem]{ulem}
\newcommand{\soutthick}[1]{%
    \renewcommand{\ULthickness}{2pt}%
       \sout{#1}%
    \renewcommand{\ULthickness}{.4pt}% Resetting to ulem default
}
\newcommand\hl{\bgroup\markoverwith
  {\textcolor{yellow}{\rule[-.5ex]{1pt}{2.5ex}}}\ULon}
\usetheme{UoE}

\usepackage{xcolor}

% \definecolor{ediblue}{RGB}{22,39,74} 
\definecolor{ediblue}{RGB}{0,79, 113} 
% \definecolor{edired}{RGB}{172,0,64} 
\definecolor{edired}{RGB}{214,30,56} 
\definecolor{edipurple}{RGB}{208,0,111} 


\definecolor{figcaption}{RGB}{58, 59, 60} 


\usepackage[most]{tcolorbox}
\newtcolorbox{summary}[2]{
	colback=ediblue!5!white,colframe=ediblue!75!black,fonttitle=\bfseries\rmfamily, title={#1}, breakable}
\newtcolorbox{proposal}[1]{
	colback=edired!5!white,colframe=edired!75!black,fonttitle=\bfseries\rmfamily, title={#1 Proposal}, breakable}

\tcbset{
	highlight math style={
			enhanced,%<-- needed for the `remember' options
			colframe=edired,
			colback=edired!10!white,
			boxsep=0pt},
	%
	every box on layer 2/.style={ % <---- applied to the nested boxes on layer 2
			highlight math style={
					enhanced,%<-- needed for the `remember' options
					colframe=edired,
					colback=edired!10!white,
					boxsep=0pt}}}

\usepackage{sectsty}
\usepackage[LGR, T1]{fontenc}
\usepackage{CrimsonPro}
\usepackage[default]{sourcesanspro}
\allsectionsfont{\rmfamily}


\usepackage{amsmath} % for \dfrac
\usepackage{bm} % \bm
\usepackage{physics}
\usepackage{tikz,pgfplots}
\usepackage[outline]{contour} % glow around text
\usetikzlibrary{angles,quotes} % for pic (angle labels)
\usetikzlibrary{calc}
\usetikzlibrary{decorations.markings}
\usetikzlibrary{patterns,snakes}
\tikzset{>=latex} % for LaTeX arrow head
\contourlength{1.6pt}
\usepackage{xcolor}
\colorlet{Bcol}{violet!90}
\colorlet{BFcol}{red!60!black}
\colorlet{Scol}{green!60!black}
\colorlet{veccol}{green!45!black}
\colorlet{Icol}{blue!70!black}
\colorlet{mucol}{red!90!black}
\tikzstyle{BField}=[->,thick,Bcol]
\tikzstyle{current}=[->,Icol] %thick,
\tikzstyle{force}=[->,thick,BFcol]
\tikzstyle{vector}=[->,thick,veccol]
\tikzstyle{mu vector}=[->,thick,mucol]
\tikzstyle{spin}=[->,very thick,Scol]
\tikzstyle{velocity}=[->,very thick,vcol]
\tikzstyle{charge+}=[very thin,draw=black,top color=red!50,bottom color=red!90!black,shading angle=20,circle,inner sep=0.5]
\tikzstyle{charge-}=[very thin,draw=black,top color=blue!50,bottom color=blue!80,shading angle=20,circle,inner sep=0.5]
\tikzstyle{metal}=[top color=black!15,bottom color=black!25,middle color=black!5,shading angle=10]
\tikzset{
  BFieldLine/.style={thick,Bcol,decoration={markings,mark=at position #1 with {\arrow{latex}}},
                                 postaction={decorate}},
  BFieldLine/.default=0.5,
  pics/magnet/.style={ %args={#1}
    code={
      \def\h{0.9}
      \coordinate (-N) at (0,\h);
      \coordinate (-S) at (0,-\h);
      \draw[pic actions,thick,top color=red!60,bottom color=red!90,shading angle=20]
        (-0.8*\h/2,0) rectangle ++(0.8*\h,\h);
      \draw[pic actions,thick,top color=blue!60,bottom color=blue!90,shading angle=20]
        (-0.8*\h/2,0) rectangle ++(0.8*\h,-\h);
      \node[pic actions] at (0, \h/2) {\textbf{N}};
      \node[pic actions] at (0,-\h/2) {\textbf{S}};
  }},
}



\usepackage{tabularx}
\setbeamercolor{text}{fg=uoedarkblue}
\setbeamercolor{normal text}{fg=uoedarkblue}
\setbeamercolor{frametitle}{fg=uoedarkblue}
\usebeamercolor[fg]{normal text}
\title[]{Quantum Spintronics}
\subtitle{Multimodal Spin Based Sensors}
\author{Conner Adlington}
\institute{School of Physics and Astronomy \\ University of Edinburgh}
\date{}



%%%%%%%%%%%%%%%%%%%%%%%%%%%%%%%%%%%%%%%%NOTES
%\setbeameroption{hide notes} % Only slides
% \setbeameroption{show only notes} % Only notes
% \setbeameroption{show notes on second screen=right} % Both

\begin{document}

\begin{frame}
	\titlepage
	\note[item]{Introduce yourself, and your programme.}
	\note[item]{Introduce the project topic}
\end{frame}
% \begin{frame}{Remaining ToDos}
%     \small
%     \listoftodos
% \end{frame}


\chapter{Introduction}

% The Introduction should contain a description of your project and the
% problem you are trying to solve. It should start off at a level that
% should be understandable by anyone with a degree in physics, but it
% can become more technical later
%
% Where appropriate you should include references to work that has
% already been done on your topic and anything else which lets you set
% your work in context.
%
% One of the things you will need to do is to ensure that you have a
% suitable list of references.  To do this you should see \cite{Simon2013-lh}
% or some other suitable reference.  Note the format of the citation used
% here is the style favoured in this School.  Here is another
% reference \cite{Simon2013-lh} for good measure.
%
%
% You will also want to make sure you have no spelling or grammatical
% mistakes. To help identify speling mistukes you can use the commands
% \emph{aspell}, \emph{ispell} or \emph{spell} on most Linux/Unix
% computers. See the appropriate manual pages. Remember that spelling
% mistakes are not the only errors which can occur. Spelling checkers
% will not find errors which are, in fact, valid words such as
% \emph{there} for {\em their}, nor will they find repeated repeated
% words which sometimes occur if your concentration is broken when
% typing. \textbf{There is no substitute for thorough proof reading!}
%
% Your dissertation should be no longer than 15,000 words. In terms of
% pages, 30 pages are ok. 50 pages are fine. But it shouldn't be
% much longer than that.
%
%

% \section{Defect Orientation}

Colour centres or defects in general are part of the crystal lattice and thus have an associated orientation and direction within the lattice. This allows the definition of a \textbf{defect axis}. For example, in diamond the NV axis is defined as the vector from the vacancy towards the Nitrogen atom when the vacancy is taken as the origin of your co-ordinate system. 

In a tetragonal crystal, due to symmetry there are four possible orientations of a defect within the lattice: $111$, $1\overline{11}$, $\overline{1}1\overline{1}$ and $\overline{11}1$ directions. 

\section{Miller Indices}
The notation for defect orientation above is known as a Miller Index, and we consider the $111$ direction to be aligned with the defect axis. 

This means that if we know the orientation of our crystal then we can establish the orientations of the defect axis inside. For example, using a crystal for which all surfaces belong to the $\{001\}$ lattice planes, each surface normal is aligned with a Cartesian axis. Thus, by fixing the crystal in place, there remain just \textbf{four} possible angles which a defect axis can have with respect to the crystal surface.

Calculating the scalar product of any of the surface planar directions in the family of $\{001\}$ and the four possible orientations of the defect within the lattice we find $\cos\theta = \pm0.6$. Then, considering the physical solutions (from $0, 2\pi$) gives four possible angles that the (directed) defect axis may make with the surface of the crystal: $53.13^\circ$, $306.87^\circ$, $126.9^\circ$ and $233.13^\circ$ ($0.927$, $5.355$, $2.214$ and $4.069$ radians respectively). 






    


% \section{Spintronic Magnetometry}
The Hamiltonian, and thus the energy, of a spin is sensitive to the magnetic field because of the Zeeman interaction. 

How the electron Zeeman energy varies with magnetic field is known to a very large precision. 
Therefore, by measuring the energy difference we may determine the magnetic field. This is the mechanism which enables us to use the spin of an electron as a magnetic-field sensor. 

In practice, for example with diamond a fluorescence microscope to measure the electron spin resonances of an ensemble of NV centres. This allows the determination of both magnitude and direction of an external magnetic field. 


\subsection{Applied Magnetic Field}
To use defects as magnetometers, we must understand the nature of their spin states when an external magnetic field is applied. From this we may determine both the amplitude and direction of the external magnetic field from the electron spin resonance frequencies of the defect. 

\subsection{Spin-1 Defect}
A spin-1 defect has $S=1$ electron spin. Therefore, it has $3$ possible spin states, $m_S=0,−1,+1$. 

With no external applied magnetic field, in general, the $m_S  = \pm1$ states are degenerate, that is they have the same energy. Applying a magnetic field lifts the degeneracy and the $m_S=\pm1$ states will have different energies, $E_u$ and $E_l$ (subscripts refer to "upper" and "lower"). These are equivalent to transition frequencies by $E=hf$, which we denote $f_u$ and $f_l$.

The exact values of these frequencies are functions of both amplitude and direction of the magnetic field, specifically the cosine of the angle between the applied magnetic field and the defect axis $\theta$.

Thus, by experimentally determining the transition frequencies, the magnitude and relative angle of the applied magnetic field may be determined. Details of the derivation are included in \ref{system_hamiltonian} and we find we may determine 
\begin{equation}
\gamma B = \frac{1}{3} \sqrt{f_u^2 + f_l^2 - f_uf_l - D^2}
\end{equation}

\begin{equation}
    \cos^2 \theta = \frac{-(f_u + f_l)^3 + 3 f_u^3 + 3 f_l^3}{27 D (\gamma B)^2} + \frac{2D^2}{27(\gamma B )^2} + \frac{1}{3}
\end{equation}

Here $\gamma = 28 \ce{GHz/T}$ is the gyromagnetic ratio of the electron. $D = 2.87 \ce{GHz}$ is the zero-field splitting of the defect ground state, that is the energy difference between $m_S = 0$ and $m_S = \pm 1$ with no external field applied. 

The simplest possible case is when the defect axis aligns with the applied field for which we get a linear relationship 
\begin{equation}
    f_u = D + \gamma B \qquad f_l = D - \gamma B.
\end{equation}


\missingfigure{Include plot of the electron spin resonances vs applied magnetic field.}





% \section{Spectroscopy}


%% GENERAL GOOD REFS (Added to bib)
% Structural Analysis of Point Defects in Solids
% Point Defects in Semiconductors and Insulators: Determination of Atomic and Electronic Structure from Paramagnetic Hyperfine Interactions
% Electron Paramagnetic Resonance: Elementary Theory and Practical Applications

% \td{Need to make sure I find something from this one to avoid reference padding}
% Introduction to Magnetic Resonance with Applications to Chemistry and Chemical Physics


Solid-state colour centres, which exists in many materials such as diamond and silicon carbide, have been one of the leading systems in quantum technology
\cite{Son2020, Awschalom2018}. 
The nitrogen-vacancy (NV) centre in diamond is the most comprehensively studied solid-state spin defect. The defect spin state can be initialized by laser and controlled by microwave \cite{Zhang2020, Atatre2018, Schirhagl2014}. It has been used in various quantum technologies, such spin–photon entanglement, a quantum computing qubit register and high-sensitivity nanoscale quantum sensing, the focus of this work \cite{Hensen2015, PhysRevX.9.031045}. 


The NV centre is favoured for it's for its excellent quantum properties, but
% which include a strong optically detected magnetic resonance (ODMR) contrast and long coherence time in the room temperature regime
drawbacks of the system are a lack of established nanotechnology and the fluorescence wavelength of the NV centre, which is in the visible range and limits its wider applications \cite{Koehl2011, Christle2014, Widmann2014} 

% This is a good resource 
% https://www.frontiersin.org/journals/physics/articles/10.3389/fphy.2023.1270602/full

The field of spectroscopy studies the way atoms and molecules interact with and exchange energy with a wider physical system - specifically through electromagnetic radiation. The electric field interacts with with the electric dipole moment and the magnetic field interacts with a magnetic dipole moment.
Magnetic resonance spectroscopy focusses specifically on the interaction between the $\vec{B}$ field with magnetic moments which exist in a given material. This can be broken into two distinct fields:

\begin{description}
	\item [Nuclear Magnetic Resonance (NMR)] which studies the interaction with nuclear magnetic moments.
	\item [Electron Paramagnetic Resonance (EPR)] which studies the interaction with electron spin systems.
\end{description}

Using Planck's relationship $E = h \nu$  and $c = \lambda \nu$ we may characterise the electromagnetic radiation by its energy which is, to a constant, equivalent to the frequency or the wavelength. We see this resonance in systems where the electrons are influenced by an driving magnetic field forcing transitions between spin system energy levels. In general the measurable difference in energy levels for which the transition occurs is caused by an external magnetic field via the Zeeman effect. Depending on the symmetry of the spin system, some also exhibit energy level splitting with no applied magnetic field so called zero field splitting (ZFS).

EPR is thus a tool to manipulate electron spins in solid state materials. The transition between energy levels is quantised thus the discrete amount of energy which is lost by the system is transferred into a photon which may be detected optically \cite{carrington1967introduction}.

A particularly successful technique is optically detected magnetic resonance (ODMR) which uses an applied microwave frequency, an oscillating magnetic field with energy quanta equivalent to the transitions between spin sub-levels, to drive the repopulation of those sub-levels following a spin-preserving radiative transition.
In essence this boosts the sensitivity since the microwave driven repopulation induces a change in photoluminescence with a much higher and thus much more readily detectable energy. The techniques of ODMR are so effective that even a single electron spin may be detected this way \cite{Khler1993}.

Spintronics, a portmantau of \textbf{spin} and elec\textbf{tronics} is a technology which exploits the characteristics of spin akin to how charge is manipulated in electronics. Fundamentally, the smallest stable magnetic moment available in nature is generated by the spin of a single electron. If efficient read-out can be achieved, the sensitivity of the electron magnetic dipole cannot be matched. 
Careful construction of an appropriate system, or identification of a system with appropriate characteristics allows for the initialisation, manipulation and read-out of EPR from which we may infer the physical properties of the environment surrounding the system. 

% The ability to efficiently control spin states is the goal of semiconductor spintronics.  
% The properties of nitrogen-vacancy (NV) colour centres in diamond have catalysed major development in the field.
With ODMR of the NV centre in diamond the manipulation of individual atomic sized defects at room temperature has been demonstrated \cite{Levine} despite spin polarisation being a primarily thermodynamic effect (see section \ref{spin_polarisation}).
This is possible since optical excitation of the energy levels decay faster via a spin-preserving transition, leading to a spin polarisation to a in the ground state, provided the system is constantly irradiated for long enough for the system to reach a steady state.

Silicon carbide (SiC) is proving to be an excellent material for application in this space (discussed in detail in section \ref{SiC}). A major benefit of SiC is the existence of various polytypes, which each exhibit unique spin colour centre properties. Furthermore, even within a single polytype, these centres can occupy distinct and non-equivalent lattice positions \cite{Kimoto2014}.
The existence of these colour centres with similar properties but different energy quanta allows for selection of a specific defect with parameters suitable for the problem at hand. 

% EPR spectroscopy can be approached by different methods, relevant to this work:
% % \tdg{Consider writing a paragraph on ENDOR - only if relevant later in the project.}
% \begin{description}
% 	\item [Continuous Wave (CW)] where the magnitude of the static magnetic field ($B_0$) is swept, while the
% 	      amplitude of the driving field $B_1$ is constant with time.
% 	% \item [Pulsed] where a time-dependent driving pulse $B_1$ is applied in
%         % addition to a static magnetic field $B_0$ \cite{Baranov2017-bv}.
% 	      % \td{Need to write a section on using Pulsed EPR to measure relaxation timescales?}
%
%
%     \item[Electron-Electron Double Resonance (ELDOR)] where two microwave frequencies participate;
%         \begin{enumerate}
%             \item The “pump” microwave source, irradiates a portion of the ESR spectrum. 
%             \item The effect of this irradiation on another portion of the spectrum is monitored by an observe microwave source. \cite{Berliner2011-ww}
%
%
%         \end{enumerate}
% \end{description}

This work looks to explore how the physical characteristics which influence the Spin Hamiltonian and thus the energy of the electron spin system may be inferred by measuring the effects of those characteristics on the EPR of that system. Further, it will look to explore whether the compound effect of multiple influences may be disentangled and measured simultaneously - so called multi-modal sensing.

Chapter \ref{ch:background} gives a thorough overview of the science behind EPR, SiC, quantum sensing and the techniques of ODMR. 
In Chapter \ref{ch:design} we show in detail how different parameters may be sensed using different spin systems. We clearly summarise each approach and list the constraints of each technique. With a clear understanding of what is possible, in Chapter \ref{ch:results} we present the combinations of sensing techniques which may be applied simultaneously - achieving the goal of a multimodal sensing schema. Finally we conclude in Chapter \ref{ch:conclusions} and discuss the implications of the multimodal techniques developed as well as what future work needs to be done to realise other multimodal systems.  


\section{Spintronics}

\begin{frame}{Spintronic Devices}
	\begin{columns}[c]
		\begin{column}{0.25\textwidth}
			\begin{tikzpicture}
				[scale=2, scale line widths, transform shape, every node/.style={scale=0.3}]

				\def\re{0.3}
				\def\ang{80}
				%\draw[dashed] (\ang-180:2.5*\re) -- (\ang:3.5*\re);
				\draw[mu vector] (0,0) -- (\ang+180:3*\re) node[right] {$\vb*{\mu}_\mathrm{B}$};
				\draw[spin] (0,0) -- (\ang:2.8*\re) node[right] {$\vb{S}$};
				\draw[charge-]
				(0,0) circle (\re) node[scale=1.4] {$-$}
				node[right=10] {
						% $\mathrm{e}^-$
					};
				\draw[->,rotate=\ang-90]
				(0,-0.2*\re)++(-175:{1.6*\re} and {1.3*\re}) arc (-175:-35:{1.6*\re} and {1.3*\re})
				--++ (50:0.3*\re);
			\end{tikzpicture}

		\end{column}
		\begin{column}{0.75\textwidth}


			\begin{center}

				{\Large\rmfamily \textbf{spin - tr{\color{gray} ansport} - {\color{gray}electr}onics}}
				\pause
				\begin{itemize}
					\item Exploit spin in the same way electronics exploit charge
				\end{itemize}

			\end{center}
		\end{column}
	\end{columns}
\end{frame}






\section{Background}
\begin{frame}{Background}
\end{frame}


\section{Sensing Regimes}
\subsection{Magnetometry}
\begin{frame}{$S=1$ Magnetometry}
	\begin{summary}{$S=1$ Magnetometry Summary}{}
		\small
		\begin{enumerate}
			\item We can resolve \textbf{two frequencies} corresponding to the defect in the CW-ODMR spectra.
			      % \item We know the ZFS parameters $D$ and $E$.
			\item The ZFS parameters $D$ and $E$ are well known.
			\item We can determine the magnitude using
			      \begin{equation*}
				      \tcbhighmath{B = \frac{\sqrt{\frac{1}{3} \left(f_1^2 - f_1 f_2 + f_2^2 -D^2 -3E^2\right)}}{g \mu_B}.}
				      % \tcbhighmath{g\mu_b B = \frac{f_1 - f_2}{2 \gamma}}
				      % \tag{\ref{eq:s1_parallel_magnetometry}}
			      \end{equation*}
			\item We can determine the azimuthal angle using
			      \begin{equation*}
				      \tcbhighmath{
					      \theta = \frac{\cos^{-1} (\eta/D)}{2}.
				      }
			      \end{equation*}
		\end{enumerate}
	\end{summary}
\end{frame}
\subsection{Electrometry}
\begin{frame}{$S=1$ Electrometry}
	\begin{summary}{$S=1$ Electrometry Summary}{sum:spin1electro}
		\small
		\begin{enumerate}
			\item We can resolve \textbf{two frequencies} corresponding to the defect in the CW-ODMR spectra.
			\item The direction and magnitude of $\vec{B}$ and the ZFS parameters $D$ and $E$ are well known.
			\item In general 
                % the shift of EPR frequencies due to the applied electric field is given by
			      \begin{equation*}
				      \tcbhighmath{
					      \Delta f _\pm = d_\parallel E_z \pm \left(F(\vec{B},\vec{E},\vec{\sigma}) - F(\vec{B},0,\vec{\sigma})\right)
				      }
				      % \tcbhighmath{g\mu_b B = \frac{f_1 - f_2}{2 \gamma}}
				      % \tag{\ref{eq:s1_parallel_magnetometry}}
			      \end{equation*}
			      % Sensitivity is maximised when $\vec{B}$ is applied perpendicular to the defect axis.

			      % We can determine the vector by fitting $\theta$ and $\varphi$ to $\Delta f$ eliminating the ambiguity by applying reference fields and repeating measurements.
			% \item Despite the reduction in sensitivity, applying $\vec{B}$ parallel to the defect axis we may reduce the Hamiltonian to find the magnitude and azimuthal angle as
              \item With $\vec{B}$ parallel to the defect axis we have
			      \begin{equation*}
				      \tcbhighmath{
					      \theta = \tan^{-1} \left(\frac{\mathcal{E}_\parallel}{\mathcal{E}_\perp}\right), \quad \mathcal{E} = \sqrt{\mathcal{E}_\perp^2 + \mathcal{E}_\parallel}.
				      }
			      \end{equation*}
		\end{enumerate}

	\end{summary}
\end{frame}



\section{Multimodality}
\begin{frame}{Multimodality}
\end{frame}
\subsection{$\vec{B}$ and $T$} 
\begin{frame}{$\vec{B}$ and $T$}
\end{frame}
\section{Summary}
%
% \begin{frame}
% 	\begin{center}
% 		\Huge Summary
% 	\end{center}
% \end{frame}
%
%
% \begin{frame}{Relaxation Time - $T_1$}
% % the relaxation time T1 (or spin-lattice relaxation) corresponds to the time it takes a
% % polarised spin to decay into a thermal equilibrium state. At high temperature, this would
% % correspond to the process:
% %
% % 1 1 0
% % 0 0
% % ρ=
% % −→ ρ =
% % (112)
% % 0 1
% % 2 0 1
% % . At sufficiently low temperature and high magnetic field, such that the electron spin is
% % thermally polarised into | ↓i, the relaxation process corresponds to:
% %
% % 0 0
% % 1 0
% % ρ=
% % −→ ρ =
% % (113)
% % 0 1
% % 0 0
% % . The T1 timescale can be measured by polarising the spin, for example in | ↑i, and
% % measuring its state as a function of time t. The probability for the spin to stay in the | ↑i
% % state decays as e−t/T1 .
%     \todo[inline]{Copy summary points from pg 33}
% \end{frame}
%
% \begin{frame}{Dephasing Time - $T_2^*$}
%     \todo[inline]{Copy summary points from pg 33}
% \end{frame}
%
% \begin{frame}{Spin Echo Time - $T_2$}
%     \todo[inline]{Copy summary points from pg 33}
% \end{frame}
%
\begin{frame}
	\begin{center}
		\Huge So what?
	\end{center}
\end{frame}

% {
% \setbeamercovered{transparent}
% \begin{frame}{DiVincenzo Criteria}
%
% 	\begin{description}
%         \item[\textbf{Discrete, two-level system}] quantised
% 		\item[\textbf{Initialisable}] can be placed in a known starting state            
%     \item[\textbf{Coherent Manipulation}]\alert<+>{can be manipulated in coherent state}
% 		\item[\textbf{Efficient Readout}] final state can be easily measured
% 	\end{description}
% \end{frame}
% %
% % \begin{frame}{DiVincenzo Criteria}
% %
% % 	\begin{description}
% % 		\item[\textbf{Discreet, two-level system}] Min 2 states
% % 		\item[\textbf{Initialisable: }] Initialisable
% % 			\item[\textbf{Coherent Manipulation}: ]\alert<+>{System can be manipulated}
% % 		\item[\textbf{Efficient Readout: }]
% % 	\end{description}
% % \end{frame}
% }
%
% \begin{frame}
%     \begin{center}
%     \Large
%     Decoherence timescales inform the \textbf{working time} of a spintronic device.
%
%     \end{center}
% \end{frame}
%
%
%

\section{Conclusion}
% \begin{frame}{Conclusion (1 minute)}
% 	\begin{itemize}
%         \item Summarize the key points about leptoquarks and their potential significance.
%         \item Highlight the ongoing search efforts and the importance of future discoveries to unravel the mysteries of the subatomic world.
%         \item End with a call to action, encouraging further research and exploration in this exciting field.
%     \end{itemize}
% \end{frame}
%
\begin{frame}{Scope}
	\begin{itemize}
        \pause
        \item Motivation (SiC transistor (in place monitoring etc))
        \item Motivation2 : Microscope (As for diamond)
	\end{itemize}
\end{frame}


\begin{frame}
	\begin{center}
		\Huge Questions?
	\end{center}
\end{frame}





\end{document}
