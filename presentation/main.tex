\documentclass[10pt,xcolor={table,dvipsnames},c]{beamer}
\usepackage{todonotes}
\usepackage[normalem]{ulem}
\newcommand{\soutthick}[1]{%
    \renewcommand{\ULthickness}{2pt}%
       \sout{#1}%
    \renewcommand{\ULthickness}{.4pt}% Resetting to ulem default
}
\newcommand\hl{\bgroup\markoverwith
  {\textcolor{yellow}{\rule[-.5ex]{1pt}{2.5ex}}}\ULon}
\usetheme{UoE}

\usepackage{xcolor}

% \definecolor{ediblue}{RGB}{22,39,74} 
\definecolor{ediblue}{RGB}{0,79, 113} 
% \definecolor{edired}{RGB}{172,0,64} 
\definecolor{edired}{RGB}{214,30,56} 
\definecolor{edipurple}{RGB}{208,0,111} 


\definecolor{figcaption}{RGB}{58, 59, 60} 


\usepackage[most]{tcolorbox}
\newtcolorbox{summary}[2]{
	colback=ediblue!5!white,colframe=ediblue!75!black,fonttitle=\bfseries\rmfamily, title={#1}, breakable}
\newtcolorbox{proposal}[1]{
	colback=edired!5!white,colframe=edired!75!black,fonttitle=\bfseries\rmfamily, title={#1 Proposal}, breakable}

\tcbset{
	highlight math style={
			enhanced,%<-- needed for the `remember' options
			colframe=edired,
			colback=edired!10!white,
			boxsep=0pt},
	%
	every box on layer 2/.style={ % <---- applied to the nested boxes on layer 2
			highlight math style={
					enhanced,%<-- needed for the `remember' options
					colframe=edired,
					colback=edired!10!white,
					boxsep=0pt}}}

\usepackage{sectsty}
\usepackage[LGR, T1]{fontenc}
\usepackage{CrimsonPro}
\usepackage[default]{sourcesanspro}
\allsectionsfont{\rmfamily}

\input{diagrams/tikz_preamble.tex}
\usepackage{tabularx}
\setbeamercolor{text}{fg=uoedarkblue}
\setbeamercolor{normal text}{fg=uoedarkblue}
\setbeamercolor{frametitle}{fg=uoedarkblue}
\usebeamercolor[fg]{normal text}
\title[]{Quantum Spintronics}
\subtitle{Multimodal Spin Based Sensors}
\author{Conner Adlington}
\institute{School of Physics and Astronomy \\ University of Edinburgh}
\date{}



%%%%%%%%%%%%%%%%%%%%%%%%%%%%%%%%%%%%%%%%NOTES
%\setbeameroption{hide notes} % Only slides
% \setbeameroption{show only notes} % Only notes
% \setbeameroption{show notes on second screen=right} % Both

\begin{document}

\begin{frame}
	\titlepage
	\note[item]{Introduce yourself, and your programme.}
	\note[item]{Introduce the project topic}
\end{frame}
% \begin{frame}{Remaining ToDos}
%     \small
%     \listoftodos
% \end{frame}


\section{Introduction}

\begin{frame}{Scope}
	\begin{itemize}
        \pause
        \item Motivation (SiC transistor (in place monitoring etc))
        \item Motivation2 : Microscope (As for diamond)
	\end{itemize}
\end{frame}



\section{Spintronics}

\begin{frame}{Spintronic Devices}
	\begin{columns}[c]
		\begin{column}{0.25\textwidth}
			\begin{tikzpicture}
				[scale=2, scale line widths, transform shape, every node/.style={scale=0.3}]

				\def\re{0.3}
				\def\ang{80}
				%\draw[dashed] (\ang-180:2.5*\re) -- (\ang:3.5*\re);
				\draw[mu vector] (0,0) -- (\ang+180:3*\re) node[right] {$\vb*{\mu}_\mathrm{B}$};
				\draw[spin] (0,0) -- (\ang:2.8*\re) node[right] {$\vb{S}$};
				\draw[charge-]
				(0,0) circle (\re) node[scale=1.4] {$-$}
				node[right=10] {
						% $\mathrm{e}^-$
					};
				\draw[->,rotate=\ang-90]
				(0,-0.2*\re)++(-175:{1.6*\re} and {1.3*\re}) arc (-175:-35:{1.6*\re} and {1.3*\re})
				--++ (50:0.3*\re);
			\end{tikzpicture}

		\end{column}
		\begin{column}{0.75\textwidth}


			\begin{center}

				{\Large\rmfamily \textbf{spin - tr{\color{gray} ansport} - {\color{gray}electr}onics}}
				\pause
				\begin{itemize}
					\item Exploit spin in the same way electronics exploit charge
				\end{itemize}

			\end{center}
		\end{column}
	\end{columns}
\end{frame}






\section{Background}
\begin{frame}{Background}
\end{frame}


\section{Sensing Regimes}
\subsection{Magnetometry}
\begin{frame}{$S=1$ Magnetometry}
	\begin{summary}{$S=1$ Magnetometry Summary}{}
		\small
		\begin{enumerate}
			\item We can resolve \textbf{two frequencies} corresponding to the defect in the CW-ODMR spectra.
			      % \item We know the ZFS parameters $D$ and $E$.
			\item The ZFS parameters $D$ and $E$ are well known.
			\item We can determine the magnitude using
			      \begin{equation*}
				      \tcbhighmath{B = \frac{\sqrt{\frac{1}{3} \left(f_1^2 - f_1 f_2 + f_2^2 -D^2 -3E^2\right)}}{g \mu_B}.}
				      % \tcbhighmath{g\mu_b B = \frac{f_1 - f_2}{2 \gamma}}
				      % \tag{\ref{eq:s1_parallel_magnetometry}}
			      \end{equation*}
			\item We can determine the azimuthal angle using
			      \begin{equation*}
				      \tcbhighmath{
					      \theta = \frac{\cos^{-1} (\eta/D)}{2}.
				      }
			      \end{equation*}
		\end{enumerate}
	\end{summary}
\end{frame}
\subsection{Electrometry}
\begin{frame}{$S=1$ Electrometry}
	\begin{summary}{$S=1$ Electrometry Summary}{sum:spin1electro}
		\small
		\begin{enumerate}
			\item We can resolve \textbf{two frequencies} corresponding to the defect in the CW-ODMR spectra.
			\item The direction and magnitude of $\vec{B}$ and the ZFS parameters $D$ and $E$ are well known.
			\item In general 
                % the shift of EPR frequencies due to the applied electric field is given by
			      \begin{equation*}
				      \tcbhighmath{
					      \Delta f _\pm = d_\parallel E_z \pm \left(F(\vec{B},\vec{E},\vec{\sigma}) - F(\vec{B},0,\vec{\sigma})\right)
				      }
				      % \tcbhighmath{g\mu_b B = \frac{f_1 - f_2}{2 \gamma}}
				      % \tag{\ref{eq:s1_parallel_magnetometry}}
			      \end{equation*}
			      % Sensitivity is maximised when $\vec{B}$ is applied perpendicular to the defect axis.

			      % We can determine the vector by fitting $\theta$ and $\varphi$ to $\Delta f$ eliminating the ambiguity by applying reference fields and repeating measurements.
			% \item Despite the reduction in sensitivity, applying $\vec{B}$ parallel to the defect axis we may reduce the Hamiltonian to find the magnitude and azimuthal angle as
              \item With $\vec{B}$ parallel to the defect axis we have
			      \begin{equation*}
				      \tcbhighmath{
					      \theta = \tan^{-1} \left(\frac{\mathcal{E}_\parallel}{\mathcal{E}_\perp}\right), \quad \mathcal{E} = \sqrt{\mathcal{E}_\perp^2 + \mathcal{E}_\parallel}.
				      }
			      \end{equation*}
		\end{enumerate}

	\end{summary}
\end{frame}



\section{Multimodality}
\begin{frame}{Multimodality}
\end{frame}
\subsection{$\vec{B}$ and $T$} 
\begin{frame}{$\vec{B}$ and $T$}
\end{frame}
\section{Summary}
%
% \begin{frame}
% 	\begin{center}
% 		\Huge Summary
% 	\end{center}
% \end{frame}
%
%
% \begin{frame}{Relaxation Time - $T_1$}
% % the relaxation time T1 (or spin-lattice relaxation) corresponds to the time it takes a
% % polarised spin to decay into a thermal equilibrium state. At high temperature, this would
% % correspond to the process:
% %
% % 1 1 0
% % 0 0
% % ρ=
% % −→ ρ =
% % (112)
% % 0 1
% % 2 0 1
% % . At sufficiently low temperature and high magnetic field, such that the electron spin is
% % thermally polarised into | ↓i, the relaxation process corresponds to:
% %
% % 0 0
% % 1 0
% % ρ=
% % −→ ρ =
% % (113)
% % 0 1
% % 0 0
% % . The T1 timescale can be measured by polarising the spin, for example in | ↑i, and
% % measuring its state as a function of time t. The probability for the spin to stay in the | ↑i
% % state decays as e−t/T1 .
%     \todo[inline]{Copy summary points from pg 33}
% \end{frame}
%
% \begin{frame}{Dephasing Time - $T_2^*$}
%     \todo[inline]{Copy summary points from pg 33}
% \end{frame}
%
% \begin{frame}{Spin Echo Time - $T_2$}
%     \todo[inline]{Copy summary points from pg 33}
% \end{frame}
%
\begin{frame}
	\begin{center}
		\Huge So what?
	\end{center}
\end{frame}

% {
% \setbeamercovered{transparent}
% \begin{frame}{DiVincenzo Criteria}
%
% 	\begin{description}
%         \item[\textbf{Discrete, two-level system}] quantised
% 		\item[\textbf{Initialisable}] can be placed in a known starting state            
%     \item[\textbf{Coherent Manipulation}]\alert<+>{can be manipulated in coherent state}
% 		\item[\textbf{Efficient Readout}] final state can be easily measured
% 	\end{description}
% \end{frame}
% %
% % \begin{frame}{DiVincenzo Criteria}
% %
% % 	\begin{description}
% % 		\item[\textbf{Discreet, two-level system}] Min 2 states
% % 		\item[\textbf{Initialisable: }] Initialisable
% % 			\item[\textbf{Coherent Manipulation}: ]\alert<+>{System can be manipulated}
% % 		\item[\textbf{Efficient Readout: }]
% % 	\end{description}
% % \end{frame}
% }
%
% \begin{frame}
%     \begin{center}
%     \Large
%     Decoherence timescales inform the \textbf{working time} of a spintronic device.
%
%     \end{center}
% \end{frame}
%
%
%

\section{Conclusion}
% \begin{frame}{Conclusion (1 minute)}
% 	\begin{itemize}
%         \item Summarize the key points about leptoquarks and their potential significance.
%         \item Highlight the ongoing search efforts and the importance of future discoveries to unravel the mysteries of the subatomic world.
%         \item End with a call to action, encouraging further research and exploration in this exciting field.
%     \end{itemize}
% \end{frame}
%
\begin{frame}{Scope}
	\begin{itemize}
        \pause
        \item Motivation (SiC transistor (in place monitoring etc))
        \item Motivation2 : Microscope (As for diamond)
	\end{itemize}
\end{frame}


\begin{frame}
	\begin{center}
		\Huge Questions?
	\end{center}
\end{frame}





\end{document}
