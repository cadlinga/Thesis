\section{Summary}
%
% \begin{frame}
% 	\begin{center}
% 		\Huge Summary
% 	\end{center}
% \end{frame}
%
%
% \begin{frame}{Relaxation Time - $T_1$}
% % the relaxation time T1 (or spin-lattice relaxation) corresponds to the time it takes a
% % polarised spin to decay into a thermal equilibrium state. At high temperature, this would
% % correspond to the process:
% %
% % 1 1 0
% % 0 0
% % ρ=
% % −→ ρ =
% % (112)
% % 0 1
% % 2 0 1
% % . At sufficiently low temperature and high magnetic field, such that the electron spin is
% % thermally polarised into | ↓i, the relaxation process corresponds to:
% %
% % 0 0
% % 1 0
% % ρ=
% % −→ ρ =
% % (113)
% % 0 1
% % 0 0
% % . The T1 timescale can be measured by polarising the spin, for example in | ↑i, and
% % measuring its state as a function of time t. The probability for the spin to stay in the | ↑i
% % state decays as e−t/T1 .
%     \todo[inline]{Copy summary points from pg 33}
% \end{frame}
%
% \begin{frame}{Dephasing Time - $T_2^*$}
%     \todo[inline]{Copy summary points from pg 33}
% \end{frame}
%
% \begin{frame}{Spin Echo Time - $T_2$}
%     \todo[inline]{Copy summary points from pg 33}
% \end{frame}
%
\begin{frame}
	\begin{center}
		\Huge So what?
	\end{center}
\end{frame}

% {
% \setbeamercovered{transparent}
% \begin{frame}{DiVincenzo Criteria}
%
% 	\begin{description}
%         \item[\textbf{Discrete, two-level system}] quantised
% 		\item[\textbf{Initialisable}] can be placed in a known starting state            
%     \item[\textbf{Coherent Manipulation}]\alert<+>{can be manipulated in coherent state}
% 		\item[\textbf{Efficient Readout}] final state can be easily measured
% 	\end{description}
% \end{frame}
% %
% % \begin{frame}{DiVincenzo Criteria}
% %
% % 	\begin{description}
% % 		\item[\textbf{Discreet, two-level system}] Min 2 states
% % 		\item[\textbf{Initialisable: }] Initialisable
% % 			\item[\textbf{Coherent Manipulation}: ]\alert<+>{System can be manipulated}
% % 		\item[\textbf{Efficient Readout: }]
% % 	\end{description}
% % \end{frame}
% }
%
% \begin{frame}
%     \begin{center}
%     \Large
%     Decoherence timescales inform the \textbf{working time} of a spintronic device.
%
%     \end{center}
% \end{frame}
%
%
%
