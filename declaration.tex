\begin{center}
\textrm{\bfseries\Huge Declaration}%
\end{center}%
\vspace{1em}

I declare that this dissertation was composed entirely by myself. Chapter \ref{ch:background} gives a detailed overview of the science and does not contain original research. Chapter \ref{ch:design} covers the application of the science outlined in the previous chapter and is primarily a collection of already established techniques. Many of the calculations were made independently and verified by comparison to literature to ensure the consistency of this work. The multimodality section at the end of the chapter is entirely my own work and represents the approaches which warranted inclusion in this work. Many more attempts at combining schemas were made, but were unsuccessful and cannot be concisely included in this work. 
Then, in Chapter \ref{ch:results} is a review of the proposals made at the end of the previous chapter and is my own work.
Finally, \ref{ch:conclusions} summarises the work completed and discusses wider scientific context and possible directions for  
future work which are my own views. 

My computer modelling was done using Python where specifically the NumPy, MatPlotLib (Pyplot) and SymPy packages were used to numerically diagonalise and plot Hamiltonians and simulate ODMR spectra.  

% \lipsum[1]
% \todo[inline, color=red]{Need to write a declaration}
% Chapters 2 and 3 provide an introduction to the subject area and a
% description of previous work on this topic. They do not contain
% original research.
%
% Chapter~4 describes work that was carried out entirely by me. The results of
% this chapter have been obtained previously by Professor Anne T Matta of the
% University of Kinlochteuchter, but the methods used here are different
% in some important (or minor) ways.
%
% Chapters 4 through 6 contain my original work. The work described in
% Chapter~4 was carried out in collaboration with Professor Carole Ann O'Malley
% and her PhD student Jack O'Bean. Chapter~5 presents original work done
% entirely by me.
%
% \bigskip
%
% State whether calculations were done using Mathematica, Maple, MATLAB,
% SymPy, etc, with (or without) gamma-matrix manipulation code, master
% integrals, the Super-Duper software package, etc. In other words, you
% should refer to any software that you used during your project. For
% example, Monte Carlo simulation packages, hydrodynamics packages,
% measurement code, fitting code, tensor algebra and/or calculus packages,
% Feynman diagram evaluation packages, etc.
%
% State whether any software you used was written by you from scratch,
% by your supervisor (or by whoever), or if it's a standard package.
%
\vspace*{\fill}
\newpage
%
%
