\chapter{To Sort}
\section{Spin}
As well as the orbital magnetic moment generated by the orbital angular momentum of the electron, the electron also possesses an intrinsic magnetic moment. Classically this implies an intrinsic angular momentum hence the magnetic moment of elementary particles is termed \index{spin}. 

For a single electron spin may take the value $\pm 1/2$ since the system has only been observed in two possible states \cite{Gerlach1922} and experiments confirm that the orbital angular momentum and spin angular momentum are of the same nature and thus may be summed. 
The magnetic moment of the spin may thus be expressed as \eqref{eq:orbital_magnetic_moment_operator_bohr_magneton_g_factor} \cite{Povh2002-fj} where $g\approx2.0023$ \cite{electron-g-factor, PhysRevLett.130.071801}. 

In reality the electron is point-like and thus the current loop model is unsuitable. Spin is actually a purely quantum mechanical effect and a consequence of the algebra required to satisfy the Dirac equation of relativistic quantum mechanics. The manifestation of this degree of freedom however has the same dimensionality as $\vec{L}$, allowing us to work with the combination of $\vec{L}$ and $\vec{S}$. 

We thus consider the \index{total angular momentum} of a system $J$ given by 
\begin{equation}
    J = L + S 
    \label{eq:total_angular_momentum}
\end{equation}
which make take the values $L+S, L+ S - 1, \dots, |L-S|$. 




\section{Zeeman Effect}\label{zeeman}
When no magnetic field is applied to a system, the magnetic dipoles of the orbital electron and spin have no preferred direction. 
The energy levels for all combinations of $L$ and $S$ (all $J$) are equivalent. 

If a magnetic field is applied the magnetic moments interact with that field via the \index{Zeeman!Zeeman interaction}{Zeeman interaction}. 
The \index{Zeeman!Zeeman effect}{Zeeman effect} consists of atomic energy level splitting when an external magnetic field is imposed on a sample \cite{Nabokov2002}. 

The classical expression for the energy of a dipole in a magnetic field
\begin{equation}
    E = -\vec{\mu}\cdot\vec{B}
    \label{eq:}
\end{equation}
may be replaced with the Hamiltonian for a quantum mechanical system 
\begin{equation}
    \hat{H}_{\ce{Zeeman} = - \hat{\vec{\mu}}\cdot \vec{B}. 
    \label{eq:}
\end{equation}

The negative sign indicates that when the magnetic moment is parallel to the magnetic field the lowest energy is achieved. 

Thus distinct quantum systems with different $J$ and thus different projections of angular momentum ($m_J$) have different energies due to their interaction with a magnetic field. 

Considering a simple two-level system ($S=1/2$), the energy difference between the spin being aligned or anti-aligned with the field is called the Zeeman energy. 

The Hamiltonian to describe the energy is, using the total angular momentum form of \eqref{eq:orbital_magnetic_moment_operator_bohr_magneton_g_factor}, 
\begin{equation}
    \hat{H}_{\ce{Zeeman}} = g \mu_B \hat{\vec{S}}\cdot\vec{B}. 
    \label{eq:Zeeman_Hamiltonian}
\end{equation}

Without loss of generality we may direct the magnetic field along the $z$ axis and reduce the scalar product to only the $z$ component. Now, using $S=1/2$ quantised along the $z$ axis, i.e. $m_S = \pm 1/2$ we find the Zeeman energy by solving the Shr\"odinger equation 
\begin{equation}
    \hat{H}_{\ce{Zeeman}} \ket{S, m_S} = E_{\ce{Zeeman}}\ket{S, m_S} 
    % \label{eq:}
\end{equation}
which, to a factor is equivalent to, by \eqref{eq:zthcomponent}, to
\begin{equation}
    \hat{S}_{z} \ket{S, m_S} = m_S\ket{S, m_S}.
    % \label{eq:}
\end{equation}

Thus we find the two eigenvalues to be
$$E_+ =\frac{1}{2}g\mu_BB, \qquad E_-=-\frac{1}{2}g\mu_BB$$
and thus the Zeeman energy is given by $g\mu_B B$. 

The $S=1/2$ system is thus doubly \index{degeneracy!degenerate}{degenerate} and the \index{degeneracy}{degeneracy} is lifted by the application a magnetic field. The Zeeman energy is the difference between the two states and it grows linearly with $B$. 

This may be generalised to a more complex system by considering the total angular momentum $J$ where the energy difference between states is given by 
\begin{equation}
   \Delta E = g_J \mu_B B. 
    \label{eq:zeeman_energy}
\end{equation}






\input{Sections/HahnEcho.tex}
\input{Sections/Hyperfine.tex}
\input{Sections/SpinBaths.tex}
\section{Spin-Orbit Interaction}
The orbital magnetic dipole may interact with the intrinsic spin magnetic dipole via the \index{spin-orbit interaction}{spin-orbit interaction}. This is represented by the spin-orbit Hamiltonian with $\lambda$ representing the constant of the coupling: 
\begin{equation}
    H_{\ce{SO}} = \lambda \hat{\vec{L}}\cdot\hat{\vec{S}}. 
    \label{eq:spin_orbit_hamiltonian}
\end{equation}

This is caused by the interaction between the magnetic field generated by the relativistic motion of the electron around the nucleus and that of the spin magnetic moment. The coupling is proportional to the atomic mass. 


\subsubsection{Extending to Quantum Mechanics}
Since the gyromagnetic ratio was calculated considering the motion of dipole in a loop, we may extend this to an electron in an orbit within the atom. The fundamental change required to extend the model to quantum mechanics is the treatment of angular momentum which should now be quantised. 
Thus, we replace our classical approximation of $\vec{G} = \vec{r} \times \vec{p}$ with the equation for the eigenvalues of the quantum mechanical representation of orbital angular momentum:
\begin{equation}
    \hat{G} = \hbar \hat{J} 
    \label{eq:orbital_angular_momentum}
\end{equation}
where $\hat{J}$ is the operator of the orbital angular momentum (quantum number of orbital momentum). 


The angular momentum and total energy are conserved in general in a closed system\tdr{Write up or expand on Noether currents?}. 

We consider the time independent Shr\"odinger equation
\begin{equation}
    \hat{H} \Psi_n = E_n \Psi_n 
    \label{eq:TISE}
\end{equation}

We may choose $\Psi_n$ such that it is an eigenfunction of the Hamiltonian, the total angular momentum squared ($J^2 = J_x^2 + J_y^2 + J_z^2$) and exactly one directional component \td{Need to develop why?}of the angular momentum which is by convention chosen as $J_z$.

According to quantum mechanics the projection of $J$ along the quantisation axis ($M_J$) may take integer values $-J, -J + 1, \dots, J-1, J$. \td{Discuss why?}
Thus, we may describe a given quantum state by the spin $J$ and the projection of the spin $M_j$. Thus, using Dirac notation we may write 

\begin{eqnarray}
    &\hat{H}\ket{J, M_J} &= E\ket{J, M_J} \\ 
    &\hat{J^2}\ket{J, M_J} &= J(J+1)\ket{J, M_J} \\ 
    &\hat{J_z}\ket{J, M_J} &= M_L\ket{J, M_J}. \label{eq:zthcomponent} 
\end{eqnarray}

Thus, the operator which describes the orbital magnetic moment may be written as (using equations \ref{eq:gyromagnetic_ratio}, \ref{eq:orbital_angular_momentum})
\begin{equation}
    \hat{\vec{\mu}}_J = \gamma \hat{\vec{G}}_J = \gamma \hbar \hat{\vec{J}} = \frac{e\hbar}{2m_e c}\hat{\vec{J}}.
    \label{eq:orbital_magnetic_moment_operator}
\end{equation}

This leads to a quantity known as the \textbf{Bohr Magneton}, $\mu_B$, given by 
\begin{equation}
    \mu_B = \frac{|e|\hbar}{2m_e c}.
    \label{eq:bohr_magneton}
\end{equation}

Using this we may write equation \ref{eq:orbital_magnetic_moment_operator} as 
\begin{equation}
    \hat{\vec{\mu}}_J = -\mu_B\hat{\vec{J}}. 
    \label{eq:orbital_magnetic_moment_operator_bohr_magneton}
\end{equation}


\tdr{Change all J's above this to L's}



\subsection{g-factor}
The above expression is valid for the orbital electron but may be extended to a more general system by introducing a g-factor. The g-factor is equivalent to a dimensionless gyromagnetic ratio \cite{giancoli2008physics}, so equation \ref{eq:orbital_magnetic_moment_operator_bohr_magneton} may be written with $g=1$ as 
\begin{equation}
    \hat{\vec{\mu}}_L = -g\mu_B\hat{\vec{L}}. 
    \label{eq:orbital_magnetic_moment_operator_bohr_magneton_g_factor}
\end{equation}



\input{Sections/SpinCoupling.tex}
\section{Defect Orientation}

Colour centres or defects in general are part of the crystal lattice and thus have an associated orientation and direction within the lattice. This allows the definition of a \textbf{defect axis}. For example, in diamond the NV axis is defined as the vector from the vacancy towards the Nitrogen atom when the vacancy is taken as the origin of your co-ordinate system. 

In a tetragonal crystal, due to symmetry there are four possible orientations of a defect within the lattice: $111$, $1\overline{11}$, $\overline{1}1\overline{1}$ and $\overline{11}1$ directions. 

\section{Miller Indices}
The notation for defect orientation above is known as a Miller Index, and we consider the $111$ direction to be aligned with the defect axis. 

This means that if we know the orientation of our crystal then we can establish the orientations of the defect axis inside. For example, using a crystal for which all surfaces belong to the $\{001\}$ lattice planes, each surface normal is aligned with a Cartesian axis. Thus, by fixing the crystal in place, there remain just \textbf{four} possible angles which a defect axis can have with respect to the crystal surface.

Calculating the scalar product of any of the surface planar directions in the family of $\{001\}$ and the four possible orientations of the defect within the lattice we find $\cos\theta = \pm0.6$. Then, considering the physical solutions (from $0, 2\pi$) gives four possible angles that the (directed) defect axis may make with the surface of the crystal: $53.13^\circ$, $306.87^\circ$, $126.9^\circ$ and $233.13^\circ$ ($0.927$, $5.355$, $2.214$ and $4.069$ radians respectively). 






    


\input{Sections/SpinDetection.tex}
\input{Sections/SpinRelaxation.tex}
\input{Sections/DensityMatrices.tex}
\subsection{Magnetic Dipole}
Classically, the magnetic dipole may be modelled as a closed loop that carries an
electric current. 

Its \index{magnetic dipole moment}{magnetic dipole moment}, $\vec{\mu}$, is defined as the vector which points out of the plane
of the current loop, 
\begin{equation}
    \vec{\mu} = IS \vec{n}
    \label{eq:dipole_moment}
\end{equation}
where $I$ is the current in, and $S$ the surface area enclosed by, the loop. 

\begin{wrapfigure}{l}{0.5\textwidth}%
    \centering%
    % \includegraphics[width=0.38\textwidth]{figures/SiC-non-equiv-sites.pdf}%
        % B FIELD through current loop
\begin{tikzpicture}[scale=1.2, thick]
  \def\Rx{1.45}
  \def\Ry{0.43}
  \def\h{0.5}
  \def\H{3}
  \def\L{4}
  \def\NB{5}
  \def\ang{36}
  \coordinate (O) at (0,0);
  \coordinate (N) at (0,0.24*\H);
  \coordinate (M) at (0,0.45*\H);
  \coordinate (B) at (\ang:\H);
  
  % MAGNETIC FIELD
  \draw (-\Rx,0) arc (180:0:{\Rx} and {\Ry});
  \begin{scope}
    \clip ({-0.5*\L*cos(\ang)},-0.4*\H) rectangle ++({\L*cos(\ang)},\H);
    %\foreach \i [evaluate={\y=(\i-0.5)*\H/(\NB-0.5)/2;
    %                       \yl=-\H/2+(\i-0.5)*\H/(\NB-0.5)/2;}] in {1,...,\NB}{
    %  %\draw[BFieldLine,thin] (0,\y)++(\ang-180:0.5*\L) --++ (\ang:\L);
    %  %\draw[BFieldLine,thin] (0,-\y)++(\ang-180:0.5*\L) --++ (\ang:\L);
    %  \draw[BFieldLine,thin] (-\H/2,\y) -- ({-\H/2+(\H/2-\y)*cos(\ang)},\H/2);
    %  \draw[BFieldLine,thin] (-\H/2,-\y) -- ({-\H/2+(\H/2+\y)*cos(\ang)},+\H/2);
    %  \draw[BFieldLine,thin] ({\H/2-(\H/2+\yl)*cos(\ang)},-\H/2) -- (\H/2,\yl);
    %  \draw[BFieldLine,thin] ({\H/2-(\H/2-\yl)*cos(\ang)},-\H/2) -- (\H/2,-\yl);
    %}
    % \foreach \i [evaluate={\x=-0.31*\H+(\i-1)*0.62*\H/(\NB-1);
    %                        \y=-cot(\ang)*\x;
    %                        \a=0.50+0.017*\i}] in {1,...,\NB}{ %0.58-0.02*(\i-\NB/2-1)^2
    %   \draw[BFieldLine=\a] (\x,\y)++(\ang-180:\H) --++ (\ang:2*\H);
      %\fill[red] (\x,\y) circle (0.05);
    % }
  \end{scope}
  % \node[Bcol] at (\H/2,0.49*\H) {$\vb{B}$};
  
  % CIRCUIT
  \draw[white,very thick]
        (-\Rx,0) arc (-180:0:{\Rx} and {\Ry});
  \draw (-\Rx,0) arc (-180:0:{\Rx} and {\Ry});
  %\draw[white,very thick] (0,0) ellipse ({\R} and {0.3*\R});
  %\draw (0,0) ellipse ({\R} and {0.3*\R});
  %\draw (0,0) ellipse ({\R} and {0.3*\R});
  \draw[mu vector] (0,0) -- (M) node[above=-1, left=0] {$\vb*{\mu}$};
  \draw[vector] (0,0) -- (N) node[below=0,left=0] {$\vu{n}$};
  % \draw pic[->,"\small$\;\theta$",draw=black,angle radius=14,angle eccentricity=1.4]
    % {angle = B--O--N};
  \draw[white,very thick]
    (-150:{1.1*\Rx} and {1.16*\Ry}) arc (-150:-80:{1.1*\Rx} and {1.16*\Ry});
  \draw[current]
    (-135:{1.1*\Rx} and {1.16*\Ry}) arc (-135:-90:{1.1*\Rx} and {1.16*\Ry})
    node[midway,right=2,below] {$I$};
  
\end{tikzpicture}


  \caption{Schematic of current loop and induced magnetic moment.}%
\end{wrapfigure}%


The \index{magnetic dipole}{magnetic dipole} produces a magnetic field $\vec{B}$, which for points a large distance from the dipole may be calculated as \cite{Griffiths2012-pt}:
\begin{equation}
    \vec{B} = \frac{\mu_0}{4\pi} \frac{1}{r^3} \left[\frac{3(\vec{\mu} \cdot \vec{r}) \cdot \vec{r}}{r^2} - \vec{\mu}\right]
    \label{eq:}
\end{equation}

The symmetry of the field enables us to consider the direction of the dipole as aligned to the $z$-axis. Then, defining $x,y$ as usual by $r \cos\theta$ and $r \sin\theta$ respectively. We may decompose the \index{magnetic field}{magnetic field} in two separate components, parallel ($B_z$) and perpendicular ($B_x, B_y$): 
$$B_\parallel =\frac{\mu_0}{r^3}(3\cos^2 \theta - 1), \quad B_\perp = \frac{3\mu_0}{r^3}\cos\theta\sin\theta.$$
Where we use the Pythagorean principle to determine the overall magnitude $B = |\vec{B}|$ as
$$B = \sqrt{B_\parallel^2 + B_\perp^2}.$$

\input{Sections/RabiOscillations.tex}
\subsection{Gyromagnetic Ratio}
\subsubsection{Classical Derivation}
The current in \eqref{eq:dipole_moment} is directly proportional to the current (i.e. the angular momentum of the charge). Specifically, we note that without angular momentum $\vec{G} = \vec{r} \times \vec{p}$ where $\vec{r}, \vec{p}$ represent the radius and the momentum respectively, the dipole moment would be zero.

Dividing the magnetic dipole moment by the angular momentum we find the \textbf{\index{gyromagnetic ratio}{gyromagnetic ratio}} \cite{Chen2020}
\begin{equation}
    \gamma = \frac{\vec{\mu}}{\vec{G}}.
    \label{eq:gyromagnetic_ratio}
\end{equation}

Without loss of generality we may consider the most simple case, in which the magnetic dipole moment is aligned with the angular momentum for which we may consider only the magnitudes of the dipole moment and the current (angular momentum)
\begin{equation}
    \mu = IS, \quad I = 
    % \underbrace{\frac{q}{2\pi R}}_{\rho \text{ (charge density)}}v,
    \frac{qv}{2\pi R},
    \quad S = \pi R^2 
    % \label{eq:}
\end{equation}
we substitute $I$ and $S$ to find 
\begin{equation}
    \mu = \frac{qvR}{2}
    % \label{eq:}
\end{equation}
% which we substitute into our equation for the gyromagnetic ratio 
% \begin{equation}
%     \gamma = \frac{\frac{qvR}{2}}{\vec{G}}. 
%     \label{eq:789}
% \end{equation}
and further, we equate the angular momentum vector, using the model of a planar loop to 
\begin{equation}
   G= m_q v R 
    % \label{eq:}
\end{equation}
leaving 
\begin{equation}
    \gamma = \frac{q}{2m_q } . 
    % \label{eq:}
\end{equation}

We finally consider that we may represent the, currently arbitrary, charge and mass as a sum of electron charges and masses. 
\begin{equation}
    \gamma = \frac{q}{2m_q } = \frac{\cancel{N}e}{2\cancel{N} m_e} \implies \gamma = \frac{e}{2 m_e}
    \label{eq:gyromagnetic_ratio_electrons}
\end{equation}

We therefore find that the gyromagnetic ratio of the electron depends only on fundamental constants \cite{bromley2000quantum}.

%pg 329 
% https://www.google.co.uk/books/edition/_/7qCMUfwoQcAC?hl=en&gbpv=1&bsq=walter%20greiner%20theoretical%20physics



\input{Sections/SpinInitialisation.tex}
\section{Lattice Symmetry}
Tetragonal lattice has the $$

\section{\index{Stark!Stark Effect}{Stark Effect}}\label{stark}
For our Spin Hamiltonian given by 
\begin{equation}
    \begin{align}
        H &= H_{\ce{ZFS}} + H_{\ce{Zeeman}} + H_{\ce{Hyperfine}} + H_{\ce{Zeeman(n)}}+ H_{\ce{Quadrupole}}\\ 
        H &= \hat{\vec{S}}\cdot{\vec{D}}\cdot{\vec{S}} + g\mu_b \hat{\vec{S}}\cdot \vec{B}  + \hat{\vec{S}}\cdot A \cdot \hat{\vec{I}}  -  \mu_n g_n \hat{\vec{I}}\cdot\vec{B} + \hat{\vec{I}} \cdot Q \cdot \hat{\vec{I}}.
    \end{align}
    % \label{eq:}
\end{equation}
In the most general sense, an applied electrical field could change any of the parameters. We will not consider the effect of an applied electrical field on the nuclear Zeeman term as the nuclear is \index{paramagnetically shielded}{paramagnetically shielded} \cite{mims}. 
Thus, in a general sense we may add the contributions of an applied electrical field $\vec{E}$ as $H + H_{\ce{Stark}}$ where 
\begin{equation}
    H_{\ce{Stark}} = \vec{E}\cdot\left(\hat{\vec{S}}\cdot R \cdot \hat{\vec{S}} + T \mu_B \hat{\vec{S}}\cdot \vec{B} + \hat{\vec{S}}\cdot F \cdot \hat{\vec{I}} + \hat{\vec{I}} \cdot q \cdot \hat{\vec{I}} \right). 
    \label{eq:}
\end{equation}%

Here $R, T, F, q$ are matrices for each component of the electric field given by 
\begin{equation}
    R_{ijk} = \frac{\partial D_{jk}}{\partial E_i}, \quad 
    T_{ijk} = \frac{\partial g_{jk}}{\partial E_i}, \quad 
    F_{ijk} = \frac{\partial A_{jk}}{\partial E_i}, \quad 
    q_{ijk} = \frac{\partial Q_{jk}}{\partial E_i}.
    \label{eq:}
\end{equation}

We may immediately simplify $T$ as for this work we assume isotropic and constant $g$, for which $T$ is zero.  

Further, as discussed in section \ref{nuclear} we will not include the contributions of the nuclear Hamiltonians. 

This allows us to then consider only the energy change due to the shift in the $D$ parameter, that is $R$, which is a square matrix for each component of the applied $\vec{E}$. 
Exactly as we did for ZFS in section \ref{zfs}, we may reduce each of these symmetric matrices to a traceless form. 
% Here we define $D$ and $E$ in exactly the same way as \ref{}\td{link D E def}. 

% Additionally, each matrix $R$ must be symmetric, that is $R^T = R$ since the matrix $D$ in the ZFS Hamiltonian is always symmetric. 

Consider the expansion of $\hat{\vec{S}}\cdot R \cdot \hat{\vec{S}}$ which we calculate explicitly 
\begin{equation}
    \begin{align}
        \hat{\vec{S}}\cdot R \cdot \hat{\vec{S}} &= 
    \begin{pmatrix}
        \hat{S}_x & \hat{S}_y & \hat{S}_z
    \end{pmatrix}
    \cdot 
    \begin{pmatrix}
        R_{xx} & R_{xy} & R_{xz} \\
        R_{xy} & R_{yy} & R_{yz} \\
        R_{xz} & R_{yz} & R_{zz} \\
    \end{pmatrix}
    \cdot 
    \begin{pmatrix}
        \hat{S}_x \\ \hat{S}_y \\ \hat{S}_z
    \end{pmatrix}\\ 
         &= R_{xx} \hat{S}_x^2  + R_{yy} \hat{S}_y^2 + R_{zz} \hat{S}_z^2 \\
         &+   R_{xy}(\hat{S}_{x}\hat{S}_y + \hat{S}_y \hat{S}_x) 
        +   R_{xz}(\hat{S}_{x}\hat{S}_z + \hat{S}_z \hat{S}_x) 
        +   R_{yz}(\hat{S}_{y}\hat{S}_z + \hat{S}_z \hat{S}_y).
    \end{align}
    \label{eq:}
\end{equation}

We set the constant trace equal to zero and rewrite 
\begin{equation}
 R_{xx} \hat{S}_x^2  + R_{yy} \hat{S}_y^2 + R_{zz} \hat{S}_z^2   = R_D \left(\hat{S}_z^2 - \frac{1}{3}S(S+1)\right) + R_E\left(\hat{S}_x^2 - \hat{S}_y^2\right).
    \label{eq:}
\end{equation}
Where \index{Stark!$R_D$}{$R_D$} and \index{Stark!$R_E$}{$R_E$} are defined in terms of $R$ the same way $D$ and $E$ are in terms of $D$, see \eqref{fs_D}, \eqref{fs_E}. 

Then, we may write $H_{\ce{Stark}}$ in this basis as 
\begin{equation}
    \begin{align}
        H_{\ce{Stark}} &=\vec{E} \cdot \Biggl(  R_D \left(\hat{S}_z^2 - \frac{1}{3}S(S+1)\right) + R_E\left(\hat{S}_x^2 - \hat{S}_y^2\right) \\ &+ 
   R_{xy}(\hat{S}_{x}\hat{S}_y + \hat{S}_y \hat{S}_x) 
        +   R_{xz}(\hat{S}_{x}\hat{S}_z + \hat{S}_z \hat{S}_x) 
        +   R_{yz}(\hat{S}_{y}\hat{S}_z + \hat{S}_z \hat{S}_y) \Biggr).
    \end{align}
    \label{eq:}
\end{equation}

The final step is to reduce the number of coefficients by exploiting the symmetry of the system. We will study systems with the \index{point group symmetry}{point group symmetry} of $C_{3v}$ \cite{Davidsson2018} which reduces the Hamiltonian again to \cite{mims}
\begin{equation}
    \begin{align}
        H_{\ce{Stark}} &= R_{113}\left(E_x (\hat{S}_x\hat{S}_y + \hat{S}_z\hat{S}_x) + E_y(\hat{S}_y\hat{S}_z + \hat{S}_z\hat{S}_y) \right) \\ 
                       &- R_{2E} \left(E_x(\hat{S}_x\hat{S}_y + \hat{S}_y\hat{S}_x) + E_y(\hat{S}_x^2 - \hat{S}_y^2   )\right) \\ 
                       &+ R_{3D}E_z\left(\hat{S}_z^2 - \frac{1}{3} S(S+1)\right)
    \end{align}
    \label{eq:}
\end{equation}


The coefficient $R_{113}$ represents a mixing of the $m_S = 0$ and $m_S = \pm 1$ states which have an energy splitting of $\mathcal{O}(10^9)$ Hz. The Stark energies are $\sim \mathcal{O} (10^3)$ Hz and of at least second order, thus may be ignored \cite{VanOort1990}.  

Thus finally, we write our \index{Stark!Stark Hamiltonian}{Stark Hamiltonian} as 

\begin{equation}
    \begin{align}
        H_{\ce{Stark}} &=
                        d_\parallel E_z\left(\hat{S}_z^2 - \frac{1}{3} S(S+1)\right)
        - d_\perp  E_y(\hat{S}_x^2 - \hat{S}_y^2   ) + d_\perp E_x(\hat{S}_x\hat{S}_y + \hat{S}_y\hat{S}_x)  
    \end{align}
    \label{eq:stark_hamiltonian}
\end{equation}

where we have labelled the axial contribution as \index{Stark!$d_\parallel$}{$d_\parallel$} and the off-axis contribution as \index{Stark!$d_\perp$}{$d_\perp$} to match the convention of existing literature. 

By direct comparison to \eqref{eq:ZFS_hamiltonian} it is easy to see the first two terms of $\eqref{eq:stark_hamiltonian}$ will contribute to the effective ZFS.  

In general longitudinal fields along the defect’s symmetry axis result in equal shifts of all levels, whereas transverse fields split the orbitals into two branches whose energy difference grows with increasing field \cite{Acosta2012, Bassett2011}. This allows the parameters $d_\perp$ and $d_\parallel$ to be measured experimentally. 

\begin{figure}[H]
    \begin{center}
        \includegraphics[width=0.95\textwidth]{figures/EFieldParallel.png}
        % \includegraphics[width=0.95\textwidth]{figures/EFieldPerp.png}
        % \missingfigure{ODMR/Energy plot showing effect of parallel/perp E field}
    \end{center}
    \caption{Eigenvalue plot showing... \td{better caption}}\label{fig:}
\end{figure}
\begin{figure}[H]
    \begin{center}
        % \includegraphics[width=0.95\textwidth]{figures/EFieldParallel.png}
        \includegraphics[width=0.95\textwidth]{figures/EFieldPerp.png}
        % \missingfigure{ODMR/Energy plot showing effect of parallel/perp E field}
    \end{center}
    \caption{Eigenvalue plot showing... \td{better caption}}\label{fig:}
\end{figure}



% The spacings
% between ground-state sublevels remain relatively unaf-
% fected by electric fields [34, 35]

% [34] E. van Oort and M. Glasbeek, Chemical Physics Letters
% 168, 529 (1990).
% [35] F. Dolde, H. Fedder, M. Doherty, T. Nöbauer,
% F. Rempp, G. Balasubramanian, T. Wolf, F. Reinhard,
% L. Hollenberg, F. Jelezko, and Others, Nature Physics
% 7, 459 (2011).

\input{Sections/StokesShift.tex}
\section{Linear Combination of Atomic Orbitals}

% The quantum mechanical system of the electron spin is described by a Hamiltonian; 

\section{Quantum Sensing}
Quantum sensing involves using a qubit system acting as a quantum sensor that interacts with an external variable of
interest, such as a magnetic field, electric field, strain or acoustic wave, or temperature \cite{Castelletto_2024}. 

Quantum sensors have a higher sensitivity within a nanoscale or microscale sampling volume compared to a fully classical counterpart which would require higher field densities or higher volume interrogation to be effective. 

% Quantum sensing, for example, uses the spin state to acquire a phase shift from interactions with the environment10, and an optical interface (that is, spin-to-photon conversion) allows optical readout of a spin qubit, potentially enhanced by spin-to-charge conversion
\cite{Wolfowicz2021}

% Quantum sensors detect weak physical signals in nanoscale by quantum coherence, quantum properties or quantum entanglement. 
\cite{Kin2021}


% Unlike heritage designs, the magnetometer does not require inductive sensing elements, high frequency radio, and/or optical circuitry and can be made significantly more compact and lightweight,
%....
%Additionally, the robustness of the SiC semiconductor allows for operation in extreme conditions
\cite{Cochrane2016}


% We achieve two-spin interference with a phase sensitivity of 1.79 ± 0.06 dB beyond the SQL and three-spin interference with a phase sensitivity of 2.77 ± 0.10 dB. Besides, a magnetic sensitivity of 0.87 ± 0.09 dB beyond the SQL is achieved by two-spin interference for detecting a real magnetic field. Particularly, the deterministic and joint initialization of NV negative state, NV electron spin, and two nuclear spins is realized at room temperature. The techniques used here are of fundamental importance for quantum sensing and computing, and naturally applicable to other solid-state spin systems.
\cite{Xie2021}

% \paragraph{Qubits}
% A quantum bit or qubit is a simple quantum mechanical system with two eigenstates. 

% DiVincenzo Criteria
\cite{DiVincenzo1995}


\subsection{DiVincenzo Criteria}
\cite{RevModPhys.89.035002}
% To construct a working quantum sensor with any candidate material, DiVincenzo[39] and Degen[6] outlined a set of three necessary conditions that must be followed: i) The quantum system must have discrete resolvable energy levels (or an ensemble of two-level systems with a lower energy state |0⟩ and an upper energy state |1⟩) that are separated by a finite transition energy; ii) it must be possible to initialize the quantum sensor into a well-known state and to read out its state; iii) the quantum sensor can be coherently manipulated, typically by time-dependent fields.
\cite{Crawford2021}

\subsection{Crystal Defects}
% Because of about 250 SiC polytypes are known, there should exist more than thousand different spin defects in SiC with distinct characteristics14,15. One can select defects with the most suitable properties for a concrete task, which is not possible for one universal sensor.

\cite{Kraus2014}

% Spin defect centers with long quantum coherence times (T2) are key solid-state platforms for a variety of quantum applications. 
\cite{Kanai2022}

\subsubsection{Quantisation}
\subsubsection{Polarisation}
 % Based on magnetic-dipole forbidden spin transitions, this scheme enables spatially confined spin control, the imaging of GHz-frequency electric fields, and the characterization of defect spin multiplicity
\cite{PhysRevLett.112.087601}

\subsubsection{Coherent Manipulation}
\cite{Widmann2014}

% We find that simultaneous optical reionization and qubit manipulation can be carried out at room temperature with photoexcitation at the typical excitation wavelength used for readout of the divacancy qubits in 4H SiC
\cite{PhysRevB.105.165108}

% These spin defects can be optically addressed and coherently controlled even at room temperature, and their fluorescence spectrum and optically detected magnetic resonance spectra are different from those of any previously discovered defects. Moreover, the generation of these defects can be well controlled by optimizing the annealing temperature after implantation. These defects demonstrate high thermal stability with coherently controlled electron spins, facilitating their application in quantum sensing and masers under harsh conditions.
\cite{Yan2020}


% These defects are optically active near telecommunication wavelengths11, and are found in a host material for which there already exist industrial-scale crystal growth12 and advanced microfabrication techniques13. In addition, they possess desirable spin coherence properties that are comparable to those of the diamond nitrogen–vacancy centre. This makes them promising candidates for various photonic, spintronic and quantum information applications that merge quantum degrees of freedom with classical electronic and optical technologies
\cite{Koehl2011}

% Coherent manipulation of NV centers in SiC has been achieved with Rabi and Ramsey oscillations. 
\cite{Mu2020}


% Hence, coherent control of NV center spins is achieved at room temperature, and the coherence time 𝑇2 can be reached to around 17.1  𝜇⁢s. Furthermore, an investigation of fluorescence properties of single NV centers shows that they are room-temperature photostable single-photon sources at telecom range.
\cite{PhysRevLett.124.223601}

\subsubsection{Efficient Readout}
% Overcoming poor readout is an increasingly urgent challenge for devices based on solid-state spin defects, particularly given their rapid adoption in quantum sensing, quantum information, and tests of fundamental physics. 
% Our results pave a clear path to achieve unity readout fidelity of solid-state spin sensors through increased ensemble size, reduced spin-resonance linewidth, or improved cavity quality factor.

\cite{Eisenach2021}

% we demonstrate the first ever implementation of SCC for VV0 in SiC by performing spin-selective ionization followed by all-optical single-shot readout of the charge state. Using this technique, we can determine an initially prepared spin state with over 80% fidelity. 
\cite{Anderson2022-sf}


% Here, we demonstrate a photo-electrical detection technique for electron spins of silicon vacancy ensembles in the 4H polytype of silicon carbide (SiC). Further, we show coherent spin state control, proving that this electrical readout technique enables detection of coherent spin motion. Our readout works at ambient conditions, while other electrical readout approaches are often limited to low temperatures or high magnetic fields.
\cite{Niethammer2019}


% High-fidelity single-shot spin readout in silicon opens the way to the development of a new generation of quantum computing and spintronic devices, built using the most important material in the semiconductor industry.
\cite{Morello2010}

% Even if the signal-to-noise ratio is reasonably low, a well-trained convolutional neural network (CNN) can predict the resonance peaks of ODMR spectra or the period of Rabi oscillations. Because of the fast output of predictions by the CNN, this method can be used to sense the magnetic field in the environment and microwave intensities in real time.
\cite{PhysRevApplied.17.034046}


\subsection{Coherence}
\cite{Christle2014},\cite{Soltamov2019}, \cite{Gilardoni2020} \cite{BulanceaLindvall2021}, \cite{Astner2022}

% Long coherence times are key to the performance of quantum bits (qubits). 
\cite{Seo2016-ed}

\subsubsection{Spin Relaxation}
\subsubsection{Dephasing}
\subsubsection{Hahn Echo}
% The coherence time of NV defects in the presence of noise originating from parasitic spins located at the diamond interface can be improved by orders of magnitude by using high-order spin echoes.
\cite{Wu2016}
\subsubsection{Example: NV Diamond}


\subsection{Sensitivity}
\cite{RevModPhys.92.015004}

% Our approach is suitable for ensemble as well as single spin-3/2 color centers, allowing for angle-resolved magnetometry on the nanoscale at ambient conditions.
\cite{PhysRevApplied.4.014009}


 % We report the realization of nanotesla shot-noise-limited ensemble magnetometry based on optically detected magnetic resonance with the silicon vacancy in 4⁢𝐻 silicon carbide. By coarsely optimizing the anneal parameters and minimizing power broadening, we achieve a sensitivity of 50nT/√Hz and a theoretical shot-noise-limited sensitivity of 3.5nT/√Hz.
\cite{PhysRevApplied.15.064022}

% By inserting an NV-doped diamond membrane between two ferrite cones in a bowtie configuration, we realize a ∼250-fold increase of the magnetic field amplitude within the diamond. We demonstrate a sensitivity of ∼0.9⁢pTs1/2 to magnetic fields in the frequency range between 10 and 1000Hz. 
\cite{PhysRevResearch.2.023394}


% For all these color centers we saw an enhancement of the photostable fluorescence emission of at least a factor of 6 using micro-photoluminescence systems. Using custom confocal microscopy setups, we characterized the emission of VSi measuring an enhancement by up to a factor of 20, and of NCVSi with an enhancement up to a factor of 7. The experimental results are supported by finite element method simulations. Our study provides the pathway for device design and fabrication with an integrated ultra-bright ensemble of VSi and NCVSi for in vivo imaging and sensing in the infrared.
\cite{Castelletto2019}

 % low photon count rate significantly limits their applications. We strongly enhanced the brightness by 7 times and spin-control strength by 14 times of single divacancy defects in 4H-SiC membranes using a surface plasmon generated by gold film coplanar waveguides.
\cite{Zhou2023}

\subsection{ODMR}
% The results suggest that magnetic field sensing sensitivity can be greatly improved for the optimized laser and microwave power range.
\cite{PhysRevB.101.064102}


% We measure increased photoluminescence from divacancy ensembles by up to three orders of magnitude using near-ultraviolet excitation, depending on the substrate, and without degrading the electron spin coherence time.
\cite{Wolfowicz2017}

\subsection{Multimodal Sensors}
% Using these properties, an integrated magnetic field and temperature sensor can be implemented on the same center.
\cite{Anisimov2016}

 % The effect can be detected as an abrupt reduction of the photoluminescence intensity under optical pumping without application of microwave fields. 
\cite{PhysRevB.100.094104}


% The coherent control of divacancy demonstrates that coherence time decreases as pressure increases. Based on these, the pressure-induced magnetic phase transition of Nd2Fe14B sample at high pressures was detected. These experiments pave the way to use divacancy in quantum technologies such as pressure sensing and magnetic detection at high pressures
\cite{Liu2022}


 % In this paper, we present a self-protected infrared high-sensitivity thermometry based on spin defects in silicon carbide. Based on the conclusion that the Ramsey oscillations of the spin sensor are robust against magnetic noise due to a self-protected mechanism from the intrinsic transverse strain of the defect, we experimentally demonstrate the Ramsey-based thermometry. The self-protected infrared silicon-carbide thermometry may provide a promising platform for high sensitivity and high-spatial-resolution temperature sensing in a practical noisy environment, especially in biological systems and microelectronics systems.
\cite{PhysRevApplied.8.044015}


% Here we show that the defect charge state can also be used to sense the environment, in particular high-frequency (megahertz to gigahertz) electric fields
\cite{Wolfowicz2018}

% we identify the mechanism that polarizes the spin under optical drive, obtain the ordering of its dark doublet states, point out a path for electric field or strain sensing, and find the theoretical value of its ground-state zero-field splitting to be 68 MHz, in good agreement with experiment. Moreover, we present two distinct protocols of a spin-photon interface based on this defect. Our results pave the way toward quantum information and quantum metrology applications with silicon carbide.
\cite{PhysRevB.93.081207}


% We discuss the experimental achievements in magnetometry and thermometry based on the spin state mixing at level anticrossings in an external magnetic field and the underlying microscopic mechanisms. We also discuss spin fluctuations in an ensemble of vacancies caused by interaction with environment.
\cite{Tarasenko2017}

% Moreover, as an example of an application, we demonstrate thermal sensing using the Ramsey method at about 450 K. Our experimental results would be useful for the investigation of high-temperature properties of defect spins and silicon carbide–based broad-temperature-range quantum sensing.
\cite{PhysRevApplied.10.044042}

% The experiments pave the way for the application of silicon carbide-based high-sensitivity thermometers in the semiconductor industry, biology, and materials sciences.
\cite{D3NR00430A}


% The experiment implies the feasibility of using implanted NV centers in high-quality diamonds to detect temperatures in biology, chemistry, materials science, and microelectronic systems with high sensitivity and nanoscale resolution.
\cite{PhysRevB.91.155404}

% . We then use it to detect the strength of an external magnetic field. Finally, we use the Ramsey methods to realize a temperature sensing with a sensitivity of 163.2 mK/Hz1/2. The experiments demonstrate that the compact fiber-coupled divacancy quantum sensor can be used for multiple practical quantum sensing.
\cite{Quan:23}

% These results establish SiC color centers as compelling systems for sensing nanoscale electric and strain fields.
\cite{PhysRevLett.112.087601}

\section{Silicon Carbide}
% large-area, high-quality SiC substrates readily available for applications such as high-frequency transmitters and solid-state white lighting
\cite{Eddy2009}


% has emerged as the most mature of the wide-bandgap (2.0 eV ≲ Eg ≲ 7.0 eV) semiconductors
\cite{CASADY19961409}


% Here, we present the coherent manipulation of single divacancy spins in 4H-SiC with a high readout contrast (⁠⁠) and a high photon count rate (150 kilo counts per second) under ambient conditions, which are competitive with the nitrogen-vacancy centres in diamond. 
\cite{10.1093/nsr/nwab122}


% Point defects in wide-bandgap semiconductors can have both ground and excited states within the energy gap and, hence, are luminescent centers or often called color centers. With the ground state being deep in the bandgap, the luminescence is often stable even at room temperature. Many color centers also possess a non-zero electron spin and can be excellent candidates for optical spin quantum bits (qubits).
\cite{Son2021}

% silicon vacancy in an industry-friendly platform, SiC, has the potential for various magnetometry applications under ambient conditions.
\cite{PhysRevApplied.6.034001}

% Silicon carbide is an emerging platform for quantum technologies that provides wafer scale and low-cost industrial fabrication.
\cite{Jiang2023}

% Although NV centers in diamond have shown remarkable spin properties, diamond as a host material is not compatible with conventional electronic circuits (14). By comparison, silicon carbide (SiC) is a technology-friendly material with a large-scale production capacity and mature doping techniques
\cite{Jiang2023}


% SiC is a transparent material from the UV to the infrared, possess nonlinear optical properties from the visible to the mid-infrared and it is a meta-material in the mid-infrared range. SiC fluorescence due to color centers can be associated with single photon emitters and can be used as spin qubits for quantum computation and communication networks and quantum sensing. This unique combination of excellent electronic, photonic and spintronic properties has prompted research to develop novel devices and sensors in the quantum technology domain.
\cite{Castelletto2022}


% current applications in single-photon sources, quantum sensing of strain, magnetic and electric fields, spin-photon interface are also described. Finally, the efforts in the integration of these color centers in photonics devices and their fabrication challenges are described.
\cite{Castelletto2020-ie}


\subsection{Production of $\ce{SiC}$}
% The on-demand engineering of optically active spins in technologically friendly materials is a crucial step toward implementation of both maser amplifiers, requiring high-density spin ensembles, and qubits based on single spins. 
\cite{Fuchs2015}


% we discuss energetic particle irradiation, especially proton beam writing (PBW), in which proton microbeams with MeV range are used, as a method to create VSi in SiC since PBW can create VSi in certain locations with micrometer accuracy and this is very useful to introduce VSi in electronic devices without the degradation of their electrical characteristics.
\cite{Ohshima2018}

% Finally, we show the successful generation and characterization of single nitrogen vacancy (NV) center in SiC employing ion implantation. 
\cite{Mu2020}


 % we present the generation and characterization of shallow silicon vacancies in silicon carbide by using different implanted ions and annealing conditions. 
\cite{Wang2019}


% Here, we demonstrate a scalable approach of manufacturing solid-immersion lenses (SILs) on 4H–SiC.
\cite{Sardi2020}

\subsection{Colour Defects in $\ce{SiC}$}
\subsubsection{Electronic Structure}
\subsubsection{Charge State}
\subsubsection{Spin System}


\subsection{Wider Scientific Context}
% The analysis based on the experimentally obtained parameters shows that this property can be used to implement solid-state masers and extraordinarily sensitive radiofrequency amplifiers.
\cite{Kraus2013}




\chapter{Task}
\section{Brief}

I think, as a start can go through section 3.2.4 in the attached PhD thesis? In particular check in details how to diagonalise the NV centre spin S=1 Hamiltonian to get Eq. 3.31?
You could also do some python simulations to plot how the spin levels (i.e. the eigenvalues of the spin Hamiltonian) change with applied magnetic field.

Once we've learned this, we can apply it to other spin defects in SiC.

\section{Work}
\subsection{Concepts and Nomenclature}
\subsubsection{Spin-Spin Interactions}
\subsubsection{Zeeman Splitting}
\subsubsection{Hyperfine Interaction}
\subsection{System Hamiltonian}\label{system_hamiltonian}
The ground state of the $\ce{NV^-}$ spin system in diamond is a triplet state, thus a $S=1$ system. 

The corresponding Hamiltonian, which it seems can be generalised to an electron spin system of a defect, can be expressed as: 
\begin{equation}
    H_{\ce{NV}} = H_{\ce{D}} + H_{\ce{Zeeman}} + H_{\ce{HF}} 
    \label{eq:nv_hamil}
\end{equation}

Here the labels $\ce{D}$, $\ce{Z}$ and $\ce{HF}$ describe the electron spin-spin interactions, the Zeeman interaction with an external magnetic field and the hyperfine interaction between the nuclparallel spin $I$ and the electron spin $S$ of the NV. 

They have the following forms: 
\begin{eqnarray}
    H_{\ce{D}} &=& D S_z^2 + E(S_x^2 + S_y^2) \label{H_D} \\
    H_{\ce{Z}} &=& g \mu_B \sum_{j}^{x,y,z} B_j \cdot S_j \label{H_Z} \\
    H_{\ce{HF}} &=& \vec{S} \cdot A \cdot \vec{I}. \label{H_HF}
\end{eqnarray}

\subsubsection{Spin-Spin Interaction}


The $E$ and $D$ in equation \ref{H_D} the fine structure constants of the spin
system, describing the spin-spin interaction and $S_j$ the corresponding spin operators
in x,y and z-direction. 

D is non-zero in system with axis of threefold (or other manifold) symmetry. 
The symmetry or spin quantization axis points along the connection of the nitrogen
atom and vacancy forming the defect. In bulk diamond $D$ is around 2.87 GHz at room temperature.

The definiteness, orientation and magnitude of $D$ is thus dependent on the specific spin system being studied. 

E occurs when there is a distortion of the point group symmetry, for example strain or an electrical field. In bulk diamond $E$ is typically negligibly small but especially in NDs, $E$ can be of the order of several MHz. 

Thus, similarly, the value of $E$ is a characteristic of the nature of the distortion and 
the specifics of the spin system being studied. 

\subsubsection{Zeeman Interaction}

$B_j$ in equation \ref{H_Z} is the magnetic field along the $x$, $y$ and $z$ direction, $g$ is the $g$-factor of the vacancy and $\mu_B$ the Bohr-Magneton, a constant. 

It seems often the scaled parameter $g\mu_B$ is considered, for the $\ce{NV^-}$ system this is around $28\;\ce{ GHz T^{-1}}$, but again, will be a characteristic of the system being studied. 

\subsubsection{Hyperfine Interaction}

Equation $\ref{H_HF}$ related the nuclear spin to the electron spin via the hyperfine tensor $A$ which has the form 
\begin{equation}
    A = \begin{pmatrix}
        A_\perp & 0 & 0 \\ 
        0 & A_\perp & 0 \\ 
        0 & 0 & A_\parallel
    \end{pmatrix}.
    \label{eq:hyperfine_tensor}
\end{equation}

$A_\parallel$ and $A_\perp$ are the axial and non-axial hyperfine parameters which encode two different interactions. 

\paragraph{Fermi Contact Interaction.}
This interaction is calculated by 
\begin{equation}
    f_A = \frac{A_\parallel + 2 A_\perp}{3}.
    \label{eq:fermi_contact}
\end{equation}

\paragraph{Anisotropic Interaction.}
This interaction is found by considering both spins as magnetic dipoles is calculated by 
\begin{equation}
    d_A = \frac{A_\parallel - A_\perp}{3}.
    \label{eq:anisotropic}
\end{equation}

For the $\ce{NV^-}$ system in diamond specifically, using the values for $A_\parallel$ and $A_\perp$ we calculate that $f_A$ is an order of magnitude stronger than $d_A$ for both $\ce{N^{14}}$ and $\ce{N^{15}}$. 

\subsubsection{Reduced Hamiltonian}
By combining $H_{\ce{D}}$ and $H_{\ce{Z}}$ 
and neglecting $H_{\ce{HF}}$ \todo{why (specifically) do we get to neglect this? Can we generalise?}
we find 
\begin{equation}
    H_{\ce{NV}} = D S_z^2 + E(S_x^2 + S_y^2) + g \mu_B \sum_{j}^{x,y,z} B_j \cdot S_j 
    \label{eq:reduced_H_NV}
\end{equation}

The spin operators $S_j$ in matrix representation are 
\begin{equation}
    S_x = \frac{1}{\sqrt{2}} \begin{pmatrix}
        0 & 1 & 0 \\ 
        1 & 0 & 1 \\ 
        0 & 1 & 0
    \end{pmatrix}, 
    S_y = \frac{i}{\sqrt{2}} \begin{pmatrix}
        0 & -1 & 0 \\ 
        1 & 0 & -1 \\ 
        0 & 1 & 0
    \end{pmatrix}, 
    S_z = \frac{1}{\sqrt{2}} \begin{pmatrix}
        1 & 0 & 0 \\ 
        0 & 0 & 0 \\ 
        0 & 0 & -1
    \end{pmatrix}. 
    \label{eq:spin_operators}
\end{equation}

Then, aligning the magnetic field (with strength $B_0$) along the $z$-axis (the quantisation axis), the reduced Hamiltonian will have the form 
\begin{equation}
    H_{\ce{NV}} = \begin{pmatrix}
        D + B_0 & 0 & E \\ 
        0 & 0 & 0 \\ 
        E & 0 & D-B_0
    \end{pmatrix},
    \label{eq:reduced_H_NV_matrix}
\end{equation}

with Eigenvalues 

\begin{equation}
    E_x = E_y = D \pm \sqrt{B_0^2  + E^2}, \; E_z = 0.
    \label{eq:reduced_H_NV_eigenvalues}
\end{equation}

The corresponding non-normalised Eigenvectors are then 

\begin{eqnarray}
    \ket{X} = \frac{1}{E} \left(B_0 + \sqrt{B_0^2 + E^2}\right) \ket{+1} + \ket{-1} \\ 
    \ket{Y} = \frac{1}{E} \left(B_0 - \sqrt{B_0^2 + E^2}\right) \ket{+1} + \ket{-1} \\ 
    \ket{Z} = \ket{0},
\end{eqnarray}
with
\begin{equation}
    \ket{1} = \begin{pmatrix}
        1 & 0 & 0 
    \end{pmatrix}, \; 
    \ket{0} = \begin{pmatrix}
        0 & 1 & 0 
    \end{pmatrix}\;, 
    \ket{-1} = \begin{pmatrix}
        0 & 0 & 1 
    \end{pmatrix},
    \label{eq:base_states}
\end{equation}
the Eigenvectors for $H_{\ce{NV}}$ with $E=0$\dots

In the case where $E \ll B_0$ the Eigenvectors are well described by the bases $\ket{0}$ and $\ket{\pm 1}$.

For $E \gg B_0$, when transforming the spin operators $S_j$ into the diagonalised system with Hamiltonian $H_{\ce{NV}}$ they read 
\begin{equation}
    \hat{S}_x^\parallel \propto \begin{pmatrix}
        0 & 1 & 0 \\ 
        1 & 0 & 0 \\ 
        0 & 0 & 0 
    \end{pmatrix} \; , 
    \hat{S}_y^\parallel \propto \begin{pmatrix}
        0 & 0 & 0 \\ 
        0 & 0 & -i \\ 
        0 & i & 0 
    \end{pmatrix} \; , 
    \hat{S}_z \propto \begin{pmatrix}
        0 & 0 & -1 \\ 
        0 & 0 & 0 \\ 
        -1 & 0 & 0 
    \end{pmatrix} , 
    \label{eq:diagonalised_spin_operators}
\end{equation}
and 
\begin{equation}
    \hat{H}_{\ce{NV}} = \begin{pmatrix}
        D + \sqrt{B_0^2 + E^2} & 0 & 0 \\ 
        0 & 0 & 0 \\ 
        0 & 0 & D - \sqrt{B_0^2 - E^2}. 
    \end{pmatrix} 
\end{equation}

Another solution for a $\pi/2$ shifted, modulating magnetic field leads to 
\begin{equation}
    \hat{S}_x^\perp \propto \begin{pmatrix}
        0 & 0 & 0 \\ 
        0 & 0 & 1 \\ 
        0 & 1 & 0 
    \end{pmatrix} \; , 
    \hat{S}_y^\perp \propto \begin{pmatrix}
        0 & -i & 0 \\ 
        i & 0 & 0 \\ 
        0 & 0 & 0 
    \end{pmatrix} \; , 
    \hat{S}_z \propto \begin{pmatrix}
        0 & 0 & 1 \\ 
        0 & 0 & 0 \\ 
        1 & 0 & 0 
    \end{pmatrix} , 
    \label{eq:diagonalised_spin_operators}
\end{equation}
a physical interpretation of which is that a linear modulating B-field aligned
along the $x$-axis where strain is applied only allows transitions between the state
$\ket{X}$ and $\ket{0}$, whereas fields perpendicular to the strain and the NV quantization axis
only allow coupling between $\ket{Y}$ and $\ket{0}$.

For an arbitrary external magnetic field, $H_\ce{NV}$ can be expressed using spherical co-ordinates: 
\begin{equation}
    H_{\ce{NV}} = \begin{pmatrix}
        D + B_0 \cdot \cos \theta & \frac{B_0}{\sqrt{2}} \cdot e^{-i\cdot \varphi} \cdot \sin\theta & E \\ 
        \frac{B_0}{\sqrt{2}} \cdot e^{i \cdot \varphi} \cdot \sin\theta & 0 & \frac{B_0}{\sqrt{2}} e^{-i\cdot \varphi} \cdot \sin\theta \\ 
        E & \frac{B_0}{\sqrt{2}} \cdot e^{i \cdot \varphi} \cdot \sin\theta & D - B_0 \cdot \cos \theta
    \end{pmatrix}
    \label{eq:nv_hamil_spherical_matrix}
\end{equation}


Here, we transformed the magnitude of the arbitrary magnetic field into spherical co-ordinates as 
\begin{eqnarray}
    B_x  &=& B_0 \cos\varphi \sin\theta \\ 
    B_y  &=& B_0 \sin\varphi \sin\theta \\ 
    B_z  &=& B_0 \cos\theta 
\end{eqnarray}
with $\theta$ the azimuthal and $\varphi$ the polar angle. Then using equations \ref{eq:reduced_H_NV} and \ref{eq:spin_operators} we compute \ref{eq:nv_hamil_spherical_matrix}.  

It immediately follows from the characteristic equation that Eigenvalues $\lambda$ satisfy 
\begin{equation}
    0 = \lambda^3 - 2\cdot \lambda^2 \cdot D + \frac{D \cdot B_0^2}{2} + \lambda(D^2 - E^2 - B_0^2) - \frac{1}{2}B_0^2\underbrace{\left(D \cdot \cos(2\theta) - 2 \cdot E \cos(2\varphi) \cdot \sin(\theta)^2\right)}_{\Delta_{\varphi \theta}}
    \label{eq:nv_spherical_characteristic_equation}
\end{equation}
% \cite{balasubramanian2009}




