\chapter{Introduction}

% The Introduction should contain a description of your project and the
% problem you are trying to solve. It should start off at a level that
% should be understandable by anyone with a degree in physics, but it
% can become more technical later
%
% Where appropriate you should include references to work that has
% already been done on your topic and anything else which lets you set
% your work in context.
%
% One of the things you will need to do is to ensure that you have a
% suitable list of references.  To do this you should see \cite{Simon2013-lh}
% or some other suitable reference.  Note the format of the citation used
% here is the style favoured in this School.  Here is another
% reference \cite{Simon2013-lh} for good measure.
%
%
% You will also want to make sure you have no spelling or grammatical
% mistakes. To help identify speling mistukes you can use the commands
% \emph{aspell}, \emph{ispell} or \emph{spell} on most Linux/Unix
% computers. See the appropriate manual pages. Remember that spelling
% mistakes are not the only errors which can occur. Spelling checkers
% will not find errors which are, in fact, valid words such as
% \emph{there} for {\em their}, nor will they find repeated repeated
% words which sometimes occur if your concentration is broken when
% typing. \textbf{There is no substitute for thorough proof reading!}
%
% Your dissertation should be no longer than 15,000 words. In terms of
% pages, 30 pages are ok. 50 pages are fine. But it shouldn't be
% much longer than that.
%
%

\section{Defect Orientation}

Colour centres or defects in general are part of the crystal lattice and thus have an associated orientation and direction within the lattice. This allows the definition of a \textbf{defect axis}. For example, in diamond the NV axis is defined as the vector from the vacancy towards the Nitrogen atom when the vacancy is taken as the origin of your co-ordinate system. 

In a tetragonal crystal, due to symmetry there are four possible orientations of a defect within the lattice: $111$, $1\overline{11}$, $\overline{1}1\overline{1}$ and $\overline{11}1$ directions. 

\section{Miller Indices}
The notation for defect orientation above is known as a Miller Index, and we consider the $111$ direction to be aligned with the defect axis. 

This means that if we know the orientation of our crystal then we can establish the orientations of the defect axis inside. For example, using a crystal for which all surfaces belong to the $\{001\}$ lattice planes, each surface normal is aligned with a Cartesian axis. Thus, by fixing the crystal in place, there remain just \textbf{four} possible angles which a defect axis can have with respect to the crystal surface.

Calculating the scalar product of any of the surface planar directions in the family of $\{001\}$ and the four possible orientations of the defect within the lattice we find $\cos\theta = \pm0.6$. Then, considering the physical solutions (from $0, 2\pi$) gives four possible angles that the (directed) defect axis may make with the surface of the crystal: $53.13^\circ$, $306.87^\circ$, $126.9^\circ$ and $233.13^\circ$ ($0.927$, $5.355$, $2.214$ and $4.069$ radians respectively). 






    


\section{Spintronic Magnetometry}
The Hamiltonian, and thus the energy, of a spin is sensitive to the magnetic field because of the Zeeman interaction. 

How the electron Zeeman energy varies with magnetic field is known to a very large precision. 
Therefore, by measuring the energy difference we may determine the magnetic field. This is the mechanism which enables us to use the spin of an electron as a magnetic-field sensor. 

In practice, for example with diamond a fluorescence microscope to measure the electron spin resonances of an ensemble of NV centres. This allows the determination of both magnitude and direction of an external magnetic field. 


\subsection{Applied Magnetic Field}
To use defects as magnetometers, we must understand the nature of their spin states when an external magnetic field is applied. From this we may determine both the amplitude and direction of the external magnetic field from the electron spin resonance frequencies of the defect. 

\subsection{Spin-1 Defect}
A spin-1 defect has $S=1$ electron spin. Therefore, it has $3$ possible spin states, $m_S=0,−1,+1$. 

With no external applied magnetic field, in general, the $m_S  = \pm1$ states are degenerate, that is they have the same energy. Applying a magnetic field lifts the degeneracy and the $m_S=\pm1$ states will have different energies, $E_u$ and $E_l$ (subscripts refer to "upper" and "lower"). These are equivalent to transition frequencies by $E=hf$, which we denote $f_u$ and $f_l$.

The exact values of these frequencies are functions of both amplitude and direction of the magnetic field, specifically the cosine of the angle between the applied magnetic field and the defect axis $\theta$.

Thus, by experimentally determining the transition frequencies, the magnitude and relative angle of the applied magnetic field may be determined. Details of the derivation are included in \ref{system_hamiltonian} and we find we may determine 
\begin{equation}
\gamma B = \frac{1}{3} \sqrt{f_u^2 + f_l^2 - f_uf_l - D^2}
\end{equation}

\begin{equation}
    \cos^2 \theta = \frac{-(f_u + f_l)^3 + 3 f_u^3 + 3 f_l^3}{27 D (\gamma B)^2} + \frac{2D^2}{27(\gamma B )^2} + \frac{1}{3}
\end{equation}

Here $\gamma = 28 \ce{GHz/T}$ is the gyromagnetic ratio of the electron. $D = 2.87 \ce{GHz}$ is the zero-field splitting of the defect ground state, that is the energy difference between $m_S = 0$ and $m_S = \pm 1$ with no external field applied. 

The simplest possible case is when the defect axis aligns with the applied field for which we get a linear relationship 
\begin{equation}
    f_u = D + \gamma B \qquad f_l = D - \gamma B.
\end{equation}


\missingfigure{Include plot of the electron spin resonances vs applied magnetic field.}




