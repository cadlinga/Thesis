\chapter{Conclusions}\label{ch:conclusions}
%%%%%%%%%%%%%%%%%%%%%%%%%%%%%%%%%%%% 1-2 Pages %%%%%%%%%%%%%%%%%%%%%%%%%%%%
\begin{proposal}{Checklist:}
    \begin{itemize}
        \item Work on interplay between temp and other parameters in V2
        \item Exploration of other defects, specifically looking to identify insensitivities 
        \item The development of a S=3/2 schema for measuring the $\vec{E}$ field 
        \item Exploration of phase-accumulation techniques to boost sensitivity and coherence time
        \item More work to determine "1 fixed" sensing, with less tight contraints (i.e. a known $\vec{B}$ field but not $\perp$)
        \item Wider scientific context of why this is useful
    \end{itemize}
\end{proposal}



% \lipsum[1]
\section{Multimodal Spin Based Sensors}
% \lipsum[2-4]
\section{Wider Scientific Context}
% The analysis based on the experimentally obtained parameters shows that this property can be used to implement solid-state masers and extraordinarily sensitive radiofrequency amplifiers.
\cite{Kraus2013}
% \lipsum[5-8]

\section{Future Work}
% \lipsum[9]
%
% This is the place to put your conclusions about your work. You can
% split it into different sections if appropriate. You may want to include
% a section of future work which could be carried out to continue your
% research.
%
% The conclusion section should be at least one page long, preferably 2
% pages, but not much longer.
%
%
