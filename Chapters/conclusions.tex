\chapter{Conclusions}\label{ch:conclusions}
% \todo[inline, color=edired]{Write conclusion}
%     \begin{itemize}
%         \item Work on interplay between temp and other parameters in V2
%         \item Exploration of other defects, specifically looking to identify insensitivities 
%         \item The development of a S=3/2 schema for measuring the $\vec{E}$ field 
%         \item Exploration of phase-accumulation techniques to boost sensitivity and coherence time
%         \item More work to determine "1 fixed" sensing, with less tight contraints (i.e. a known $\vec{B}$ field but not $\perp$)
%         \item Wider scientific context of why this is useful
%     \end{itemize}
% \section{Multimodal Spin Based Sensors}

Overall, the scope of multimodality which can be achieved depends very heavily on the complex interconnection of the parameter influence. By constraining a freedom of one of the parameters the interplay is significantly simplified and allows more possibility for multimodal measurement. Whilst more restrictive, this is still relevant as many physical systems, for which nanoscale sensors of this kind will be useful, will have predictable characteristics. This is particularly valid as the PL6 and V2 only occupy the c-axis in 4H-SiC meaning comparing the defect axis to the laboratory frame is simplified. 

By means of an example, a transistor on a chip will have a direction in which current may flow and the magnetic field will curl around that flow of current. These defects may be implanted (or the system) designed such that the incident field is parallel to the defect axis. Which, if the interplay between temperature and pressure was developed would allow for optical determination of the current across the transistor (inferred from $|\vec{B}|$), temperature and pressure. 

% It has been shown in this work that by controlling other factors, defects in silicon carbide may be used as quantum sensors to detect both the electric and magnetic fields, pressure, strain and temperature. 

For multimodal sensing without constraint, the temperature independence of the ZFS $D$ parameter in the V2 Silicon vacancy showed the most obvious application. Further research into the interplay between temperature and any other characteristic will expand the scope of what it is possible to simultaneously sense. Additionally, any work to investigate the properties of defects in SiC which may identify another defect with a particular insensitivity would enable great development in the multimodal space. 

% This allowed for sensing of magnetic field and temperature simultaneously - possibly even with the same defect family. With a long enough integration time, so assuming a steady state system, this measurement could potentially be made using only a single defect.

For constrained multimodal sensing, schemas were more readily developed which exploit the reduction in complexity when the off diagonal terms in the Hamiltonian were suppressed. 
Another area which could aid development would be the design of a $S= 3/2$ $\vec{E}$ field schema. This would allow us to exploit the temperature stable ZFS $D$ in the V2 defect to combine detection of the electric field and temperature.

We showed that if the interplay between $T$ and $P$ and the combined effect on ZFS $D$ could be understood then we may simultaneously measure temperature and pressure and even extend a trimodal sensing. 

An approach considered in this work, albeit unsuccessful was to attempt to reduce the ambiguity within some of the systems by comparing the effects on basal and c-axis defects. Similar schemes are used with the Diamond nitrogen vacancy, however that system has four distinct defect orientations and so integration into three dimensional space is overdetermined. Conversely for SiC there are only two distinct orientations and thus integration into three dimensional space is underdetermined. Further, at temperatures above cryogenic, some of the alternative defects produce very low contrast. Thus, in the context of multimodality, even if solved, the technique would likely be inappropriate. 

\section{Wider Scientific Context}
Equipped with the methods developed in this work, specific systems may be identified (similar to the transistor example above) where the constrained multimodal schemes could be implemented into already established electronic systems. Further, due to the resilience of SiC and the scale of the proposed sensors the multimodal techniques presented here may also be applied in the laboratory and aid the development of other fields, particularly in harsh environmental conditions. For example, investigating the threshold voltage instability in SiC and monitoring the parameters using already implanted defects. 

Overall, this work represents a foot in the door of the possibilities for multimodal spin based quantum sensing and has highlighted clear and specific opportunities for impactful future research. 

% The analysis based on the experimentally obtained parameters shows that this property can be used to implement solid-state masers and extraordinarily sensitive radiofrequency amplifiers.
% \cite{Kraus2013}
% \lipsum[5-8]



 
%%%%%%%%%%%%%%%%%%%%%%%%%%%%%%%%%%%% 1-2 Pages %%%%%%%%%%%%%%%%%%%%%%%%%%%%



% \lipsum[1]
% \lipsum[2-4]
% \lipsum[9]
%
% This is the place to put your conclusions about your work. You can
% split it into different sections if appropriate. You may want to include
% a section of future work which could be carried out to continue your
% research.
%
% The conclusion section should be at least one page long, preferably 2
% pages, but not much longer.
%
%
