\chapter{Conclusions}\label{ch:conclusions}
\todo[inline, color=edired]{Write conclusion}
    \begin{itemize}
        \item Work on interplay between temp and other parameters in V2
        \item Exploration of other defects, specifically looking to identify insensitivities 
        \item The development of a S=3/2 schema for measuring the $\vec{E}$ field 
        \item Exploration of phase-accumulation techniques to boost sensitivity and coherence time
        \item More work to determine "1 fixed" sensing, with less tight contraints (i.e. a known $\vec{B}$ field but not $\perp$)
        \item Wider scientific context of why this is useful
    \end{itemize}
\section{Multimodal Spin Based Sensors}

Overall, the scope of multimodality which can be achieved depends very heavily on the complex interconnection of the parameter influence. Often, by constraining a freedom of one of the parameters the interplay is significantly simplified and allows for multimodal measurement. This is of course still relevant as many physical systems for which nanoscale sensors of this kind will be useful will have predictable characteristics. 

It has been shown in this work that by controlling other factors, defects in silicon carbide may be used as quantum sensors to detect both the electric and magnetic fields, pressure, strain and temperature. 

For multimodal sensing without constraint, the temperature independence of the ZFS $D$ parameter in the V2 Silicon vacancy showed the most obvious application. As briefly mentioned, further research into the interplay between temperature and any other characteristic will expand the scope of what it is possible to simultaneously sense. 

This allowed for sensing of magnetic field and temperature simultaneously - possibly even with the same defect family. With a long enough integration time, so assuming a steady state system, this measurement could potentially be made using only a single defect.

For constrained multimodal sensing, three more schemas were developed which exploit the reduction in complexity when the off diagonal terms in the Hamiltonian are reduced by precise field alignment. 

Without a developed schema for $\vec{E}$ field measurement with a $S=3/2$ system, the temperature stable ZFS $D$ was not able to be exploited and so we were not able to determine an unconstrained method for multimodal sensing involving the $\vec{E}$ field. 

An approach considered in this work, albeit unsuccessful was to attempt to reduce the ambiguity within some of the systems by comparing the effects on basal and c-axis defects. Similar schemes are used with the Diamond nitrogen vacancy, however that system has four distinct defect orientations and so integration into three dimensional space is overdetermined. Conversely for SiC there are only two distinct orientations and thus integration into three dimensional space is underdetermined. Further, at temperatures above cryogenic, some of the alternative defects produce very low contrast. Thus, in the context of multimodality, even if solved, the technique would likely be inappropriate. 
\section{Wider Scientific Context}
% The analysis based on the experimentally obtained parameters shows that this property can be used to implement solid-state masers and extraordinarily sensitive radiofrequency amplifiers.
\cite{Kraus2013}
% \lipsum[5-8]

\section{Future Work}



%%%%%%%%%%%%%%%%%%%%%%%%%%%%%%%%%%%% 1-2 Pages %%%%%%%%%%%%%%%%%%%%%%%%%%%%



% \lipsum[1]
% \lipsum[2-4]
% \lipsum[9]
%
% This is the place to put your conclusions about your work. You can
% split it into different sections if appropriate. You may want to include
% a section of future work which could be carried out to continue your
% research.
%
% The conclusion section should be at least one page long, preferably 2
% pages, but not much longer.
%
%
