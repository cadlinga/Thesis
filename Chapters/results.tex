\chapter{Results and Analysis}
%
% This section should detail the obtained results in a clear,
% easy-to-follow manner. It is important to make clear what are original
% results and what are repeats of previous calculations or computations.
% Remember that long tables of numbers are just as boring to read as
% they are to type-in!
%
% Use graphs to present your results wherever practicable.
%
% Results or computations should be presented with uncertainties
% (errors), both statistical and systematic where applicable.
%
% Be selective in what you include: half a dozen \emph{e.g.}~tables that
% contain wrong data you collected while you forgot to switch on the
% computer are not relevant and may mask the correct results.
%
%
% \section{Some results}
% Here are some results.
%
% \subsection{More results}
% When showing results you are likely to use tables and graphs. You can
% create tables easily in \LaTeX.
%
% \begin{table}[h]
% \begin{center}
% \begin{tabular}{||l|c|l||}
% \hline
% \textbf{File names} & \textbf{Satellite} & \textbf{Resolution}\\
% \hline
%   worldr            &  Meteosat          &   5km\\
%   worldg            &  Meteosat          &   5km\\
%   worldb            &  Meteosat          &   5km\\
% \hline
% \end{tabular}
% \end{center}
% \caption{This is a simple table. More complicated tables can have
%   headings which pass over more than one column. The caption should
%   explain exactly in some detail what is displayed in the table.}
% \label{simple_table}
% \end{table}
%
% If you want to produce fancier tables than shown in Table \ref{simple_table}
% refer to the \LaTeX\ manual or ask Madame La Google.
%
% One of the simplest ways to produce simple graphs is to use gnuplot
% which produces \LaTeX\  output. Graph~(\ref{fig:gnu}) was produced using
% gnuplot with output designated as \LaTeX\  so that a \LaTeX\  output file is
% produced which you can include directly or keep separate and refer to
% using the \emph{include} command.
%
% Another approach is to draw simple figures using \emph{xfig} which allows
% you to export diagrams in \LaTeX\  picture format so that the diagram can
% be included directly.
%
% Perhaps the most robust way to include graphs is to convert them to
% PostScript or PDF and include them in the same was as was done in
% Figure~\ref{fig:eucrest} for the University Crest. You can usually do
% this with most packages, including Microsoft ones; one trick for
% producing PostScript is to print to a dummy PostScript printer.
%
% % in practice you would probably keep this in a separate file and use
% % the \include{filename} command to insert it here.
%
% \begin{figure}
% % GNUPLOT: LaTeX picture
% \setlength{\unitlength}{0.240900pt}
% \ifx\plotpoint\undefined\newsavebox{\plotpoint}\fi
% \sbox{\plotpoint}{\rule[-0.200pt]{0.400pt}{0.400pt}}%
% \begin{picture}(1500,1200)(0,0)
% \font\gnuplot=cmr10 at 10pt
% \gnuplot
% \sbox{\plotpoint}{\rule[-0.200pt]{0.400pt}{0.400pt}}%
% \put(220.0,113.0){\rule[-0.200pt]{292.934pt}{0.400pt}}
% \put(220.0,113.0){\rule[-0.200pt]{0.400pt}{245.477pt}}
% \put(220.0,113.0){\rule[-0.200pt]{4.818pt}{0.400pt}}
% \put(198,113){\makebox(0,0)[r]{$0$}}
% \put(1416.0,113.0){\rule[-0.200pt]{4.818pt}{0.400pt}}
% \put(220.0,317.0){\rule[-0.200pt]{4.818pt}{0.400pt}}
% \put(198,317){\makebox(0,0)[r]{$0.2$}}
% \put(1416.0,317.0){\rule[-0.200pt]{4.818pt}{0.400pt}}
% \put(220.0,521.0){\rule[-0.200pt]{4.818pt}{0.400pt}}
% \put(198,521){\makebox(0,0)[r]{$0.4$}}
% \put(1416.0,521.0){\rule[-0.200pt]{4.818pt}{0.400pt}}
% \put(220.0,724.0){\rule[-0.200pt]{4.818pt}{0.400pt}}
% \put(198,724){\makebox(0,0)[r]{$0.6$}}
% \put(1416.0,724.0){\rule[-0.200pt]{4.818pt}{0.400pt}}
% \put(220.0,928.0){\rule[-0.200pt]{4.818pt}{0.400pt}}
% \put(198,928){\makebox(0,0)[r]{$0.8$}}
% \put(1416.0,928.0){\rule[-0.200pt]{4.818pt}{0.400pt}}
% \put(220.0,1132.0){\rule[-0.200pt]{4.818pt}{0.400pt}}
% \put(198,1132){\makebox(0,0)[r]{$1$}}
% \put(1416.0,1132.0){\rule[-0.200pt]{4.818pt}{0.400pt}}
% \put(220.0,113.0){\rule[-0.200pt]{0.400pt}{4.818pt}}
% \put(220,68){\makebox(0,0){$0$}}
% \put(220.0,1112.0){\rule[-0.200pt]{0.400pt}{4.818pt}}
% \put(414.0,113.0){\rule[-0.200pt]{0.400pt}{4.818pt}}
% \put(414,68){\makebox(0,0){$1$}}
% \put(414.0,1112.0){\rule[-0.200pt]{0.400pt}{4.818pt}}
% \put(607.0,113.0){\rule[-0.200pt]{0.400pt}{4.818pt}}
% \put(607,68){\makebox(0,0){$2$}}
% \put(607.0,1112.0){\rule[-0.200pt]{0.400pt}{4.818pt}}
% \put(801.0,113.0){\rule[-0.200pt]{0.400pt}{4.818pt}}
% \put(801,68){\makebox(0,0){$3$}}
% \put(801.0,1112.0){\rule[-0.200pt]{0.400pt}{4.818pt}}
% \put(995.0,113.0){\rule[-0.200pt]{0.400pt}{4.818pt}}
% \put(995,68){\makebox(0,0){$4$}}
% \put(995.0,1112.0){\rule[-0.200pt]{0.400pt}{4.818pt}}
% \put(1188.0,113.0){\rule[-0.200pt]{0.400pt}{4.818pt}}
% \put(1188,68){\makebox(0,0){$5$}}
% \put(1188.0,1112.0){\rule[-0.200pt]{0.400pt}{4.818pt}}
% \put(1382.0,113.0){\rule[-0.200pt]{0.400pt}{4.818pt}}
% \put(1382,68){\makebox(0,0){$6$}}
% \put(1382.0,1112.0){\rule[-0.200pt]{0.400pt}{4.818pt}}
% \put(220.0,113.0){\rule[-0.200pt]{292.934pt}{0.400pt}}
% \put(1436.0,113.0){\rule[-0.200pt]{0.400pt}{245.477pt}}
% \put(220.0,1132.0){\rule[-0.200pt]{292.934pt}{0.400pt}}
% \put(45,622){\makebox(0,0){\shortstack{This is\\the\\$y$ axis}}}
% \put(828,23){\makebox(0,0){This is the $x$ axis}}
% \put(828,1177){\makebox(0,0){This is a plot of $y=\sin(x)$}}
% \put(220.0,113.0){\rule[-0.200pt]{0.400pt}{245.477pt}}
% \sbox{\plotpoint}{\rule[-0.500pt]{1.000pt}{1.000pt}}%
% \put(1306,1067){\makebox(0,0)[r]{sin(x)}}
% \multiput(1328,1067)(20.756,0.000){4}{\usebox{\plotpoint}}
% \put(1394,1067){\usebox{\plotpoint}}
% \put(220,113){\usebox{\plotpoint}}
% \multiput(220,113)(3.768,20.411){4}{\usebox{\plotpoint}}
% \multiput(232,178)(4.132,20.340){3}{\usebox{\plotpoint}}
% \multiput(245,242)(3.825,20.400){3}{\usebox{\plotpoint}}
% \multiput(257,306)(3.884,20.389){3}{\usebox{\plotpoint}}
% \multiput(269,369)(3.944,20.377){3}{\usebox{\plotpoint}}
% \multiput(281,431)(4.326,20.300){3}{\usebox{\plotpoint}}
% \multiput(294,492)(4.137,20.339){3}{\usebox{\plotpoint}}
% \multiput(306,551)(4.276,20.310){3}{\usebox{\plotpoint}}
% \multiput(318,608)(4.693,20.218){3}{\usebox{\plotpoint}}
% \multiput(331,664)(4.583,20.243){2}{\usebox{\plotpoint}}
% \multiput(343,717)(4.754,20.204){3}{\usebox{\plotpoint}}
% \multiput(355,768)(5.034,20.136){2}{\usebox{\plotpoint}}
% \multiput(367,816)(5.760,19.940){2}{\usebox{\plotpoint}}
% \multiput(380,861)(5.579,19.992){3}{\usebox{\plotpoint}}
% \put(398.00,923.50){\usebox{\plotpoint}}
% \multiput(404,943)(7.049,19.522){2}{\usebox{\plotpoint}}
% \multiput(417,979)(7.288,19.434){2}{\usebox{\plotpoint}}
% \put(433.18,1021.10){\usebox{\plotpoint}}
% \multiput(441,1040)(8.982,18.712){2}{\usebox{\plotpoint}}
% \put(460.41,1076.97){\usebox{\plotpoint}}
% \put(471.84,1094.28){\usebox{\plotpoint}}
% \put(484.84,1110.41){\usebox{\plotpoint}}
% \put(500.42,1124.01){\usebox{\plotpoint}}
% \multiput(503,1126)(19.159,7.983){0}{\usebox{\plotpoint}}
% \put(519.48,1131.37){\usebox{\plotpoint}}
% \multiput(527,1132)(20.136,-5.034){0}{\usebox{\plotpoint}}
% \put(539.74,1128.60){\usebox{\plotpoint}}
% \put(557.04,1117.38){\usebox{\plotpoint}}
% \put(570.79,1101.95){\usebox{\plotpoint}}
% \put(582.44,1084.80){\usebox{\plotpoint}}
% \put(593.09,1066.99){\usebox{\plotpoint}}
% \multiput(601,1053)(8.430,-18.967){2}{\usebox{\plotpoint}}
% \put(619.18,1010.54){\usebox{\plotpoint}}
% \multiput(625,996)(7.413,-19.387){2}{\usebox{\plotpoint}}
% \multiput(638,962)(6.403,-19.743){2}{\usebox{\plotpoint}}
% \multiput(650,925)(5.830,-19.920){2}{\usebox{\plotpoint}}
% \multiput(662,884)(5.461,-20.024){2}{\usebox{\plotpoint}}
% \multiput(674,840)(5.533,-20.004){2}{\usebox{\plotpoint}}
% \multiput(687,793)(4.937,-20.160){3}{\usebox{\plotpoint}}
% \multiput(699,744)(4.667,-20.224){2}{\usebox{\plotpoint}}
% \multiput(711,692)(4.858,-20.179){3}{\usebox{\plotpoint}}
% \multiput(724,638)(4.276,-20.310){3}{\usebox{\plotpoint}}
% \multiput(736,581)(4.205,-20.325){3}{\usebox{\plotpoint}}
% \multiput(748,523)(4.070,-20.352){3}{\usebox{\plotpoint}}
% \multiput(760,463)(4.326,-20.300){3}{\usebox{\plotpoint}}
% \multiput(773,402)(3.884,-20.389){3}{\usebox{\plotpoint}}
% \multiput(785,339)(3.884,-20.389){3}{\usebox{\plotpoint}}
% \multiput(797,276)(4.070,-20.352){3}{\usebox{\plotpoint}}
% \multiput(810,211)(3.825,-20.400){3}{\usebox{\plotpoint}}
% \multiput(822,147)(3.607,-20.440){2}{\usebox{\plotpoint}}
% \put(828,113){\usebox{\plotpoint}}
% \end{picture}
% \caption{Simple Gnuplot example. The caption should tell the reader
%   what is plotted against what, and explain in some detail the various
%   sets of curves of data points. It shouldn't just say ``plot of
%   results for the purple function in green gauge'' without further explanation.}
% \label{fig:gnu}
% \end{figure}
%
% \section{Discussion of your results}
%
% This section should give a picture of what you have taken out of your
% project and how you can put it into context.
%
% This section should summarise the results obtained, detail conclusions
% reached, suggest future work, and changes that you would make if you
% repeated the project.
%
%
