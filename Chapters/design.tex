\chapter{Design}
%
% This section should be written in standard scientific
% language. Standard techniques in your research field should not be
% written out in detail. In computational projects this section should
% be used to explain the algorithms used and the layout of the
% computational code. A copy of the actual code may be given in the
% appendices if appropriate.
%
% This section should emphasise the philosophy of the approach used and
% detail novel techniques. However please note: this section should not
% be a blow-by-blow account of what you did throughout the project. It
% should not contain large detailed sections about things you tried and
% found to be completely wrong! However, if you find that a technique
% that was expected to work failed, that is a valid result and should be
% included.
%
% Here logical structure is particularly important, and you may find
% that to maintain good structure you may have to present the
% explorations/calculations/computations/whatever in a different order
% from the one in which you carried them out.
%
%
% You might sometimes want to include multiple equations in one place
% \begin{eqnarray}
%   E &=& ma^{2} \\
%   E &=& mb^{2} \\
%   E &=& mc^{2}
% \end{eqnarray}
% You might want to include multiple equations in one place without
% numbering them
% \begin{eqnarray*}
%   E &=& ma^{2} \\
%   E &=& mb^{2} \\
%   E &=& mc^{2}
% \end{eqnarray*}
% You might want to include multiple equations in one place without
% numbering \emph{all} of them
% \begin{eqnarray}
%   E &=& ma^{2} \nonumber \\
%   E &=& mb^{2} \nonumber \\
%   E &=& mc^{2}
% \end{eqnarray}
%
% You might also want to include diagrams.  The example shows the use of
% the special command which allows existing pdf files to be included.
% You would normally keep your figures separate from the text.  These
% pictures might be images or pdf output from some program.
%
% Here, I created a figure which is centred and stretched to 30\% of the
% width of the page \verb+{0.30\hsize}+ and with the height stretched by
% the same amount \verb+{!}+ to preserve the aspect ratio. If you omit
% the extension (ie .eps, .ps or .pdf) on the file name then \LaTeX\ will
% pick up the postscript copy whereas pdflatex will automatically pick
% up the PDF version.
%
%
% \begin{figure}
%
% \begin{center}
%   \resizebox{0.30\hsize}{!}{\includegraphics{crest.pdf}}
% \end{center}
%
% \caption{The coloured version of the University crest. The caption should explain exactly in some detail what is displayed in the table.}
% \label{fig:eucrest}
%
% \end{figure}
%
% You should find the file crest.pdf on this wiki.
%
% % note that labels do not need to include a description of the object
% % they are labelling but it can be helpful, eg \label{fig:figurename}.
%
% You can use a label on a figure to refer to it later. The university
% crest is in Figure~(\ref{fig:eucrest}). Note that you should not use
% phrases like ``the figure above'' or ``the following figure'' since
% \LaTeX\ may move the figure relative to the text if it cannot be fitted
% onto the current page. The figure on the next page is an example.
%
%
%
