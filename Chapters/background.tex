\chapter{Background Theory}
\section{Magnetism}
\subsection{Magnetic Dipole}
Where charge ($\mathbf{E}$-field) has an intuitive elementary source unit of a point charge (or monopole) which may be positively or negatively charged. Conversely the elementary source unit of magnetism ($\mathbf{B}$-field) is the magnetic dipole. 

Classically, the magnetic dipole may be modelled as a closed loop that carries an
electric current. 
Its magnetic dipole moment, $\vec{\mu}$, is defined as the vector which points out of the plane
of the current loop, 

\begin{equation}
    \vec{\mu} = IS \vec{n}
    \label{eq:dipole_moment}
\end{equation}
where $I$ is the current in the loop and $S$ is the surface area enclosed by the loop. 

The magnetic dipole produces a magnetic field $\vec{B}$, which for points a large distance from the dipole may be calculated as \tdr{Type up derivation from David Tong notes}:
$$\vec{B} = \frac{\mu_0}{4\pi} \frac{1}{r^3} \left[\frac{3(\vec{\mu} \cdot \vec{r}) \cdot \vec{r}}{r^2} - \vec{\mu}\right]$$

The symmetry of the field enables us to, without any loss of generality, consider the direction of the dipole the $z$-axis. Then, defining $x,y$ as usual by $r \cos\theta$ and $r \sin\theta$ respectively. We may then consider magnetic field in two separate components, parallel ($B_z$) and perpendicular ($B_x, B_y$): 
$$B_\parallel =\frac{\mu_0}{r^3}(3\cos^2 \theta - 1), \quad B_\perp = \frac{3\mu_0}{r^3}\cos\theta\sin\theta.$$
Then, we may use the Pythagorean principle to determine the overall magnitude $B$ as
$$B = \sqrt{B_\parallel^2 + B_\perp^2}.$$

\subsection{Gyromagnetic Ratio}
\subsubsection{Classical Derivation}
The current in equation \ref{eq:dipole_moment} is proportional to the angular momentum of the charge. That is, the dipole moment is always associated with an angular momentum $\vec{G} = \vec{r} \times \vec{p}$ with $\vec{r}$ the radius and $\vec{p}$ the momentum. 

Dividing the magnetic dipole moment by the angular momentum we find the \textbf{gyromagnetic ratio}. 
\begin{equation}
    \gamma = \frac{\vec{\mu}}{\vec{G}}.
    \label{eq:gyromagnetic_ratio}
\end{equation}

Without loss of generality we may consider the most simple case which is where the magnetic dipole moment is parallel (or anti-parallel) to the angular momentum. Then we may consider the absolute values for the dipole moment and the angular momentum: 
\begin{equation}
    \mu = IS, \quad I = \underbrace{\frac{q}{2\pi R}}_{\rho \text{ (charge density)}}v, \quad S = \pi R^2 
    % \label{eq:}
\end{equation}
We substitute $I$ and $S$ to find 
\begin{equation}
    \mu = \frac{qvR}{2} 
    % \label{eq:}
\end{equation}
% which we substitute into our equation for the gyromagnetic ratio 
% \begin{equation}
%     \gamma = \frac{\frac{qvR}{2}}{\vec{G}}. 
%     \label{eq:789}
% \end{equation}
and further, we equate the angular momentum vector, using the model of a planar loop to 
\begin{equation}
   G= m_q v R 
    % \label{eq:}
\end{equation}
leaving 
\begin{equation}
    \gamma = \frac{q}{2m_q } . 
    % \label{eq:}
\end{equation}

We finally consider that we may represent the, currently unknown, charge and mass as a sum of electron charges and masses. We therefore find that the gyromagnetic ratio of the electron depends only on constants 
\begin{equation}
    \gamma = \frac{q}{2m_q } = \frac{\cancel{N}e}{2\cancel{N} m_e} \implies \gamma = \frac{e}{2 m_e}.
    % \label{eq:}
\end{equation}


%pg 329 
% https://www.google.co.uk/books/edition/_/7qCMUfwoQcAC?hl=en&gbpv=1&bsq=walter%20greiner%20theoretical%20physics
\cite{bromley2000quantum}


\subsubsection{Extending to Quantum Mechanics}
Since the gyromagnetic ratio was calculated considering the motion of dipole in a loop, we may extend this to an electron in an orbit within the atom. The fundamental change required to extend the model to quantum mechanics is the treatment of angular momentum which should now be quantized. 
Thus, we replace our classical approximation of $\vec{G} = \vec{r} \times \vec{p}$ with the equation for the eigenvalues of the quantum mechanical representation of orbital angular momentum:
\begin{equation}
    \hat{G} = \hbar \hat{J} 
    % \label{eq:}
\end{equation}
where $\hat{J}$ is the operator of the orbital angular momentum (quantum number of orbital momentum). 

\subsection{Electron Magnetic Moment}
%The electron magnetic moment, −𝜇/𝜇𝐵=𝑔/2=1.001  159 652  180 59 (13)
\cite{PhysRevLett.130.071801}




% \section{The easy bits}
% This is just to show how to break things into sections.
%
% Many paragraphs in this demonstration document are here to provide some
% padding so that sections last for more than one page to illustrate what
% happens on subsequent pages. Note that the page numbering style is usually
% different on the first page of a new chapter than on subsequent pages.
%
% Here is a padding paragraph.  Rhubarb.  More rhubarb.  Yet more
% rhubarb.  Rhubarb.  More rhubarb.  Yet more rhubarb.  Rhubarb.  More
% rhubarb.  Yet more rhubarb.  Rhubarb.  More rhubarb.  Yet more
% rhubarb.  Rhubarb.  More rhubarb.  Yet more rhubarb.  Rhubarb.  More
% rhubarb.  Yet more rhubarb.  Rhubarb.  More rhubarb.  Yet more
% rhubarb.  Rhubarb.  More rhubarb.  Yet more rhubarb.  Rhubarb.  More
% rhubarb.  Yet more rhubarb.  Rhubarb. More rhubarb.  Yet more rhubarb.
% Rhubarb.  More rhubarb.  Yet more rhubarb.  Rhubarb.  More rhubarb.
% Yet more rhubarb. Rhubarb. More rhubarb.  Yet more rhubarb.  Rhubarb.
% More rhubarb. Yet more rhubarb.  Rhubarb.  More rhubarb.  Yet more
% rhubarb. Rhubarb.  More rhubarb.  Yet more rhubarb.  Rhubarb.  More
% rhubarb.  Yet more rhubarb.  Rhubarb. Yet more rhubarb.
% Too much rhubarb. No more rhubarb!
%
% \section{The more difficult bits}
% Some bits are hard.
%
% You might want to include an equation here:
%
% \begin{equation}
%   \delta N_{\nu} = (\delta N_{\nu})_{ex} + (\delta N_{\nu})_{au} 
%   \label{equation:delsplit}
%   % note that the label is optional, but it allows you to refer to this
%   % equation later.
% \end{equation}
%
%
% Here is another padding paragraph.  Bananas.  More bananas.  Yet more
% bananas.  Bananas.  More bananas.  Yet more bananas.  Bananas.  More
% bananas.  Yet more bananas.  Bananas.  More bananas.  Yet more
% bananas.  Bananas.  More bananas.  Yet more bananas.  Bananas.  More
% bananas.  Yet more bananas.  Bananas.  More bananas.  Yet more
% bananas.  Bananas.  More bananas.  Yet more bananas.  Bananas.  More
% bananas.  Yet more bananas.  Bananas.  More bananas.  Yet more
% bananas.  Bananas.  More bananas.  Yet more bananas.  Bananas.  More
% bananas.  Yet more bananas.  Bananas.  More bananas.  Yet more
% bananas.  Bananas.  More bananas.  Yet more bananas.  Bananas.  More
% bananas.  Yet more bananas.  Bananas.  More bananas.  Yet more
% bananas.  Bananas.  More bananas.  Yet more bananas.  Bananas.  More
% bananas.  Yet more bananas.  Bananas.  More bananas.  Yet more
% bananas.  Bananas.  More bananas.  Yet more bananas.  Far too many
% bananas.
%
% \subsection{Hard bits}
% You might want to include another equation or three here:
%
% \begin{equation}
%   \delta N_{\nu} = (\delta N_{\nu})_{ex} + (\delta N_{\nu})_{au} 
%   \label{equation:delsplit2}
%   % note that the label is optional but allows you to refer to this
%   % equation later.
% \end{equation}
%
% Almost the same equation again.
%
% \begin{equation}
%   \delta P_{\nu} = (\delta P_{\nu})_{ex} + (\delta Q_{\nu})_{au} 
%   \label{equation:delsplit3}
%   % note that the label is optional but allows you to refer to this
%   % equation later.
% \end{equation}
%
% You should use a different label for each equation.
%
% Here is a padding paragraph.  Bananas.  More bananas.  Yet more
% bananas.  Bananas.  More bananas.  Yet more bananas.  Bananas.  More
% bananas.  Yet more bananas.  Bananas.  More bananas.  Yet more
% bananas.  Bananas.  More bananas.  Yet more bananas.  Bananas.  More
% bananas.  Yet more bananas.  Bananas.  More bananas.  Yet more
% bananas.  Bananas.  More bananas.  Yet more bananas.  Bananas.  More
% bananas.  Yet more bananas.  Bananas.  More bananas.  Yet more
% bananas.  Bananas.  More bananas.  Yet more bananas.  Bananas.  More
% bananas.  Yet more bananas.  Bananas.  More bananas.  Yet more
% bananas.  Bananas.  More bananas.  Yet more bananas.  Bananas.  More
% bananas.  Yet more bananas.  Bananas.  More bananas.  Yet more
% bananas.  Bananas.  More bananas.  Yet more bananas.  Bananas.  More
% bananas.  Yet more bananas.  Bananas.  More bananas.  Yet more
% bananas.  Bananas.  More bananas.  Yet more bananas. And another
% equation.
%
% \begin{equation}
%   \delta Q_{\nu} = (\delta L_{\nu})_{ex} + (\delta X_{\nu})_{au} 
%   \label{equation:delsplit4}
%   % note that the label is optional but allows you to refer to this
%   % equation later.
% \end{equation}
%
% Here is a pudding paragraph.  Rhubarb crumble.  More rhubarb crumble.
% Yet more rhubarb crumble.  Rhubarb crumble.  More rhubarb crumble.
% Yet more rhubarb crumble.  Rhubarb crumble.  More rhubarb crumble.
% Yet more rhubarb crumble.  Rhubarb crumble.  More rhubarb crumble.
% Yet more rhubarb crumble.  Rhubarb crumble.  More rhubarb crumble.
% Yet more rhubarb crumble.  Rhubarb crumble.  More rhubarb crumble.
% Yet more rhubarb crumble.  Rhubarb crumble.  More rhubarb crumble.
% Yet more rhubarb crumble.  Apple crumble.  More apple crumble.
% Yet more apple crumble.  Apple crumble.  More apple crumble.
% Yet more apple crumble.  Apple crumble.  More apple crumble.
% Yet more apple crumble.  Apple crumble.  More apple crumble.
% Yet more rhubarb crumble.  Rhubarb crumble.  More rhubarb crumble.
% Yet more rhubarb crumble.  Rhubarb crumble.  More rhubarb crumble.
% Yet more rhubarb crumble.  Rhubarb crumble.  More rhubarb crumble.
% Yet more rhubarb crumble.  Rhubarb crumble.  More rhubarb crumble.
% Way too much rhubarb crumble.
%
% \subsection{Even harder bits}
%
% You might sometimes want to include equations without numbering them.
% \[
%   E=mc^{2}
% \]
% And this might be one of the places where you might want to refer to
% equation (\ref{equation:delsplit}). You will usually need to use the
% \LaTeX\ command twice to make cross-references like this work properly.
% The cross-reference information is stored in the \emph{.aux} file so
% don't delete it.
%
% \subsubsection{Numbering}
% You can keep subdividing but eventually you get to a level where
% numbering stops. This text is in a subsubsection which is not numbered
% by default.
%
%
% \paragraph{More on numbering:}
%
% This text is in a paragraph which is also not numbered by default and
% the ``title'' of the paragraph is not on a separate line.
% If you want to increase the depth to which sections are numbered you
% should see the section on setting the secnumdepth counter in the manual. 
%
%
%
