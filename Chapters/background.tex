\chapter{Background Theory}

\section{Spintronics}
Spintronics, a portmantau of \textbf{spin} and elec\textbf{tronics} is a technology which exploits the characteristics of spin akin to how charge is manipulated in electronics. Fundamentally, the smallest stable magnetic moment available in nature is generated by the spin of a single electron. Careful construction of an appropriate system allows for this magnetic moment to be initialised, manipulated and measured to infer the physical properties of the environment surrounding the system. 

% The quantum mechanical system of the electron spin is described by a Hamiltonian; 

\section{Quantum Sensing}
Quantum sensing involves using a qubit system acting as a quantum sensor that interacts with an external variable of
interest, such as a magnetic field, electric field, strain or acoustic wave, or temperature \cite{Castelletto_2024}. 

Quantum sensors have a higher sensitivity within a nanoscale or microscale sampling volume compared to a fully classical counterpart which would require higher field densities or higher volume interrogation to be effective. 

% Quantum sensing, for example, uses the spin state to acquire a phase shift from interactions with the environment10, and an optical interface (that is, spin-to-photon conversion) allows optical readout of a spin qubit, potentially enhanced by spin-to-charge conversion
\cite{Wolfowicz2021}

% Quantum sensors detect weak physical signals in nanoscale by quantum coherence, quantum properties or quantum entanglement. 
\cite{Kin2021}


% Unlike heritage designs, the magnetometer does not require inductive sensing elements, high frequency radio, and/or optical circuitry and can be made significantly more compact and lightweight,
%....
%Additionally, the robustness of the SiC semiconductor allows for operation in extreme conditions
\cite{Cochrane2016}


% We achieve two-spin interference with a phase sensitivity of 1.79 ± 0.06 dB beyond the SQL and three-spin interference with a phase sensitivity of 2.77 ± 0.10 dB. Besides, a magnetic sensitivity of 0.87 ± 0.09 dB beyond the SQL is achieved by two-spin interference for detecting a real magnetic field. Particularly, the deterministic and joint initialization of NV negative state, NV electron spin, and two nuclear spins is realized at room temperature. The techniques used here are of fundamental importance for quantum sensing and computing, and naturally applicable to other solid-state spin systems.
\cite{Xie2021}

% \paragraph{Qubits}
% A quantum bit or qubit is a simple quantum mechanical system with two eigenstates. 

\subsection{DiVincenzo Criteria}
\cite{RevModPhys.89.035002}
% To construct a working quantum sensor with any candidate material, DiVincenzo[39] and Degen[6] outlined a set of three necessary conditions that must be followed: i) The quantum system must have discrete resolvable energy levels (or an ensemble of two-level systems with a lower energy state |0⟩ and an upper energy state |1⟩) that are separated by a finite transition energy; ii) it must be possible to initialize the quantum sensor into a well-known state and to read out its state; iii) the quantum sensor can be coherently manipulated, typically by time-dependent fields.
\cite{Crawford2021}

\subsection{Crystal Defects}
% Because of about 250 SiC polytypes are known, there should exist more than thousand different spin defects in SiC with distinct characteristics14,15. One can select defects with the most suitable properties for a concrete task, which is not possible for one universal sensor.

\cite{Kraus2014}

% Spin defect centers with long quantum coherence times (T2) are key solid-state platforms for a variety of quantum applications. 
\cite{Kanai2022}

\subsubsection{Quantisation}
\subsubsection{Polarisation}
 % Based on magnetic-dipole forbidden spin transitions, this scheme enables spatially confined spin control, the imaging of GHz-frequency electric fields, and the characterization of defect spin multiplicity
\cite{PhysRevLett.112.087601}

\subsubsection{Coherent Manipulation}
\cite{Widmann2014}

% We find that simultaneous optical reionization and qubit manipulation can be carried out at room temperature with photoexcitation at the typical excitation wavelength used for readout of the divacancy qubits in 4H SiC
\cite{PhysRevB.105.165108}

% These spin defects can be optically addressed and coherently controlled even at room temperature, and their fluorescence spectrum and optically detected magnetic resonance spectra are different from those of any previously discovered defects. Moreover, the generation of these defects can be well controlled by optimizing the annealing temperature after implantation. These defects demonstrate high thermal stability with coherently controlled electron spins, facilitating their application in quantum sensing and masers under harsh conditions.
\cite{Yan2020}


% These defects are optically active near telecommunication wavelengths11, and are found in a host material for which there already exist industrial-scale crystal growth12 and advanced microfabrication techniques13. In addition, they possess desirable spin coherence properties that are comparable to those of the diamond nitrogen–vacancy centre. This makes them promising candidates for various photonic, spintronic and quantum information applications that merge quantum degrees of freedom with classical electronic and optical technologies
\cite{Koehl2011}

% Coherent manipulation of NV centers in SiC has been achieved with Rabi and Ramsey oscillations. 
\cite{Mu2020}


% Hence, coherent control of NV center spins is achieved at room temperature, and the coherence time 𝑇2 can be reached to around 17.1  𝜇⁢s. Furthermore, an investigation of fluorescence properties of single NV centers shows that they are room-temperature photostable single-photon sources at telecom range.
\cite{PhysRevLett.124.223601}

\subsubsection{Efficient Readout}
% Overcoming poor readout is an increasingly urgent challenge for devices based on solid-state spin defects, particularly given their rapid adoption in quantum sensing, quantum information, and tests of fundamental physics. 
% Our results pave a clear path to achieve unity readout fidelity of solid-state spin sensors through increased ensemble size, reduced spin-resonance linewidth, or improved cavity quality factor.

\cite{Eisenach2021}

% we demonstrate the first ever implementation of SCC for VV0 in SiC by performing spin-selective ionization followed by all-optical single-shot readout of the charge state. Using this technique, we can determine an initially prepared spin state with over 80% fidelity. 
\cite{Anderson2022-sf}


% Here, we demonstrate a photo-electrical detection technique for electron spins of silicon vacancy ensembles in the 4H polytype of silicon carbide (SiC). Further, we show coherent spin state control, proving that this electrical readout technique enables detection of coherent spin motion. Our readout works at ambient conditions, while other electrical readout approaches are often limited to low temperatures or high magnetic fields.
\cite{Niethammer2019}


% High-fidelity single-shot spin readout in silicon opens the way to the development of a new generation of quantum computing and spintronic devices, built using the most important material in the semiconductor industry.
\cite{Morello2010}

% Even if the signal-to-noise ratio is reasonably low, a well-trained convolutional neural network (CNN) can predict the resonance peaks of ODMR spectra or the period of Rabi oscillations. Because of the fast output of predictions by the CNN, this method can be used to sense the magnetic field in the environment and microwave intensities in real time.
\cite{PhysRevApplied.17.034046}


\subsection{Coherence}
\cite{Christle2014},\cite{Soltamov2019}, \cite{Gilardoni2020} \cite{BulanceaLindvall2021}, \cite{Astner2022}

% Long coherence times are key to the performance of quantum bits (qubits). 
\cite{Seo2016-ed}

\subsubsection{Spin Relaxation}
\subsubsection{Dephasing}
\subsubsection{Hahn Echo}
% The coherence time of NV defects in the presence of noise originating from parasitic spins located at the diamond interface can be improved by orders of magnitude by using high-order spin echoes.
\cite{Wu2016}
\subsubsection{Example: NV Diamond}


\subsection{Sensitivity}
\cite{RevModPhys.92.015004}

% Our approach is suitable for ensemble as well as single spin-3/2 color centers, allowing for angle-resolved magnetometry on the nanoscale at ambient conditions.
\cite{PhysRevApplied.4.014009}


 % We report the realization of nanotesla shot-noise-limited ensemble magnetometry based on optically detected magnetic resonance with the silicon vacancy in 4⁢𝐻 silicon carbide. By coarsely optimizing the anneal parameters and minimizing power broadening, we achieve a sensitivity of 50nT/√Hz and a theoretical shot-noise-limited sensitivity of 3.5nT/√Hz.
\cite{PhysRevApplied.15.064022}

% By inserting an NV-doped diamond membrane between two ferrite cones in a bowtie configuration, we realize a ∼250-fold increase of the magnetic field amplitude within the diamond. We demonstrate a sensitivity of ∼0.9⁢pTs1/2 to magnetic fields in the frequency range between 10 and 1000Hz. 
\cite{PhysRevResearch.2.023394}


% For all these color centers we saw an enhancement of the photostable fluorescence emission of at least a factor of 6 using micro-photoluminescence systems. Using custom confocal microscopy setups, we characterized the emission of VSi measuring an enhancement by up to a factor of 20, and of NCVSi with an enhancement up to a factor of 7. The experimental results are supported by finite element method simulations. Our study provides the pathway for device design and fabrication with an integrated ultra-bright ensemble of VSi and NCVSi for in vivo imaging and sensing in the infrared.
\cite{Castelletto2019}

 % low photon count rate significantly limits their applications. We strongly enhanced the brightness by 7 times and spin-control strength by 14 times of single divacancy defects in 4H-SiC membranes using a surface plasmon generated by gold film coplanar waveguides.
\cite{Zhou2023}

\subsection{ODMR}
% The results suggest that magnetic field sensing sensitivity can be greatly improved for the optimized laser and microwave power range.
\cite{PhysRevB.101.064102}


% We measure increased photoluminescence from divacancy ensembles by up to three orders of magnitude using near-ultraviolet excitation, depending on the substrate, and without degrading the electron spin coherence time.
\cite{Wolfowicz2017}

\subsection{Multimodal Sensors}
% Using these properties, an integrated magnetic field and temperature sensor can be implemented on the same center.
\cite{Anisimov2016}

 % The effect can be detected as an abrupt reduction of the photoluminescence intensity under optical pumping without application of microwave fields. 
\cite{PhysRevB.100.094104}


% The coherent control of divacancy demonstrates that coherence time decreases as pressure increases. Based on these, the pressure-induced magnetic phase transition of Nd2Fe14B sample at high pressures was detected. These experiments pave the way to use divacancy in quantum technologies such as pressure sensing and magnetic detection at high pressures
\cite{Liu2022}


 % In this paper, we present a self-protected infrared high-sensitivity thermometry based on spin defects in silicon carbide. Based on the conclusion that the Ramsey oscillations of the spin sensor are robust against magnetic noise due to a self-protected mechanism from the intrinsic transverse strain of the defect, we experimentally demonstrate the Ramsey-based thermometry. The self-protected infrared silicon-carbide thermometry may provide a promising platform for high sensitivity and high-spatial-resolution temperature sensing in a practical noisy environment, especially in biological systems and microelectronics systems.
\cite{PhysRevApplied.8.044015}


% Here we show that the defect charge state can also be used to sense the environment, in particular high-frequency (megahertz to gigahertz) electric fields
\cite{Wolfowicz2018}

% we identify the mechanism that polarizes the spin under optical drive, obtain the ordering of its dark doublet states, point out a path for electric field or strain sensing, and find the theoretical value of its ground-state zero-field splitting to be 68 MHz, in good agreement with experiment. Moreover, we present two distinct protocols of a spin-photon interface based on this defect. Our results pave the way toward quantum information and quantum metrology applications with silicon carbide.
\cite{PhysRevB.93.081207}


% We discuss the experimental achievements in magnetometry and thermometry based on the spin state mixing at level anticrossings in an external magnetic field and the underlying microscopic mechanisms. We also discuss spin fluctuations in an ensemble of vacancies caused by interaction with environment.
\cite{Tarasenko2017}

% Moreover, as an example of an application, we demonstrate thermal sensing using the Ramsey method at about 450 K. Our experimental results would be useful for the investigation of high-temperature properties of defect spins and silicon carbide–based broad-temperature-range quantum sensing.
\cite{PhysRevApplied.10.044042}

% The experiments pave the way for the application of silicon carbide-based high-sensitivity thermometers in the semiconductor industry, biology, and materials sciences.
\cite{D3NR00430A}


% The experiment implies the feasibility of using implanted NV centers in high-quality diamonds to detect temperatures in biology, chemistry, materials science, and microelectronic systems with high sensitivity and nanoscale resolution.
\cite{PhysRevB.91.155404}

% . We then use it to detect the strength of an external magnetic field. Finally, we use the Ramsey methods to realize a temperature sensing with a sensitivity of 163.2 mK/Hz1/2. The experiments demonstrate that the compact fiber-coupled divacancy quantum sensor can be used for multiple practical quantum sensing.
\cite{Quan:23}

% These results establish SiC color centers as compelling systems for sensing nanoscale electric and strain fields.
\cite{PhysRevLett.112.087601}

\section{Silicon Carbide}
% large-area, high-quality SiC substrates readily available for applications such as high-frequency transmitters and solid-state white lighting
\cite{Eddy2009}


% has emerged as the most mature of the wide-bandgap (2.0 eV ≲ Eg ≲ 7.0 eV) semiconductors
\cite{CASADY19961409}


% Here, we present the coherent manipulation of single divacancy spins in 4H-SiC with a high readout contrast (⁠⁠) and a high photon count rate (150 kilo counts per second) under ambient conditions, which are competitive with the nitrogen-vacancy centres in diamond. 
\cite{10.1093/nsr/nwab122}


% Point defects in wide-bandgap semiconductors can have both ground and excited states within the energy gap and, hence, are luminescent centers or often called color centers. With the ground state being deep in the bandgap, the luminescence is often stable even at room temperature. Many color centers also possess a non-zero electron spin and can be excellent candidates for optical spin quantum bits (qubits).
\cite{Son2021}

% silicon vacancy in an industry-friendly platform, SiC, has the potential for various magnetometry applications under ambient conditions.
\cite{PhysRevApplied.6.034001}

% Silicon carbide is an emerging platform for quantum technologies that provides wafer scale and low-cost industrial fabrication.
\cite{Jiang2023}

% Although NV centers in diamond have shown remarkable spin properties, diamond as a host material is not compatible with conventional electronic circuits (14). By comparison, silicon carbide (SiC) is a technology-friendly material with a large-scale production capacity and mature doping techniques
\cite{Jiang2023}


% SiC is a transparent material from the UV to the infrared, possess nonlinear optical properties from the visible to the mid-infrared and it is a meta-material in the mid-infrared range. SiC fluorescence due to color centers can be associated with single photon emitters and can be used as spin qubits for quantum computation and communication networks and quantum sensing. This unique combination of excellent electronic, photonic and spintronic properties has prompted research to develop novel devices and sensors in the quantum technology domain.
\cite{Castelletto2022}


% current applications in single-photon sources, quantum sensing of strain, magnetic and electric fields, spin-photon interface are also described. Finally, the efforts in the integration of these color centers in photonics devices and their fabrication challenges are described.
\cite{Castelletto2020-ie}


\subsection{Production of $\ce{SiC}$}
% The on-demand engineering of optically active spins in technologically friendly materials is a crucial step toward implementation of both maser amplifiers, requiring high-density spin ensembles, and qubits based on single spins. 
\cite{Fuchs2015}


% we discuss energetic particle irradiation, especially proton beam writing (PBW), in which proton microbeams with MeV range are used, as a method to create VSi in SiC since PBW can create VSi in certain locations with micrometer accuracy and this is very useful to introduce VSi in electronic devices without the degradation of their electrical characteristics.
\cite{Ohshima2018}

% Finally, we show the successful generation and characterization of single nitrogen vacancy (NV) center in SiC employing ion implantation. 
\cite{Mu2020}


 % we present the generation and characterization of shallow silicon vacancies in silicon carbide by using different implanted ions and annealing conditions. 
\cite{Wang2019}


% Here, we demonstrate a scalable approach of manufacturing solid-immersion lenses (SILs) on 4H–SiC.
\cite{Sardi2020}

\subsection{Colour Defects in $\ce{SiC}$}
\subsubsection{Electronic Structure}
\subsubsection{Charge State}
\subsubsection{Spin System}


\subsection{Wider Scientific Context}
% The analysis based on the experimentally obtained parameters shows that this property can be used to implement solid-state masers and extraordinarily sensitive radiofrequency amplifiers.
\cite{Kraus2013}





% \section{The easy bits}
% This is just to show how to break things into sections.
%
% Many paragraphs in this demonstration document are here to provide some
% padding so that sections last for more than one page to illustrate what
% happens on subsequent pages. Note that the page numbering style is usually
% different on the first page of a new chapter than on subsequent pages.
%
% Here is a padding paragraph.  Rhubarb.  More rhubarb.  Yet more
% rhubarb.  Rhubarb.  More rhubarb.  Yet more rhubarb.  Rhubarb.  More
% rhubarb.  Yet more rhubarb.  Rhubarb.  More rhubarb.  Yet more
% rhubarb.  Rhubarb.  More rhubarb.  Yet more rhubarb.  Rhubarb.  More
% rhubarb.  Yet more rhubarb.  Rhubarb.  More rhubarb.  Yet more
% rhubarb.  Rhubarb.  More rhubarb.  Yet more rhubarb.  Rhubarb.  More
% rhubarb.  Yet more rhubarb.  Rhubarb. More rhubarb.  Yet more rhubarb.
% Rhubarb.  More rhubarb.  Yet more rhubarb.  Rhubarb.  More rhubarb.
% Yet more rhubarb. Rhubarb. More rhubarb.  Yet more rhubarb.  Rhubarb.
% More rhubarb. Yet more rhubarb.  Rhubarb.  More rhubarb.  Yet more
% rhubarb. Rhubarb.  More rhubarb.  Yet more rhubarb.  Rhubarb.  More
% rhubarb.  Yet more rhubarb.  Rhubarb. Yet more rhubarb.
% Too much rhubarb. No more rhubarb!
%
% \section{The more difficult bits}
% Some bits are hard.
%
% You might want to include an equation here:
%
% \begin{equation}
%   \delta N_{\nu} = (\delta N_{\nu})_{ex} + (\delta N_{\nu})_{au} 
%   \label{equation:delsplit}
%   % note that the label is optional, but it allows you to refer to this
%   % equation later.
% \end{equation}
%
%
% Here is another padding paragraph.  Bananas.  More bananas.  Yet more
% bananas.  Bananas.  More bananas.  Yet more bananas.  Bananas.  More
% bananas.  Yet more bananas.  Bananas.  More bananas.  Yet more
% bananas.  Bananas.  More bananas.  Yet more bananas.  Bananas.  More
% bananas.  Yet more bananas.  Bananas.  More bananas.  Yet more
% bananas.  Bananas.  More bananas.  Yet more bananas.  Bananas.  More
% bananas.  Yet more bananas.  Bananas.  More bananas.  Yet more
% bananas.  Bananas.  More bananas.  Yet more bananas.  Bananas.  More
% bananas.  Yet more bananas.  Bananas.  More bananas.  Yet more
% bananas.  Bananas.  More bananas.  Yet more bananas.  Bananas.  More
% bananas.  Yet more bananas.  Bananas.  More bananas.  Yet more
% bananas.  Bananas.  More bananas.  Yet more bananas.  Bananas.  More
% bananas.  Yet more bananas.  Bananas.  More bananas.  Yet more
% bananas.  Bananas.  More bananas.  Yet more bananas.  Far too many
% bananas.
%
% \subsection{Hard bits}
% You might want to include another equation or three here:
%
% \begin{equation}
%   \delta N_{\nu} = (\delta N_{\nu})_{ex} + (\delta N_{\nu})_{au} 
%   \label{equation:delsplit2}
%   % note that the label is optional but allows you to refer to this
%   % equation later.
% \end{equation}
%
% Almost the same equation again.
%
% \begin{equation}
%   \delta P_{\nu} = (\delta P_{\nu})_{ex} + (\delta Q_{\nu})_{au} 
%   \label{equation:delsplit3}
%   % note that the label is optional but allows you to refer to this
%   % equation later.
% \end{equation}
%
% You should use a different label for each equation.
%
% Here is a padding paragraph.  Bananas.  More bananas.  Yet more
% bananas.  Bananas.  More bananas.  Yet more bananas.  Bananas.  More
% bananas.  Yet more bananas.  Bananas.  More bananas.  Yet more
% bananas.  Bananas.  More bananas.  Yet more bananas.  Bananas.  More
% bananas.  Yet more bananas.  Bananas.  More bananas.  Yet more
% bananas.  Bananas.  More bananas.  Yet more bananas.  Bananas.  More
% bananas.  Yet more bananas.  Bananas.  More bananas.  Yet more
% bananas.  Bananas.  More bananas.  Yet more bananas.  Bananas.  More
% bananas.  Yet more bananas.  Bananas.  More bananas.  Yet more
% bananas.  Bananas.  More bananas.  Yet more bananas.  Bananas.  More
% bananas.  Yet more bananas.  Bananas.  More bananas.  Yet more
% bananas.  Bananas.  More bananas.  Yet more bananas.  Bananas.  More
% bananas.  Yet more bananas.  Bananas.  More bananas.  Yet more
% bananas.  Bananas.  More bananas.  Yet more bananas. And another
% equation.
%
% \begin{equation}
%   \delta Q_{\nu} = (\delta L_{\nu})_{ex} + (\delta X_{\nu})_{au} 
%   \label{equation:delsplit4}
%   % note that the label is optional but allows you to refer to this
%   % equation later.
% \end{equation}
%
% Here is a pudding paragraph.  Rhubarb crumble.  More rhubarb crumble.
% Yet more rhubarb crumble.  Rhubarb crumble.  More rhubarb crumble.
% Yet more rhubarb crumble.  Rhubarb crumble.  More rhubarb crumble.
% Yet more rhubarb crumble.  Rhubarb crumble.  More rhubarb crumble.
% Yet more rhubarb crumble.  Rhubarb crumble.  More rhubarb crumble.
% Yet more rhubarb crumble.  Rhubarb crumble.  More rhubarb crumble.
% Yet more rhubarb crumble.  Rhubarb crumble.  More rhubarb crumble.
% Yet more rhubarb crumble.  Apple crumble.  More apple crumble.
% Yet more apple crumble.  Apple crumble.  More apple crumble.
% Yet more apple crumble.  Apple crumble.  More apple crumble.
% Yet more apple crumble.  Apple crumble.  More apple crumble.
% Yet more rhubarb crumble.  Rhubarb crumble.  More rhubarb crumble.
% Yet more rhubarb crumble.  Rhubarb crumble.  More rhubarb crumble.
% Yet more rhubarb crumble.  Rhubarb crumble.  More rhubarb crumble.
% Yet more rhubarb crumble.  Rhubarb crumble.  More rhubarb crumble.
% Way too much rhubarb crumble.
%
% \subsection{Even harder bits}
%
% You might sometimes want to include equations without numbering them.
% \[
%   E=mc^{2}
% \]
% And this might be one of the places where you might want to refer to
% equation (\ref{equation:delsplit}). You will usually need to use the
% \LaTeX\ command twice to make cross-references like this work properly.
% The cross-reference information is stored in the \emph{.aux} file so
% don't delete it.
%
% \subsubsection{Numbering}
% You can keep subdividing but eventually you get to a level where
% numbering stops. This text is in a subsubsection which is not numbered
% by default.
%
%
% \paragraph{More on numbering:}
%
% This text is in a paragraph which is also not numbered by default and
% the ``title'' of the paragraph is not on a separate line.
% If you want to increase the depth to which sections are numbered you
% should see the section on setting the secnumdepth counter in the manual. 
%
%
%
