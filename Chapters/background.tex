\chapter{Background Theory}
\section{Magnetism}
Where charge ($\mathbf{E}$-field) has an intuitive elementary source unit of a point charge (or monopole) which may be positively or negatively charged. Conversely the elementary source unit of magnetism ($\mathbf{B}$-field) is the magnetic dipole. 

Classically, the magnetic dipole may be modelled as a closed loop that carries an
electric current. 
Its magnetic dipole moment, $\vec{\mu}$, is defined as the vector which points out of the plane
of the current loop. 

The magnetic dipole produces a magnetic field $\vec{B}$, which for points a large distance from the dipole may be calculated as \tdr{Type up derivation from David Tong notes}:
$$\vec{B} = \frac{\mu_0}{4\pi} \frac{1}{r^3} \left[\frac{3(\vec{\mu} \cdot \vec{r}) \cdot \vec{r}}{r^2} - \vec{\mu}\right]$$

The symmetry of the field enables us to, without any loss of generality, consider the direction of the dipole the $z$-axis. Then, defining $x,y$ as usual by $r \cos\theta$ and $r \sin\theta$ respectively. We may then consider magnetic field in two separate components, parallel ($B_z$) and perpendicular ($B_x, B_y$): 
$$B_\parallel =\frac{\mu_0}{r^3}(3\cos^2 \theta - 1), \quad B_\perp = \frac{3\mu_0}{r^3}\cos\theta\sin\theta.$$
Then, we may use the Pythagorean principle to determine the overall magnitude $B$ as
$$B = \sqrt{B_\parallel^2 + B_\perp^2}.$$

\section{Gyromagnetic Ratio}


\chapter{Background Theory (to sort)}



% \section{The easy bits}
% This is just to show how to break things into sections.
%
% Many paragraphs in this demonstration document are here to provide some
% padding so that sections last for more than one page to illustrate what
% happens on subsequent pages. Note that the page numbering style is usually
% different on the first page of a new chapter than on subsequent pages.
%
% Here is a padding paragraph.  Rhubarb.  More rhubarb.  Yet more
% rhubarb.  Rhubarb.  More rhubarb.  Yet more rhubarb.  Rhubarb.  More
% rhubarb.  Yet more rhubarb.  Rhubarb.  More rhubarb.  Yet more
% rhubarb.  Rhubarb.  More rhubarb.  Yet more rhubarb.  Rhubarb.  More
% rhubarb.  Yet more rhubarb.  Rhubarb.  More rhubarb.  Yet more
% rhubarb.  Rhubarb.  More rhubarb.  Yet more rhubarb.  Rhubarb.  More
% rhubarb.  Yet more rhubarb.  Rhubarb. More rhubarb.  Yet more rhubarb.
% Rhubarb.  More rhubarb.  Yet more rhubarb.  Rhubarb.  More rhubarb.
% Yet more rhubarb. Rhubarb. More rhubarb.  Yet more rhubarb.  Rhubarb.
% More rhubarb. Yet more rhubarb.  Rhubarb.  More rhubarb.  Yet more
% rhubarb. Rhubarb.  More rhubarb.  Yet more rhubarb.  Rhubarb.  More
% rhubarb.  Yet more rhubarb.  Rhubarb. Yet more rhubarb.
% Too much rhubarb. No more rhubarb!
%
% \section{The more difficult bits}
% Some bits are hard.
%
% You might want to include an equation here:
%
% \begin{equation}
%   \delta N_{\nu} = (\delta N_{\nu})_{ex} + (\delta N_{\nu})_{au} 
%   \label{equation:delsplit}
%   % note that the label is optional, but it allows you to refer to this
%   % equation later.
% \end{equation}
%
%
% Here is another padding paragraph.  Bananas.  More bananas.  Yet more
% bananas.  Bananas.  More bananas.  Yet more bananas.  Bananas.  More
% bananas.  Yet more bananas.  Bananas.  More bananas.  Yet more
% bananas.  Bananas.  More bananas.  Yet more bananas.  Bananas.  More
% bananas.  Yet more bananas.  Bananas.  More bananas.  Yet more
% bananas.  Bananas.  More bananas.  Yet more bananas.  Bananas.  More
% bananas.  Yet more bananas.  Bananas.  More bananas.  Yet more
% bananas.  Bananas.  More bananas.  Yet more bananas.  Bananas.  More
% bananas.  Yet more bananas.  Bananas.  More bananas.  Yet more
% bananas.  Bananas.  More bananas.  Yet more bananas.  Bananas.  More
% bananas.  Yet more bananas.  Bananas.  More bananas.  Yet more
% bananas.  Bananas.  More bananas.  Yet more bananas.  Bananas.  More
% bananas.  Yet more bananas.  Bananas.  More bananas.  Yet more
% bananas.  Bananas.  More bananas.  Yet more bananas.  Far too many
% bananas.
%
% \subsection{Hard bits}
% You might want to include another equation or three here:
%
% \begin{equation}
%   \delta N_{\nu} = (\delta N_{\nu})_{ex} + (\delta N_{\nu})_{au} 
%   \label{equation:delsplit2}
%   % note that the label is optional but allows you to refer to this
%   % equation later.
% \end{equation}
%
% Almost the same equation again.
%
% \begin{equation}
%   \delta P_{\nu} = (\delta P_{\nu})_{ex} + (\delta Q_{\nu})_{au} 
%   \label{equation:delsplit3}
%   % note that the label is optional but allows you to refer to this
%   % equation later.
% \end{equation}
%
% You should use a different label for each equation.
%
% Here is a padding paragraph.  Bananas.  More bananas.  Yet more
% bananas.  Bananas.  More bananas.  Yet more bananas.  Bananas.  More
% bananas.  Yet more bananas.  Bananas.  More bananas.  Yet more
% bananas.  Bananas.  More bananas.  Yet more bananas.  Bananas.  More
% bananas.  Yet more bananas.  Bananas.  More bananas.  Yet more
% bananas.  Bananas.  More bananas.  Yet more bananas.  Bananas.  More
% bananas.  Yet more bananas.  Bananas.  More bananas.  Yet more
% bananas.  Bananas.  More bananas.  Yet more bananas.  Bananas.  More
% bananas.  Yet more bananas.  Bananas.  More bananas.  Yet more
% bananas.  Bananas.  More bananas.  Yet more bananas.  Bananas.  More
% bananas.  Yet more bananas.  Bananas.  More bananas.  Yet more
% bananas.  Bananas.  More bananas.  Yet more bananas.  Bananas.  More
% bananas.  Yet more bananas.  Bananas.  More bananas.  Yet more
% bananas.  Bananas.  More bananas.  Yet more bananas. And another
% equation.
%
% \begin{equation}
%   \delta Q_{\nu} = (\delta L_{\nu})_{ex} + (\delta X_{\nu})_{au} 
%   \label{equation:delsplit4}
%   % note that the label is optional but allows you to refer to this
%   % equation later.
% \end{equation}
%
% Here is a pudding paragraph.  Rhubarb crumble.  More rhubarb crumble.
% Yet more rhubarb crumble.  Rhubarb crumble.  More rhubarb crumble.
% Yet more rhubarb crumble.  Rhubarb crumble.  More rhubarb crumble.
% Yet more rhubarb crumble.  Rhubarb crumble.  More rhubarb crumble.
% Yet more rhubarb crumble.  Rhubarb crumble.  More rhubarb crumble.
% Yet more rhubarb crumble.  Rhubarb crumble.  More rhubarb crumble.
% Yet more rhubarb crumble.  Rhubarb crumble.  More rhubarb crumble.
% Yet more rhubarb crumble.  Apple crumble.  More apple crumble.
% Yet more apple crumble.  Apple crumble.  More apple crumble.
% Yet more apple crumble.  Apple crumble.  More apple crumble.
% Yet more apple crumble.  Apple crumble.  More apple crumble.
% Yet more rhubarb crumble.  Rhubarb crumble.  More rhubarb crumble.
% Yet more rhubarb crumble.  Rhubarb crumble.  More rhubarb crumble.
% Yet more rhubarb crumble.  Rhubarb crumble.  More rhubarb crumble.
% Yet more rhubarb crumble.  Rhubarb crumble.  More rhubarb crumble.
% Way too much rhubarb crumble.
%
% \subsection{Even harder bits}
%
% You might sometimes want to include equations without numbering them.
% \[
%   E=mc^{2}
% \]
% And this might be one of the places where you might want to refer to
% equation (\ref{equation:delsplit}). You will usually need to use the
% \LaTeX\ command twice to make cross-references like this work properly.
% The cross-reference information is stored in the \emph{.aux} file so
% don't delete it.
%
% \subsubsection{Numbering}
% You can keep subdividing but eventually you get to a level where
% numbering stops. This text is in a subsubsection which is not numbered
% by default.
%
%
% \paragraph{More on numbering:}
%
% This text is in a paragraph which is also not numbered by default and
% the ``title'' of the paragraph is not on a separate line.
% If you want to increase the depth to which sections are numbered you
% should see the section on setting the secnumdepth counter in the manual. 
%
%
%
