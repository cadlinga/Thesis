\section{Defect Orientation}

Colour centres or defects in general are part of the crystal lattice and thus have an associated orientation and direction within the lattice. This allows the definition of a \textbf{defect axis}. For example, in diamond the NV axis is defined as the vector from the vacancy towards the Nitrogen atom when the vacancy is taken as the origin of your co-ordinate system. 

In a tetragonal crystal, due to symmetry there are four possible orientations of a defect within the lattice: $111$, $1\overline{11}$, $\overline{1}1\overline{1}$ and $\overline{11}1$ directions. 

\section{Miller Indices}
The notation for defect orientation above is known as a Miller Index, and we consider the $111$ direction to be aligned with the defect axis. 

This means that if we know the orientation of our crystal then we can establish the orientations of the defect axis inside. For example, using a crystal for which all surfaces belong to the $\{001\}$ lattice planes, each surface normal is aligned with a Cartesian axis. Thus, by fixing the crystal in place, there remain just \textbf{four} possible angles which a defect axis can have with respect to the crystal surface.

Calculating the scalar product of any of the surface planar directions in the family of $\{001\}$ and the four possible orientations of the defect within the lattice we find $\cos\theta = \pm0.6$. Then, considering the physical solutions (from $0, 2\pi$) gives four possible angles that the (directed) defect axis may make with the surface of the crystal: $53.13^\circ$, $306.87^\circ$, $126.9^\circ$ and $233.13^\circ$ ($0.927$, $5.355$, $2.214$ and $4.069$ radians respectively). 






    

