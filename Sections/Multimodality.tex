\section{Multimodality}
To develop our multimodal system we will start with a very simple model with the assumption that the applied $\vec{B}$ field is parallel to the defect axis. From there we will iterate our ensemble and work to reduce the number of assumptions.


\subsection{$|\vec{B}|$ and $T$}
\cite{Degen2008}

\subsection{Angle Resolved $|\vec{B}|$ and $T$}
% We show that uniaxial color centers in silicon carbide with hexagonal lattice structure can be used to measure not only the strength but also the polar angle of the external magnetic field with respect to the defect axis with high precision. 
\cite{PhysRevApplied.4.014009}

\subsection{$\vec{B}$ and $T$}


\subsection{$|\vec{B}|$, $|\vec{E}|$ and $T$}
The influence of an $\vec{E}$ field parallel to the defect axis is indistinguishable from the influence of a change of temperature. Similarly, the influence of an $\vec{E}$ field perpendicular to the defect axis is indistinguishable from the influence of a $\vec{B}$ field parallel to the defect axis. The exception is when \td{When $B_0$ is smaller than ZFS E when the effects can be distinguished}...

\begin{figure}[H]
	\begin{center}
		% \includegraphics[width=0.95\textwidth]{figures/}
		\missingfigure{2 plots. Both of a basline energy graph and showing the similarity of T and parallel E, and B and perp E.}
	\end{center}
	\caption{\td{write caption}}\label{fig:}
\end{figure}


Thus, to extend the multi-modality to include the $\vec{E}$ field we must isolate the influence of the $\vec{E}$ field from the other environmental factors.



\subsection{$\vec{B}$, $\vec{E}$ and $T$}


% \begin{summary}{Multimodality Summary}{sum:multimodal}
% 	From the construction of our multimodal systems we have learned
% 	\begin{enumerate}
% 		\item The Silicon vacancy is resistant to temperature fluctuations so can be used to measure $\vec{B}$ \textbf{or} $\vec{E}$ while a divacancy monitors temperature.
% 	\end{enumerate}
% \end{summary}
%
