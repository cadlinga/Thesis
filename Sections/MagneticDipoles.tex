\subsection{Magnetic Dipole}

Classically, the magnetic dipole may be modelled as a closed loop that carries an
electric current. 
Its magnetic dipole moment, $\vec{\mu}$, is defined as the vector which points out of the plane
of the current loop, 

\begin{equation}
    \vec{\mu} = IS \vec{n}
    \label{eq:dipole_moment}
\end{equation}
where $I$ is the current in the loop and $S$ is the surface area enclosed by the loop. 

The magnetic dipole produces a magnetic field $\vec{B}$, which for points a large distance from the dipole may be calculated as \td{Type up derivation from David Tong notes?}:
$$\vec{B} = \frac{\mu_0}{4\pi} \frac{1}{r^3} \left[\frac{3(\vec{\mu} \cdot \vec{r}) \cdot \vec{r}}{r^2} - \vec{\mu}\right]$$

The symmetry of the field enables us to, without any loss of generality, consider the direction of the dipole the $z$-axis. Then, defining $x,y$ as usual by $r \cos\theta$ and $r \sin\theta$ respectively. We may then consider magnetic field in two separate components, parallel ($B_z$) and perpendicular ($B_x, B_y$): 
$$B_\parallel =\frac{\mu_0}{r^3}(3\cos^2 \theta - 1), \quad B_\perp = \frac{3\mu_0}{r^3}\cos\theta\sin\theta.$$
Then, we may use the Pythagorean principle to determine the overall magnitude $B$ as
$$B = \sqrt{B_\parallel^2 + B_\perp^2}.$$


