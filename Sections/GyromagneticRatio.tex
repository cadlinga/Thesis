\subsection{Gyromagnetic Ratio}
\subsubsection{Classical Derivation}
The current in equation \ref{eq:dipole_moment} is proportional to the angular momentum of the charge. That is, the dipole moment is always associated with an angular momentum $\vec{G} = \vec{r} \times \vec{p}$ with $\vec{r}$ the radius and $\vec{p}$ the momentum. 

Dividing the magnetic dipole moment by the angular momentum we find the \textbf{gyromagnetic ratio}. 
\begin{equation}
    \gamma = \frac{\vec{\mu}}{\vec{G}}.
    \label{eq:gyromagnetic_ratio}
\end{equation}

Without loss of generality we may consider the most simple case which is where the magnetic dipole moment is parallel (or anti-parallel) to the angular momentum. Then we may consider the absolute values for the dipole moment and the angular momentum: 
\begin{equation}
    \mu = IS, \quad I = 
    % \underbrace{\frac{q}{2\pi R}}_{\rho \text{ (charge density)}}v,
    \frac{qv}{2\pi R},
    \quad S = \pi R^2 
    % \label{eq:}
\end{equation}
We substitute $I$ and $S$ to find 
\begin{equation}
    \mu = \frac{qvR}{2} 
    % \label{eq:}
\end{equation}
% which we substitute into our equation for the gyromagnetic ratio 
% \begin{equation}
%     \gamma = \frac{\frac{qvR}{2}}{\vec{G}}. 
%     \label{eq:789}
% \end{equation}
and further, we equate the angular momentum vector, using the model of a planar loop to 
\begin{equation}
   G= m_q v R 
    % \label{eq:}
\end{equation}
leaving 
\begin{equation}
    \gamma = \frac{q}{2m_q } . 
    % \label{eq:}
\end{equation}

We finally consider that we may represent the, currently unknown, charge and mass as a sum of electron charges and masses. 
\begin{equation}
    \gamma = \frac{q}{2m_q } = \frac{\cancel{N}e}{2\cancel{N} m_e} \implies \gamma = \frac{e}{2 m_e}
    \label{eq:gyromagnetic_ratio}
\end{equation}

We therefore find that the gyromagnetic ratio of the electron depends only on fundamental constants \cite{bromley2000quantum}.

%pg 329 
% https://www.google.co.uk/books/edition/_/7qCMUfwoQcAC?hl=en&gbpv=1&bsq=walter%20greiner%20theoretical%20physics


