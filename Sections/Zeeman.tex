\section{Zeeman Effect}\label{zeeman}
When no magnetic field is applied to a system, the magnetic dipoles of the orbital electron and spin have no preferred direction. 
The energy levels for all combinations of $L$ and $S$ (all $J$) are equivalent. 

If a magnetic field is applied the magnetic moments interact with that field via the \index{Zeeman!Zeeman interaction}{Zeeman interaction}. 
The \index{Zeeman!Zeeman effect}{Zeeman effect} consists of atomic energy level splitting when an external magnetic field is imposed on a sample \cite{Nabokov2002}. 

The classical expression for the energy of a dipole in a magnetic field
\begin{equation}
    E = -\vec{\mu}\cdot\vec{B}
    \label{eq:}
\end{equation}
may be replaced with the Hamiltonian for a quantum mechanical system 
\begin{equation}
    \hat{H}_{\ce{Zeeman} = - \hat{\vec{\mu}}\cdot \vec{B}. 
    \label{eq:}
\end{equation}

The negative sign indicates that when the magnetic moment is parallel to the magnetic field the lowest energy is achieved. 

Thus distinct quantum systems with different $J$ and thus different projections of angular momentum ($m_J$) have different energies due to their interaction with a magnetic field. 

Considering a simple two-level system ($S=1/2$), the energy difference between the spin being aligned or anti-aligned with the field is called the Zeeman energy. 

The Hamiltonian to describe the energy is, using the total angular momentum form of \eqref{eq:orbital_magnetic_moment_operator_bohr_magneton_g_factor}, 
\begin{equation}
    \hat{H}_{\ce{Zeeman}} = g \mu_B \hat{\vec{S}}\cdot\vec{B}. 
    \label{eq:Zeeman_Hamiltonian}
\end{equation}

Without loss of generality we may direct the magnetic field along the $z$ axis and reduce the scalar product to only the $z$ component. Now, using $S=1/2$ quantised along the $z$ axis, i.e. $m_S = \pm 1/2$ we find the Zeeman energy by solving the Shr\"odinger equation 
\begin{equation}
    \hat{H}_{\ce{Zeeman}} \ket{S, m_S} = E_{\ce{Zeeman}}\ket{S, m_S} 
    % \label{eq:}
\end{equation}
which, to a factor is equivalent to, by \eqref{eq:zthcomponent}, to
\begin{equation}
    \hat{S}_{z} \ket{S, m_S} = m_S\ket{S, m_S}.
    % \label{eq:}
\end{equation}

Thus we find the two eigenvalues to be
$$E_+ =\frac{1}{2}g\mu_BB, \qquad E_-=-\frac{1}{2}g\mu_BB$$
and thus the Zeeman energy is given by $g\mu_B B$. 

The $S=1/2$ system is thus doubly \index{degeneracy!degenerate}{degenerate} and the \index{degeneracy}{degeneracy} is lifted by the application a magnetic field. The Zeeman energy is the difference between the two states and it grows linearly with $B$. 

This may be generalised to a more complex system by considering the total angular momentum $J$ where the energy difference between states is given by 
\begin{equation}
   \Delta E = g_J \mu_B B. 
    \label{eq:zeeman_energy}
\end{equation}





