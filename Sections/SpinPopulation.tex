\section{Spin Polarisation}\label{spin_polarisation}
\cite{2008}\cite{Weil2006}\cite{Goldfarb2018-he}\cite{Richert2017}
Spin polarisation in the context of EPR is the unequal population of possible spin states. 
For example the differing population of triplet sublevels under the influence of photoexcitation. 
Microwave radiation is only absorbed or emitted in a spin polarised system; so spin polarisation is essential for EPR. 

When the spin is polarised, the unequal population is brought back to equilibrium either by the thermodynamic effect of spin-lattice interactions, or by an induced magnetic resonance transition as is exploited by EPR.

\index{Boltzman statistics}{Boltzman statistics}
% BOLTZMAN STATISTICS

\subsection{Optical Polarisation}
% We find that simultaneous optical reionization and qubit manipulation can be carried out at room temperature with photoexcitation at the typical excitation wavelength used for readout of the divacancy qubits in 4H SiC
\cite{PhysRevB.105.165108}

