\section{Multimodal Sensors}
\lipsum[9-12]
% Using these properties, an integrated magnetic field and temperature sensor can be implemented on the same center.
\cite{Anisimov2016}

 % The effect can be detected as an abrupt reduction of the photoluminescence intensity under optical pumping without application of microwave fields. 
\cite{PhysRevB.100.094104}


% The coherent control of divacancy demonstrates that coherence time decreases as pressure increases. Based on these, the pressure-induced magnetic phase transition of Nd2Fe14B sample at high pressures was detected. These experiments pave the way to use divacancy in quantum technologies such as pressure sensing and magnetic detection at high pressures
\cite{Liu2022}


 % In this paper, we present a self-protected infrared high-sensitivity thermometry based on spin defects in silicon carbide. Based on the conclusion that the Ramsey oscillations of the spin sensor are robust against magnetic noise due to a self-protected mechanism from the intrinsic transverse strain of the defect, we experimentally demonstrate the Ramsey-based thermometry. The self-protected infrared silicon-carbide thermometry may provide a promising platform for high sensitivity and high-spatial-resolution temperature sensing in a practical noisy environment, especially in biological systems and microelectronics systems.
\cite{PhysRevApplied.8.044015}


% Here we show that the defect charge state can also be used to sense the environment, in particular high-frequency (megahertz to gigahertz) electric fields
\cite{Wolfowicz2018}

% we identify the mechanism that polarizes the spin under optical drive, obtain the ordering of its dark doublet states, point out a path for electric field or strain sensing, and find the theoretical value of its ground-state zero-field splitting to be 68 MHz, in good agreement with experiment. Moreover, we present two distinct protocols of a spin-photon interface based on this defect. Our results pave the way toward quantum information and quantum metrology applications with silicon carbide.
\cite{PhysRevB.93.081207}


% We discuss the experimental achievements in magnetometry and thermometry based on the spin state mixing at level anticrossings in an external magnetic field and the underlying microscopic mechanisms. We also discuss spin fluctuations in an ensemble of vacancies caused by interaction with environment.
\cite{Tarasenko2017}

% Moreover, as an example of an application, we demonstrate thermal sensing using the Ramsey method at about 450 K. Our experimental results would be useful for the investigation of high-temperature properties of defect spins and silicon carbide–based broad-temperature-range quantum sensing.
\cite{PhysRevApplied.10.044042}

% The experiments pave the way for the application of silicon carbide-based high-sensitivity thermometers in the semiconductor industry, biology, and materials sciences.
\cite{D3NR00430A}


% The experiment implies the feasibility of using implanted NV centers in high-quality diamonds to detect temperatures in biology, chemistry, materials science, and microelectronic systems with high sensitivity and nanoscale resolution.
\cite{PhysRevB.91.155404}

% . We then use it to detect the strength of an external magnetic field. Finally, we use the Ramsey methods to realize a temperature sensing with a sensitivity of 163.2 mK/Hz1/2. The experiments demonstrate that the compact fiber-coupled divacancy quantum sensor can be used for multiple practical quantum sensing.
\cite{Quan:23}

% These results establish SiC color centers as compelling systems for sensing nanoscale electric and strain fields.
\cite{PhysRevLett.112.087601}


