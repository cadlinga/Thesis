\section{Spin}
As well as the orbital magnetic moment generated by the orbital angular momentum of the electron, the electron also possesses an intrinsic magnetic moment. Classically this implies an intrinsic angular momentum hence the magnetic moment of elementary particles is termed \index{spin}. 

For a single electron spin may take the value $\pm 1/2$ since the system has only been observed in two possible states \cite{Gerlach1922} and experiments confirm that the orbital angular momentum and spin angular momentum are of the same nature and thus may be summed. 
The magnetic moment of the spin may thus be expressed as \eqref{eq:orbital_magnetic_moment_operator_bohr_magneton_g_factor} \cite{Povh2002-fj} where $g\approx2.0023$ \cite{electron-g-factor, PhysRevLett.130.071801}. 

In reality the electron is point-like and thus the current loop model is unsuitable. Spin is actually a purely quantum mechanical effect and a consequence of the algebra required to satisfy the Dirac equation of relativistic quantum mechanics. The manifestation of this degree of freedom however has the same dimensionality as $\vec{L}$, allowing us to work with the combination of $\vec{L}$ and $\vec{S}$. 

We thus consider the \index{total angular momentum} of a system $J$ given by 
\begin{equation}
    J = L + S 
    \label{eq:total_angular_momentum}
\end{equation}
which make take the values $L+S, L+ S - 1, \dots, |L-S|$. 



