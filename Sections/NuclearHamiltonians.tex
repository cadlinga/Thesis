\section{Nuclear Hamiltonians}\label{nuclear}
\tdr{Decide how in depth these derivations should be. }
There are three additional contributions to the Hamiltonian to be considered which involve an interaction with the nucleus. 

We include them here for completeness, but will disregard them in the rest of this work since their contributions are either global (thus of no interest in EPR) or very small.\tdr{Definitely need a reference or a figure here} 
\subsection{\index{Zeeman!Nuclear Zeeman}}
Equivalent to electron Zeeman but for the nuclear magnetic moment. 
\begin{equation}
    H_{\ce{Zeeman(n)}} = -g_n \mu_n \vec{B} \cdot \hat{\vec{I}}
    \label{eq:nuclear_zeeman_hamiltonian}
\end{equation}

\subsection{\index{Hyperfine Interaction}}
Equivalent to Fine Structure (ZFS) but between the nuclear and electron moment. 
\begin{equation}
    H_{\ce{Hyperfine}} = \hat{\vec{S}} \cdot A \cdot \hat{\vec{I}}
    \label{eq:hyperfine_hamiltonian_dense}
\end{equation}
\begin{equation}
    H_{\ce{Hyperfine}} = A_\parallel \hat{S}_z\hat{I}_z + A_\perp(\hat{S}_y\hat{I}_y + \hat{S}_z\hat{I}_z)
    \label{eq:hyperfine_hamiltonian_componentwise}
\end{equation}

\subsection{\index{Nuclear Quadrupole}}
Equivalent to the electron dipole-dipole but for the nuclear magnetic moment. 
\begin{equation}
    H_{\ce{Quadrupole}} = \hat{\vec{I}} \cdot Q \cdot \hat{\vec{I}}
    \label{eq:nuclear_quadrupole}
\end{equation}
