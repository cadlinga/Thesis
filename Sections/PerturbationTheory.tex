\section{Perturbation Theory}
By considering a ground, non-degenerate state and a \index{perturbation}{perturbation} in the electron Zeeman interaction and the spin-orbit coupling we can develop insight into so called \index{ZFS!zero field splitting}{zero field splitting}. The perturbation is given by 
\begin{equation}
    \hat{H}' = \hat{H}_{\ce{Zeeman}} + \hat{H}_{\ce{SO}}  
    % \label{eq:}
\end{equation}
for which we find 
\begin{equation}
    E_0 = E^{(0)}_0 + \bra{0}{\hat{H}'}\ket{0}  + \sum_{n}\frac{\bra{0} \hat{H}' \ket{n} \bra{n} \hat{H'}\ket{0}}{E_0^{(0)} - E_n^{(0)}}
    \label{eq:perturbation}
\end{equation}

Now, if we consider arbitrary interactions of forms
\begin{eqnarray}
    &\hat{H}_{\ce{Zeeman}} &= g_L \mu_B \hat{\vec{L}}\cdot\vec{B} + g_S \mu_B \hat{\vec{S}}\cdot\vec{B} \label{eq:zeeman_perturbation}\\ 
    & \hat{H}_{\ce{SO}} &= \lambda\hat{\vec{L}}\cdot \hat{\vec{S}} \label{eq:SO_perturbation}
\end{eqnarray}
we may compute the first and second order corrections. 


\subsubsection{First Order}
Substituting \eqref{eq:zeeman_perturbation} and \eqref{eq:SO_perturbation} into \eqref{eq:perturbation} and integrating only over the orbital values to deduce the Spin Hamiltonian we find
\begin{equation}
   \begin{align}
       \bra{0} \hat{H}' \ket{0} &= \bra{0} g_L \mu_B \hat{\vec{L}}\cdot \vec{B} + g_S \mu_B \hat{\vec{S}}\cdot \vec{B} + \lambda\hat{\vec{L}} \cdot \hat{\vec{S}} \ket{0}\\ 
                                & = \bra{0} g_L\mu_B \hat{\vec{L}} \cdot \vec{B} \ket{0} + \bra{0}g_S \mu_B \hat{\vec{S}}\cdot \vec{B}\ket{0} + \bra{0}\lambda\hat{\vec{L}} \cdot \hat{\vec{S}} \ket{0}\\
                                &= g_L\mu_B \vec{B} \cdot \bra{0}\hat{\vec{L}}\ket{0} + g_S \mu_B  \vec{B} \cdot \hat{\vec{S}}\braket{0|0} + \lambda \hat{\vec{S}} \cdot \braket{0|\hat{\vec{L}}|0}\\ 
                                &= g_L\mu_B \vec{B} \cdot \cancelto{0}{\bra{0}\hat{\vec{L}}\ket{0}} + g_S \mu_B  \vec{B} \cdot \hat{\vec{S}}\cancelto{1}{\braket{0|0}} + \lambda \hat{\vec{S}} \cdot \cancelto{0}{\braket{0|\hat{\vec{L}}|0}}\\ 
                                &= g_s \mu_B \hat{S}\cdot\hat{B}.
   \end{align} 
    \label{eq:first_order}
\end{equation}

Here we used the fact that $\braket{0 | \hat{\vec{L}} | 0} = 0$ since, for example in the alegbraic basis $\hat{L}_z = -i\left(x \frac{\partial}{\partial y} - y \frac{\partial }{\partial x}\right)$ is a Hermitian operator is therefore has eigenvalues which are strictly real numbers, i.e. 
\begin{equation}
    \hat{L}_z \ket{\psi} = m_L \ket{\psi}.
    \label{eq:hermitian}
\end{equation}

By considering \eqref{eq:hermitian} we see that if we apply an imaginary operator to a real valued eigenfunction the corresponding eigenvalue must be imaginary or zero. We know the state is strictly real since it is \index{degeracy!non-degenerate}{non-degenerate}\footnote{A complex wavefunction $\psi$ is at least doubly degenerate; the complex conjugate $\psi^*$ has the same energy.}. In this case, the expectation value of $\hat{L}$ can only be $0$. 

\paragraph{Zeeman Splitting.}
The result of the first order perturbation is thus a more formal confirmation of the result of section \ref{zeeman}, specifically \eqref{eq:zeeman_energy}.  

% Using this we find for some real number $r$ 
% \begin{equation}
%     \hat{L}_z \ket{\psi} \implies \braket{\hat{L}_z | \psi | \hat{L}_z} = r \braket{\psi | \psi}
%     % \label{eq:}
% \end{equation}
% which when applied above gives 
%
\subsubsection{Second Order}
At second order, again substituting \eqref{eq:zeeman_perturbation} and \eqref{eq:SO_perturbation} into \eqref{eq:perturbation} 
and integrating only over the orbital values 
we find 
\begin{equation*}
   \begin{align}
       &\sum_{n}\frac{\bra{0} \hat{H}' \ket{n} \bra{n} \hat{H'}\ket{0}}{E_0^{(0)} - E_n^{(0)}}
   \end{align}
\end{equation*}
\begin{equation}
    \begin{align}
 &=\frac{\braket{0 |g_L \mu_B \hat{\vec{L}}\cdot\vec{B} + g_S \mu_B \hat{\vec{S}}\cdot\vec{B} +\lambda\hat{\vec{L}}\cdot \hat{\vec{S}}  | n } \braket{n |g_L \mu_B \hat{\vec{L}}\cdot\vec{B} + g_S \mu_B \hat{\vec{S}}\cdot\vec{B}+\lambda\hat{\vec{L}}\cdot \hat{\vec{S}}| 0}}{E_0^{(0)} - E_n^{(0)}}\\ 
 &=\frac{\braket{0 |g_L \mu_B \hat{\vec{L}}\cdot\vec{B} + \lambda\hat{\vec{L}}\cdot \hat{\vec{S}}  | n } \braket{n |g_L \mu_B \hat{\vec{L}}\cdot\vec{B} + \lambda\hat{\vec{L}}\cdot \hat{\vec{S}}| 0}}{E_0^{(0)} - E_n^{(0)}}\\ 
 &= (g_L \mu_B \vec{B} + \lambda \hat{\vec{S}})\underbrace{\sum_n \frac{\braket{0 |\hat{\vec{L}} | n}\braket{n |\hat{\vec{L}} | 0}}{E_0^{(0)} - E_n^{(0)}}}_{{\Lambda}}(g_L \mu_B \vec{B} + \lambda \hat{\vec{S}})%
    \end{align}
    % \label{eq:}
\end{equation}

Here $\Lambda$ is a matrix composed of the elements as shown. Expanding out, this allows us to write the second order perturbation as 
\begin{equation}
    \sum_{n}\frac{\bra{0} \hat{H}' \ket{n} \bra{n} \hat{H'}\ket{0}}{E_0^{(0)} - E_n^{(0)}} = g_L^2\mu_B^2 \vec{B} \cdot \Lambda \cdot \vec{B} + 2\lambda g_L\mu_B\hat{\vec{S}}\cdot\Lambda\cdot\vec{B} + \lambda^2 \hat{\vec{S}}\cdot \Lambda \cdot \hat{\vec{S}}.
    \label{eq:second_order}
\end{equation}

Since for EPR we are only interested in the spin-dependent terms, the first term may be neglected as it represents a global shift in the energy spectra. 

\subsubsection{Combined Perturbation}
Combining \eqref{eq:first_order} and \eqref{eq:second_order} we find 
\begin{equation}
    \begin{align}
    \bra{0} \hat{H}' \ket{0}+  \sum_{n}\frac{\bra{0} \hat{H}' \ket{n} \bra{n} \hat{H'}\ket{0}}{E_0^{(0)} - E_n^{(0)}}
    &= g_S \mu_B \hat{\vec{S}} \cdot \vec{B} + 2\lambda g_L\mu_B\hat{\vec{S}}\cdot\Lambda\cdot\vec{B} + \lambda^2 \hat{\vec{S}}\cdot \Lambda \cdot \hat{\vec{S}} \\ 
    &=\mu_B \hat{\vec{S}} \cdot \underbrace{(g_S + 2g_L \lambda \Lambda)}_{g} \cdot \vec{B} + \hat{\vec{S}}\cdot \underbrace{\lambda^2 \Lambda}_{D} \cdot \hat{\vec{S}}
    \end{align}
    \label{eq:}
\end{equation}
In this expression $g$ and $D$ are matrix quantities depending on $\Lambda$ and represent the (possibly anisotropic) $g$ factor and $D$ the fine structure splitting. 

For this work we will consider only systems in which the differences in angular momentum is due only to the spin and thus $g$ is reduced to a scalar quantity in the spin Hamiltonian.\td{More like - we consider $g$ to be isotropic and symmetric} 

The term depending on $D$ has no dependence on magnetic field and thus this \index{fine-structure}{fine-structure} splitting is known as zero field splitting (ZFS) and is observed in systems with $S > 1/2$. 
\begin{equation}
    H_{\ce{FS}} = \hat{\vec{S}}\cdot D \cdot \hat{\vec{S}}. 
    \label{eq:fine_structure}
\end{equation}







