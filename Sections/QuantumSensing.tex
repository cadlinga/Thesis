\section{Quantum Sensing}
\index{quantum sensing}{Quantum sensing} involves using a qubit system acting as a quantum sensor that interacts with an external variable of
interest, such as a magnetic field, electric field, strain or acoustic wave, or temperature \cite{Castelletto_2024}. 

Quantum sensors have a higher \index{sensitivity}{sensitivity} within a nanoscale volume compared to a fully classical counterpart which would require higher field densities or higher volume interrogation to be effective. 
They detect such weak signals by exploiting quantum coherence or quantum entanglement
\cite{Kin2021}
.
For this work we will study spins states which acquire phase shifts from interactions with specific parameters which may be detected via an optical interface (see \ref{ODMR}) through spin to photon conversion
\cite{Wolfowicz2021}
. 


\subsection{Qubits}
A \index{qubit}{qubit} system is, in the simplest terms, as a two-level system. The power of a  lies in quantum coherence and/or
temporal superposition of quantum states which allow for computation or manipulation with no classical analogue before being collapsed back to a measurement basis. 

\subsection{\index{DiVincenzo criteria}{DiVincenzo Criteria}}
To construct a working quantum sensor with any candidate system, DiVincenzo and Degen outlined a set of three necessary conditions that must be followed \cite{Crawford2021, RevModPhys.89.035002, DiVincenzo1995}

\begin{enumerate}
    \item The quantum system must have discrete resolvable energy levels (or an ensemble of two-level systems with a lower energy state $\ket{0}$ and an upper energy state $\ket{1}$) that are separated by a finite transition energy. 
    \item It must be possible to initialise the quantum sensor into a well-known state and to read out its state. 
    \item The quantum sensor can be coherently manipulated, typically by time-dependent fields.
\end{enumerate} 

The colour centres in SiC satisfy the first criteria and the divacancies and Silicon vacancy are $S=1$ and $S=3/2$ respectively. 

Within the context of this work, intitialisation is described in \ref{spin_polarisation}. 

We will discuss in detail how the spin states are coherently manipulated by the influence of external fields and how the effect of the influence may be measured. 
 % Based on magnetic-dipole forbidden spin transitions, this scheme enables spatially confined spin control, the imaging of GHz-frequency electric fields, and the characterization of defect spin multiplicity
\cite{PhysRevLett.112.087601}

% Overcoming poor readout is an increasingly urgent challenge for devices based on solid-state spin defects, particularly given their rapid adoption in quantum sensing, quantum information, and tests of fundamental physics. 
% Our results pave a clear path to achieve unity readout fidelity of solid-state spin sensors through increased ensemble size, reduced spin-resonance linewidth, or improved cavity quality factor.


 % We report the realization of nanotesla shot-noise-limited ensemble magnetometry based on optically detected magnetic resonance with the silicon vacancy in 4⁢𝐻 silicon carbide. By coarsely optimizing the anneal parameters and minimizing power broadening, we achieve a sensitivity of 50nT/√Hz and a theoretical shot-noise-limited sensitivity of 3.5nT/√Hz.
\cite{PhysRevApplied.15.064022}

% By inserting an NV-doped diamond membrane between two ferrite cones in a bowtie configuration, we realize a ∼250-fold increase of the magnetic field amplitude within the diamond. We demonstrate a sensitivity of ∼0.9⁢pTs1/2 to magnetic fields in the frequency range between 10 and 1000Hz. 
\cite{PhysRevResearch.2.023394}

% For all these color centers we saw an enhancement of the photostable fluorescence emission of at least a factor of 6 using micro-photoluminescence systems. Using custom confocal microscopy setups, we characterized the emission of VSi measuring an enhancement by up to a factor of 20, and of NCVSi with an enhancement up to a factor of 7. The experimental results are supported by finite element method simulations. Our study provides the pathway for device design and fabrication with an integrated ultra-bright ensemble of VSi and NCVSi for in vivo imaging and sensing in the infrared.
\cite{Castelletto2019}

 % low photon count rate significantly limits their applications. We strongly enhanced the brightness by 7 times and spin-control strength by 14 times of single divacancy defects in 4H-SiC membranes using a surface plasmon generated by gold film coplanar waveguides.
\cite{Zhou2023}

