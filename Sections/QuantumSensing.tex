\section{Quantum Sensing}
Quantum sensing involves using a qubit system acting as a quantum sensor that interacts with an external variable of
interest, such as a magnetic field, electric field, strain or acoustic wave, or temperature \cite{Castelletto_2024}. 

Quantum sensors have a higher sensitivity within a nanoscale or microscale sampling volume compared to a fully classical counterpart which would require higher field densities or higher volume interrogation to be effective. 

% Quantum sensing, for example, uses the spin state to acquire a phase shift from interactions with the environment10, and an optical interface (that is, spin-to-photon conversion) allows optical readout of a spin qubit, potentially enhanced by spin-to-charge conversion
\cite{Wolfowicz2021}

% Quantum sensors detect weak physical signals in nanoscale by quantum coherence, quantum properties or quantum entanglement. 
\cite{Kin2021}


% Unlike heritage designs, the magnetometer does not require inductive sensing elements, high frequency radio, and/or optical circuitry and can be made significantly more compact and lightweight,
%....
%Additionally, the robustness of the SiC semiconductor allows for operation in extreme conditions
\cite{Cochrane2016}


% We achieve two-spin interference with a phase sensitivity of 1.79 ± 0.06 dB beyond the SQL and three-spin interference with a phase sensitivity of 2.77 ± 0.10 dB. Besides, a magnetic sensitivity of 0.87 ± 0.09 dB beyond the SQL is achieved by two-spin interference for detecting a real magnetic field. Particularly, the deterministic and joint initialization of NV negative state, NV electron spin, and two nuclear spins is realized at room temperature. The techniques used here are of fundamental importance for quantum sensing and computing, and naturally applicable to other solid-state spin systems.
\cite{Xie2021}

% \paragraph{Qubits}
% A quantum bit or qubit is a simple quantum mechanical system with two eigenstates. 

% DiVincenzo Criteria
\cite{DiVincenzo1995}


\subsection{DiVincenzo Criteria}
\cite{RevModPhys.89.035002}
% To construct a working quantum sensor with any candidate material, DiVincenzo[39] and Degen[6] outlined a set of three necessary conditions that must be followed: i) The quantum system must have discrete resolvable energy levels (or an ensemble of two-level systems with a lower energy state |0⟩ and an upper energy state |1⟩) that are separated by a finite transition energy; ii) it must be possible to initialize the quantum sensor into a well-known state and to read out its state; iii) the quantum sensor can be coherently manipulated, typically by time-dependent fields.
\cite{Crawford2021}

\subsection{Crystal Defects}
% Because of about 250 SiC polytypes are known, there should exist more than thousand different spin defects in SiC with distinct characteristics14,15. One can select defects with the most suitable properties for a concrete task, which is not possible for one universal sensor.

\cite{Kraus2014}

% Spin defect centers with long quantum coherence times (T2) are key solid-state platforms for a variety of quantum applications. 
\cite{Kanai2022}

\subsubsection{Quantisation}
\subsubsection{Polarisation}
 % Based on magnetic-dipole forbidden spin transitions, this scheme enables spatially confined spin control, the imaging of GHz-frequency electric fields, and the characterization of defect spin multiplicity
\cite{PhysRevLett.112.087601}

\subsubsection{Coherent Manipulation}
\cite{Widmann2014}

% We find that simultaneous optical reionization and qubit manipulation can be carried out at room temperature with photoexcitation at the typical excitation wavelength used for readout of the divacancy qubits in 4H SiC
\cite{PhysRevB.105.165108}

% These spin defects can be optically addressed and coherently controlled even at room temperature, and their fluorescence spectrum and optically detected magnetic resonance spectra are different from those of any previously discovered defects. Moreover, the generation of these defects can be well controlled by optimizing the annealing temperature after implantation. These defects demonstrate high thermal stability with coherently controlled electron spins, facilitating their application in quantum sensing and masers under harsh conditions.
\cite{Yan2020}


% These defects are optically active near telecommunication wavelengths11, and are found in a host material for which there already exist industrial-scale crystal growth12 and advanced microfabrication techniques13. In addition, they possess desirable spin coherence properties that are comparable to those of the diamond nitrogen–vacancy centre. This makes them promising candidates for various photonic, spintronic and quantum information applications that merge quantum degrees of freedom with classical electronic and optical technologies
\cite{Koehl2011}

% Coherent manipulation of NV centers in SiC has been achieved with Rabi and Ramsey oscillations. 
\cite{Mu2020}


% Hence, coherent control of NV center spins is achieved at room temperature, and the coherence time 𝑇2 can be reached to around 17.1  𝜇⁢s. Furthermore, an investigation of fluorescence properties of single NV centers shows that they are room-temperature photostable single-photon sources at telecom range.
\cite{PhysRevLett.124.223601}

\subsubsection{Efficient Readout}
% Overcoming poor readout is an increasingly urgent challenge for devices based on solid-state spin defects, particularly given their rapid adoption in quantum sensing, quantum information, and tests of fundamental physics. 
% Our results pave a clear path to achieve unity readout fidelity of solid-state spin sensors through increased ensemble size, reduced spin-resonance linewidth, or improved cavity quality factor.

\cite{Eisenach2021}

% we demonstrate the first ever implementation of SCC for VV0 in SiC by performing spin-selective ionization followed by all-optical single-shot readout of the charge state. Using this technique, we can determine an initially prepared spin state with over 80% fidelity. 
\cite{Anderson2022-sf}


% Here, we demonstrate a photo-electrical detection technique for electron spins of silicon vacancy ensembles in the 4H polytype of silicon carbide (SiC). Further, we show coherent spin state control, proving that this electrical readout technique enables detection of coherent spin motion. Our readout works at ambient conditions, while other electrical readout approaches are often limited to low temperatures or high magnetic fields.
\cite{Niethammer2019}


% High-fidelity single-shot spin readout in silicon opens the way to the development of a new generation of quantum computing and spintronic devices, built using the most important material in the semiconductor industry.
\cite{Morello2010}

% Even if the signal-to-noise ratio is reasonably low, a well-trained convolutional neural network (CNN) can predict the resonance peaks of ODMR spectra or the period of Rabi oscillations. Because of the fast output of predictions by the CNN, this method can be used to sense the magnetic field in the environment and microwave intensities in real time.
\cite{PhysRevApplied.17.034046}


\subsection{Coherence}
\cite{Christle2014},\cite{Soltamov2019}, \cite{Gilardoni2020} \cite{BulanceaLindvall2021}, \cite{Astner2022}

% Long coherence times are key to the performance of quantum bits (qubits). 
\cite{Seo2016-ed}

\subsubsection{Spin Relaxation}
\subsubsection{Dephasing}
\subsubsection{Hahn Echo}
% The coherence time of NV defects in the presence of noise originating from parasitic spins located at the diamond interface can be improved by orders of magnitude by using high-order spin echoes.
\cite{Wu2016}
\subsubsection{Example: NV Diamond}


\subsection{Sensitivity}
\cite{RevModPhys.92.015004}

% Our approach is suitable for ensemble as well as single spin-3/2 color centers, allowing for angle-resolved magnetometry on the nanoscale at ambient conditions.
\cite{PhysRevApplied.4.014009}


 % We report the realization of nanotesla shot-noise-limited ensemble magnetometry based on optically detected magnetic resonance with the silicon vacancy in 4⁢𝐻 silicon carbide. By coarsely optimizing the anneal parameters and minimizing power broadening, we achieve a sensitivity of 50nT/√Hz and a theoretical shot-noise-limited sensitivity of 3.5nT/√Hz.
\cite{PhysRevApplied.15.064022}

% By inserting an NV-doped diamond membrane between two ferrite cones in a bowtie configuration, we realize a ∼250-fold increase of the magnetic field amplitude within the diamond. We demonstrate a sensitivity of ∼0.9⁢pTs1/2 to magnetic fields in the frequency range between 10 and 1000Hz. 
\cite{PhysRevResearch.2.023394}


% For all these color centers we saw an enhancement of the photostable fluorescence emission of at least a factor of 6 using micro-photoluminescence systems. Using custom confocal microscopy setups, we characterized the emission of VSi measuring an enhancement by up to a factor of 20, and of NCVSi with an enhancement up to a factor of 7. The experimental results are supported by finite element method simulations. Our study provides the pathway for device design and fabrication with an integrated ultra-bright ensemble of VSi and NCVSi for in vivo imaging and sensing in the infrared.
\cite{Castelletto2019}

 % low photon count rate significantly limits their applications. We strongly enhanced the brightness by 7 times and spin-control strength by 14 times of single divacancy defects in 4H-SiC membranes using a surface plasmon generated by gold film coplanar waveguides.
\cite{Zhou2023}

\subsection{DiVincenzo Criteria}
To construct a working quantum sensor with any candidate system, DiVincenzo and Degen outlined a set of three necessary conditions that must be followed \cite{Crawford2021, RevModPhys.89.035002}
: 
\begin{enumerate}
    \item The quantum system must have discrete resolvable energy levels (or an ensemble of two-level systems with a lower energy state $\ket{0}$ and an upper energy state $\ket{1}$) that are separated by a finite transition energy. 
    \item It must be possible to initialise the quantum sensor into a well-known state and to read out its state. 
    \item The quantum sensor can be coherently manipulated, typically by time-dependent fields.
\end{enumerate} 

Spin defects are mostly paramagnetic and radiative point defects (or colour centres). Colour centres possessing a 
non-zero electron spin are excellent candidates for optical spin
quantum bits (qubits) \cite{Castelletto_2024}.

Colour centres can produce detectable luminescence even at room temperature. Optical radiation is generally used as a readout but the excitation can also be used for spin manipulation and control.

\begin{group}
\color{lightgray}
Most of the SiC colour centres have a residual spin and
therefore all could be in principle used in quantum sens-
ing. However, they can be distinguished and grouped by their
ground state spin value and the zero field (magnetic) split-
ting (ZFS), which defines their properties and the different
methods for their initialisation, control, and read-out. Colour
centres with the high-spin ground state (S = 1, 3/2) can be
used as two or three levels quantum systems (figure 1(c)). They
can be controlled optically and using a microwave (MW) or
radio frequency (RF) excitation due to the higher sensitivity to
the presence of the magnetic field.


The spin Hamiltonian of an S = 3/2 elec-
tron spin defect within a nuclear spin bath can be written as:

\begin{equation}
    \hat{H} = \underbrace{g \mu_B \hat{\mathbf{S}} \cdot \mathbf{B}_0}_{H_{\text{Zeeman}}} + D \left( \hat{\mathbf{S}}_z^2 - \frac{S(S+1)}{3}\right) + E (\hat{\mathbf{S}}^2_x - \hat{\mathbf{S}}^2_y) + \sum_j \hat{\mathbf{S}}_i \cdot \mathbf{A}_{ij} \cdot \hat{\mathbf{I}}_j
    \label{eq:}
\end{equation}
where g is the isotropic centre specific Lande factor (g =
2.0028), µB is the Bohr magneton, B0 is the external magnetic
field, D and E account for the zero magnetic fields splitting
for the axial (along the spin polarisation axis) or the off-axis
component of the spin defect operator Ŝ = (Sˆx , Sˆy , Ŝz ) respect-
ively. Aij is the hyperfine tensor that describes the central spin
coupling to many nuclear spins indexed by j with spin operat-
ors Iˆj .
\end{group}


