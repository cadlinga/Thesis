\begin{center}
%\vspace*{2in}
% an acknowledgements section is completely optional but if you decide
% not to include it you should still include an empty {titlepage}
% environment as this initialises things like section and page numbering.
\textbf{Acknowledgements}
\end{center}
%
% \emph{Put your acknowledgements here. Thanking your supervisor for
% his/her help is standard practice, but it's not compulsory \ldots}
%
% I'd like to thank my supervisor Professor Carole Ann O'Malley for
% making this project possible, and her PhD student Jack O'Bean for his
% patience and his detailed functional explanations of how classical
% symmetries can be broken by quantum effects. Thanks also to Wally Bee
% and Ken Garoo of the University of Woolloomooloo for sending me
% their higher-order hopping-parameter expansions.
%
% Finally, none of this would have been possible without financial
% support from Paterson's Lane Education Committee.
%
% \bigskip
%
% This document has its origins in the dissertation template for the MSc
% in High Performance Computing, which is apparently descended from a
% template developed by Professor Charles Duncan for MSc students in
% Meteorology. His acknowledgement follows:
%
% \emph{This template has been produced with help from many former
%   students who have shown different ways of doing things. Please make
%   suggestions for further improvements.}
%
% Some parts of this template were lifted unashamedly from the Edinburgh
% MPhys project report guide, with little or no modification. I have no
% idea who wrote the first version of that\ldots
%
% You don't have to use \LaTeX\ for your dissertation. You can use
% Microsoft Word, Apple Pages, LibreOffice or similar, but it's
% \emph{much} easier to typeset equations in \LaTeX, and references look
% after themselves. Whatever you use, your dissertation should have the
% general structure of this template, and it should look similar --
% especially the front page.
%

