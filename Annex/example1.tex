\chapter{Stuff that's too detailed}

Appendices should contain all the material which is considered too
detailed to be included in the main body of the text, but which is
important enough to be included in the thesis.

Perhaps this is a good place to mention \BibTeX.

You can do references in the simple way explained in the introduction,
or you can use \BibTeX.


\section{\BibTeX}
\label{sec:bibtex}

It is convenient to use \BibTeX\ to compile your bibliography.  First
you need to create a .bib file e.g.  you may call it ref.bib Then you
can put all your references into the file with entries such as
\begin{verbatim}
@Book{ob:bornwolf,
     author = "Born, M and Wolf, E",
     title  = "Principles of Optics",
     publisher = "Cambridge University Press",
     year = 1999,
     edition = {7th},
}

@Article{jr:ashkin,
Author = {A. Ashkin and J.M. Dziedzic and J.E. Bjorkholm and S. Chu},
Title = "Observation of a single beam gradient force optical tap for 
dielectric particles",
Journal = "Optics Letters",
Volume = 11,
Pages = "288-290",
Year = 1986}

@INPROCEEDINGS{seger,
 author = {J. Seger and H.J. Brockman},
 title = {What is bet-hedging?},
 editors={P.H. Harvey and L. Partridge},
 booktitle = {Oxford Surveys in Evolutionary Biology},
 year={1987},
 page={18},
 publisher={Oxford University Press},
 place={Oxford}}
\end{verbatim}
for a book, an article in a journal or an article in a proceedings volume
respectively.

Inside your \LaTeX\ file
you should include 
\begin{verbatim}
\bibliographystyle{unsrt}                      
and
\bibliography{ref}
\end{verbatim}
The first command determines the reference style, here plain and 
unsorted. With this referencing style 
a numerical referencing system (which is now the most
common in physics literature) is used and the numbering of references
will be the order in which they appear in the document. Alternatively, 
you could use
a customised `style file' but there is no real need.  The second
command just inputs your .bib file Note that only the references cited
in the text will appear in the bibliography so you can have spare
references in your .bib file.


You use the name you have given to an entry (e.g.
for the book example above the name is ob:bornwolf)
to cite the relevant article
by using the cite command in your \LaTeX\ file e.g. 
\begin{verbatim}
\cite{ob:bornwolf}
\end{verbatim}


\section{Producing your documents using \texttt{pdflatex}}

To use pdflatex your figures need to be in pdf format.  You can convert almost any image file to pdf using \texttt{convert}.  e.g. \texttt{convert myfigure.png myfigure.pdf}.

The first time you should type:
\begin{verbatim}
  pdflatex ProjectReport
  bibtex ProjectReport
  pdflatex ProjectReport
  pdflatex ProjectReport
\end{verbatim} 
This first time you run\texttt{pdflatex} it will produce a
\texttt{ProjectReport.aux}.  The \BibTeX\ command reads in the
bibliography file and makes the files \texttt{ProjectReport.bbl} and
\texttt{ProjectReport.blg} files.  These files are read in the next
\texttt{pdflatex} command, but you'll still have ``undefined
cross-reference'' errors which are sorted out by the last
\texttt{pdflatex} command.

Subsequently, you should only need to do one (or two)
\texttt{pdflatex}s, or \texttt{pdfbibtex} followed by
\texttt{pdflatex} twice if you change any references.

\vspace{5mm} You may also use plain \texttt{latex} instead of
\texttt{pdflatex}.  This requires you to use postscript graphics
instead of pdf.





