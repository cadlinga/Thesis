\begin{center}
\textbf{Personal Statement}
\end{center}

\emph{You \textbf{\emph{must}} include a Personal Statement in your
  dissertation. This should describe what you did during the project,
  and when you did it. Give an account of problems you faced and how
  you attempted to overcome them. The examples below are based on
  personal statements from MSc and MPhys projects in previous years,
  with (mostly-obvious) changes to make them anonymous. }

% \subsubsection*{Example~1: an analytical project}
%
% The project began with an introduction to the spinor-helicity
% formalism in four dimensions, with my main source material being
% H. Elvang's “Scattering Amplitudes in Gauge Theory and Gravity” [1]. I
% read the first chapter, and acquainted myself with the formalism,
% and how it worked in a practical sense.
%
% Once I felt more comfortable with it, we moved onto the
% six-dimensional spinor-helicity formalism paper, where I spent some
% time gaining as strong an understanding of how the formalism worked,
% and proving identities.
%
% The next stage was to learn about the generalised unitarity procedure,
% with the end goal being to use it to calculate coefficients for some
% one loop integral, likely involving massive particles. Learning how
% this worked took some time, and proved to be some of the most
% difficult material for me to understand.
%
% It wasn't until later that we began to consider applying what I had
% learned to a Kaluza-Klein reduction, which ended up being the main
% focus of the project. It mixed well with the general theme of
% “extra-dimensional theory” the project began with, and allowed me to
% apply all that I'd learned and prepared for so far.  The vast majority
% of my remaining time was spent calculating coefficients for the scalar
% box contribution to the gluon-gluon to two-Kaluza-Klein-particle
% amplitude, overcoming a number of problems and errors, to finally have
% human-readable, and presentable results.
%
% During the course of the project, I met with my supervisor every week,
% in order to discuss my progress and the direction I would head
% next. Toward the end, the frequency of our meetings increased
% somewhat, as I began to finish my calculations.
%
% I started writing this dissertation in mid-July, and I spent the first
% three weeks of August working on it full-time.
%
% Overall, I feel that the project was a success, and I found it to be
% extremely enjoyable throughout.
%
%
% \subsubsection*{Example~2: a computational project}
%
% I spent the first 2 weeks of the project reading the material
% surrounding my project - mainly [1] and [2]. I also began to plan out
% how I would implement the algorithms in C++, in doing this I gained an
% understanding of what the main goals of the first half of my project
% would be and how they could be achieved. I identified which Monte
% Carlo observables would be useful to measure in these simulations.
%
% For the next 3 weeks I implemented the standard Atlantic City
% algorithm and debugged my code whilst developing analysis tools in
% python. I compared the results from my simulations to the results from
% [3] (for the Random Osculator) and [4] for the EvenMoreRandom
% Osculator. Having obtained positive results for the Random Osculator I
% started reading up on Heaviside Articulation. I examined how to
% integrate a Heaviside Articulator into the simulation in order to
% produce the most efficient simulation - the solution I decided on was
% to use a package called HeaviArt[5].
%
% Following this I began to integrate the Heaviside Articulator into my
% code and test it against the regular algorithm. In addition to this I
% ran longer simulations to verify my findings without Articulation.
%
% In mid July I finished implementing Heaviside Articulation into my
% code and began looking into how to quantify any improvement in speed
% gained by this algorithm. As July progressed I started looking into
% how to integrate the EvenMoreRandom Osculator into my code - this was
% the most complicated part of the project, as discussed in the body of
% this report. Despite much effort on my part, I couldn't get the
% results produced by the new algorithm to agree with the old
% ones. Following further study of the literature, and long discussions
% with Jack O'Bean, it turned out that the original form of Heaviside
% Articulation didn't applied to the EvenMoreRandom Osculator. With the
% help of Jack and my supervisor, I then developed the new version
% described in this report. I also did analytical calculations of the
% four-point blue function to two orders higher than had
% been published previously in the literature.
%
% For the final parts of the summer I worked mainly on perfecting the
% algorithm for the Random Osculator and implementing the EvenMoreRandom
% Osculators algorithm with the improved Heaviside Articulation. The
% final results were encouraging, but more work is clearly needed. To
% this end, I have been awarded a studentship by the British University
% of Lifelong Learning to extend this work during my PhD Studies at
% the non-existent Scottish Highlands Institute of Technology in
% Inveroxter.
%
% I started writing this dissertation in mid-July, and I spent the first
% three weeks of August working on it full-time.
%
%
% \subsubsection{Example~3: a more mathematical project}
%
% My first two weeks of work on the project consisted of building up a
% working knowledge of the algebraic structures and techniques which
% would be used in the main calculations which were to be carried out,
% in particular carefully reading up on the 11-dimensional case [1, 2],
% the general structure of the Spencer complex of the Poincar\'e
% superalgebra, the Clifford and exterior algebras over a Lorentzian
% vector space in general finite dimension and the relationship between
% them, the spin group, the Lorentz algebra $so(V$) and their vector and
% spinor representations. As a toy calculation, I computed the first two
% Spencer cohomology groups of the Poincar\'e algebra and its relevant
% subalgebras in order to find its filtered (sub)deformations, finding
% that these are given by a space of algebraic curvature operators. This
% calculation is vastly simplified compared to the superalgebra
% calculation owing to the lack of spinor structure.
%
% In the following two weeks, I turned to the particular case of 5
% dimensions, initially study- ing the quaternionic structure of the
% relevant Clifford algebra, its spinor representation and symplectic
% Majorana spinors. I read up on complex and quaternionic structures and
% representations, proved various identities for products of higher-rank
% gamma matrices found in [5] and derived the Fierz identity and various
% other useful identities involving quantities defined in terms of the
% Majorana spinors. I made an initial attempt to solve the relevant
% cocycle conditions, initially finding that the space of solutions was
% given by 3-forms (or Hodge-dually 2-forms), but from the known spinor
% connection in D = 5 supergravity we knew that there was a problem with
% the solution. Working with my supervisor to fix the errors in this
% calculation, finding the most efficient way of setting it out and
% writing up the work done so far and some of the background material
% took up the following two to three weeks.  Next, I moved on to
% figuring out the fundamentals of the geometric part of the project,
% understanding spin structures on manifolds, the spin lift of the
% Levi-Civita connection, the spinorial Lie derivative and the
% definition of the superconnection. I derived the flatness conditions
% for the curvature of the superconnection and showed that these
% conditions are sufficient to cause the Killing superalgebra to close
% and form a Lie superalgebra. This work, along with further writeup,
% took another two weeks.
%
% The final weeks of the project were spent learning about Lorentzian
% and Riemannian symmetric spaces, finding all of the the maximally
% supersymmetric backgrounds, on the way learning a little about the
% supersymmetric solutions of 5-dimensional supergravity, and finally
% finishing the dissertation.
%
\newpage
