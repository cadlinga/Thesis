\topskip0pt
\vspace*{\fill}
\begin{center}
\textrm{\bfseries\Huge Abstract}%
\end{center}%
    % the context or background information for your research; the general topic under study; the specific topic of your research
Spin based sensing, particularly with silicon carbide is a growing and exciting field. A great deal of work has gone into developing complex sensing regimes to enable the detection of the magnetic and electric fields, strain, pressure and temperature with nanoscale sensors. 
    % the central questions or statement of the problem your research addresses
This work provides a detailed overview of how that sensing is achieved across both $S=1$ and $S=3/2$ systems, particularly with reference to colour centres in Silicon Carbide. 
    % what’s already known about this question, what previous research has done or shown
Whilst sensing in one mode i.e. detecting one parameter in isolation is well understood, borrowing a lot from the work done with the Diamond Nitrogen vacancy, there is not a great deal of work on where these devices may be used to simultaneously measure multiple parameters. 
    % the main reason(s), the exigency, the rationale, the goals for your research—Why is it important to address these questions? Are you, for example, examining a new topic? Why is that topic worth examining? Are you filling a gap in previous research? Applying new methods to take a fresh look at existing ideas or data? Resolving a dispute within the literature in your field? . . .
Multimodal sensors could have applications ranging from the laboratory to space and life sciences. We first provide a detailed account of methods already used for single mode sensing. Each schema is given a clear summary which describes the limitations of its implementation. We then attempt to combine the schemas to see which are compatible.
%
    % your research and/or analytical methods
    % your main findings, results, or arguments
We propose four possible combinations of modes and suggest how future work may resolve at least two more combinations. This research thus allows for the implementation of the multimodal techniques by combining existing techniques and may help to direct future research to further develop the field of multimodal sensing. 
    % the significance or implications of your findings or arguments.
\vspace{1em}
% \todo[inline, color=red]{Need to write an abstract}
% \begin{center}
%     {\color{figcaption}\rule{0.75\textwidth}{0.1pt}}
% \end{center}

\vspace*{\fill}
\newpage
\newpage
